\chapter{Math: Group Theory}\label{ch:group}
This chapter focuses on some the basic ideas of group theory and some 
applications of group theory to physics. Groups appear often in physics:
Boosts and rotations in special relativity 
generate the Lorentz group, sets of gauge transformations equipped with 
function composition form gauge groups, etc. Some of this presentation
follows sections of Dummit and Foote~\cite{dummit_abstract_2004} and 
Georgi~\cite{georgi_lie_1999}.


\section{Preliminaries}\label{sec:gpprelim}
In order for us to understand any of the above paragraph, we must lay down a
few definitions and mathematical preliminaries.
A {\it binary operation} \index{binary operation} $\bullet$ on a set 
$G$ is a function $\bullet : G\times G\to G$. A {\it group} \index{group}, 
then, is a set $G$ equipped with a binary operation $\bullet$ that satisfies 
the following axioms:
  \begin{enumerate}
    \item $\bullet$ is associative.
    \item $\Exists \id \in G$ such that $\Forall g \in G$, 
          \begin{equation}
            \id\bullet g=g\bullet \id=g. 
          \end{equation} This element $\id$ is called the {\it identity}.
          \index{identity}
    \item $\Forall g \in G\ \ \Exists g^{-1} \in G$, called the 
          {\it inverse} \index{inverse} of $g$, such that
          \begin{equation}
            g^{-1}\bullet g=g\bullet g^{-1}=\id.
          \end{equation}
  \end{enumerate}
If group elements commute under $\bullet$ the group is said to be 
{\it abelian} \index{abelian}. The {\it order} \index{order} of a group, 
denoted $|G|$, is the number of unique elements in the group\footnote{This
is at least true for finite groups.}. A {\it subgroup} \index{subgroup}
$H$ of $G$ is a non-empty subset of $G$ that itself forms a group under 
$\bullet$ and in this case we will write $H\leq G$. 
(It should be clear from context whether this symbol indicates group 
organization or magnitude.) Finally a group is {\it cyclic} \index{cyclic}
if it is generated by a single element; that is, if $\Exists g\in G$ such that 
$G=\{g^{n}:  n\in\Z\}$.

It's actually not too common for mathematicians or physicists to write the
$\bullet$ explicitly when showing the composition of two elements. So for
example you will often see $gh$ as shorthand for $g\bullet h$. In general I will
only refer to operations on algebraic structures explicitly when giving the
definition of that structure. Therefore you can expect to see $gh$ instead of
$g\bullet h$ from here on out.

\begin{proposition}{}{}
  A subset H of G is a subgroup of G if and only if
  $$a,b\in H\Rightarrow ab^{-1}\in H$$
  \begin{proof}
    ($\Rightarrow$) Follows immediately from the definition of a subgroup. To 
    show ($\Leftarrow$) let $b\in H$. Then by the above conditional, $bb^{-1}
    \in H$, which shows $\id\in H$. To show the existence of inverses in $H$, 
    note $\id,b\in H\Rightarrow \id b^{-1}\in H\Rightarrow b^{-1}\in H$. 
    Finally, associativity is inherited from G.
  \end{proof}
\end{proposition}

\begin{example*}{}{}
\begin{enumerate}
  \item $\Z,\ \Q,\ \R$, and $\C$ are all groups
        under addition, and
        \begin{equation}
          \Z \leq \Q \leq \R \leq \C.
        \end{equation}
        Each of these sets with 0 removed forms a group under
        multiplication. (We have to remove 0 because it has no multiplicative
        inverse.)
  \item Let $n\in \N$ and define an equivalence relation on 
        $\Z$ by 
        \begin{equation}
          a\equiv b\ (mod\ n)\ \Leftrightarrow\ n\ |\ (b-a).
        \end{equation}
        (We read this as ``$a$ is congruent to $b$ modulo $ n$" or ``$a$ is
        congruent to $b$ mod $n$.") Define an equivalence class by 
        \begin{equation}
          \bar{a}\coloneqq\{a+mn\ : m\in \Z\}.
        \end{equation}
        The set of all such equivalence classes is called {\it the integers 
        modulo n}\index{modular arithmetic} and is denoted by 
        $\Z/n\Z$ or $\Z_n$. It forms a group 
        under the ``addition" operation exemplified below. As a concrete 
        illustration, take $\Z_4$. It is of order 4, with 
        elements $\bar{0},\ \bar{1},\ \bar{2}$, and $\bar{3}$. To see how 
        addition works, note
        \begin{equation}
          \label{eq:chgtmaad1}
          \begin{aligned}
            \bar{1}+\bar{2}&=\{(1+2)+4(m_{1}+m_{2})\ : m_{1},m_{2}
                             \in \Z\}\\
                         &=\{3+4m\ : m\in \Z\}\\
                         &=\bar{3}
          \end{aligned}
        \end{equation}
        and
        \begin{equation}
          \label{eq:chgtmaad2}
          \begin{aligned}
            \bar{3}+\bar{3}&=\{(3+3)+4(m_{3}+m_{4})\ : m_{3},m_{4}
                             \in \Z\}\\
                         &=\{2+4(1+m_{3}+m_{4})\ : m_{3},m_{4}
                             \in \Z\}\\
                         &=\{2+4m\ : m\in \Z\}\\
                         &=\bar{2}
          \end{aligned}
        \end{equation}
        It should be clear from the above that $\bar{0}$ is the identity
        element and $\Z_n$ is abelian and cyclic. I would
        also like to emphasize that the addition defined in 
        eqs.~\eqref{eq:chgtmaad1} and \eqref{eq:chgtmaad2} is not the same
        addition as over the integers, even though I have chosen the same
        symbol for both cases. One should always be careful of what group
        operation is meant when the author is being lazy. 
  \item Sets of objects besides numbers also form groups. For example let $V$ 
        be any non-empty set of objects and let $S_{V}$ be the set of all
        permutations of $V$. Then $S_{V}$ forms a group under function
        composition called the {\it symmetric group}\index{symmetric group} 
        on the set $V$.
\end{enumerate}
\end{example*}

In many situations, one encounters two groups of the same order that behave in
essentially the same ways. In a sense, one group is the same the original group,
masquerading about as a unique mathematical object. We could recover the
original group by merely relabeling elements of the masked group and viewing
its operation differently. Let us make these ideas more precise.
Let $G$ and $H$ be groups with operations $\bullet$ and
$\circ$, respectively. A map $\phi : G\to H$ satisfying
\begin{enumerate}
  \item $\phi(\id_G)=\id_H$ and
  \item $\phi(g \bullet h)=\phi(g)\circ\phi(h)$
\end{enumerate}
$\Forall g,h\in G$ is called a {\it homomorphism}.\index{homomorphism} 
If in addition to the above property we know that $\phi$ is a bijection, 
$\phi$ is said to be an {\it isomorphism} \index{isomorphism} of $G$ and 
$H$, we say that $G$ and $H$ are {\it isomorphic}, and we write $G\cong H$. 
Finally an {\it endomorphism} \index{endomorphism} is a homomorphism 
mapping a group to itself, and an {\it automorphism} \index{automorphism}
is a bijective endomorphism (isomorphism from a group to itself).

The above definition shows us that when $G\cong H$, $H$ is really just $G$ in
disguise. Since $\phi$ is a bijection we have a way of associating each element
of $H$ with exactly one element of $G$ and vice-versa (one could consider 
$\phi$ the relabeling), and the fact that $\phi$ is a homomorphism shows us
that the binary operations of the groups act the same way.

%A {\it field} is a set $F$ equipped with two binary
%  operations  $+$ and $\cdot$  satisfying the following properties:
%  \begin{enumerate}
%    \item $F$ equipped with $+$ is an abelian group. (Denote the 
%          identity element with respect to this operation 0. 0 is often 
%          referred to as the {\it additive identity}.)
%    \item $F-\{0\}$ equipped with $\cdot$ is an abelian group. (Denote the
%          identity element with respect to this operation 1. As you may have
%          guessed, 1 is called the {\it multiplicative identity}.)
%    \item $a\cdot (b+c)=(a\cdot b)+(a\cdot c)\ \ \Forall a,b,c\in F$. This is
%          called the {\it distributive property}.
%  \end{enumerate}
%A {\it subfield} is a subset $A$ of $F$ whose elements still form a field.
%
%\begin{example}
%\leavevmode
%\begin{enumerate}
%  \item For any group $G$, $G\cong G$.
%  \item The rotational symmetries of a regular $n$-gon are isomorphic to
%        $\Z_n$.
%  \item $\Q$, $\R$, and $\C$ are all fields.
%  \item $\Z$ is not a field. 
%\end{enumerate}
%\end{example}

\section{Quotient groups}\label{sec:q}
Let's add some structure to groups. In this section we will discuss a way of
creating a new group by partitioning an old one.

  An {\it equivalence relation} \index{equivalence relation} $\sim$ on a 
  set $G$ is a binary operation that has the following properties 
  $\Forall x,y,z\in G$:
  \begin{enumerate}
    \item it is {\it reflexive}\index{reflexive}, i.e. $x\sim x$;
    \item it is {\it symmetric}\index{symmetric}, i.e. 
          $x\sim y\Leftrightarrow y\sim x$; and
    \item it is {\it transitive}\index{transitive}, i.e. $x\sim y$ and 
          $y\sim z$ $\Rightarrow$ $x\sim z$.
  \end{enumerate}
  Let $g\in G$. The set $\bar{g}=\{x\in G:x\sim g\}$ is called an {\it
  equivalence class}\index{equivalence class}.

So we see that equivalence relations are just generalizations of the =
sign. We actually use equivalence relations more than you probably realize.
For example when $x,y\in\R$ and $y=x+\epsilon$, where $\epsilon$ is
small, we often just write $y=x$. Strictly speaking, these two quantities
are not equal, but it's not hard to show that ``equal to first order in
$\epsilon$" defines an equivalence relation.

Let $H\le G$. The {\it left coset} \index{coset} of $H$ with respect to 
$G$ is the set
  $$ aH=\{ah:\ h\in H\}. $$
The {\it right coset} is defined similarly, but with $h$ and $a$
interchanged.
We are now going to see that cosets define equivalence classes. To begin with
let $H\le G$; $x,y\in H$; and define a binary relation $\sim$ by
\begin{equation}
  x\sim y \Leftrightarrow xy^{-1}\in H.
\end{equation}
\begin{proposition}{}{}
  $\sim$ is an equivalence relation.
  \begin{proof}
    We just need to check that it satisfies all the defining properties.
    \begin{enumerate}
      \item $xx^{-1}=\id$ $\Rightarrow$ $x\sim x$.
      \item $x\sim y$ $\Rightarrow$ $xy^{-1}\in H$ $\Rightarrow$
            $(xy^{-1})^{-1}\in H$ $\Rightarrow$ $yx^{-1}\in H$
            since $H$ is a group, so its elements have inverses.
            Hence $y\sim x$.
      \item $x\sim y$ and $y\sim z$ $\Rightarrow$ $xy^{-1}\in H$ and
            $yz^{-1}\in H$. Then since $H$ is closed under multiplication,
            $xy^{-1}yz^{-1}=xz^{-1}\in H$ $\Rightarrow$ $xz^{-1}\in H$ 
            $\Rightarrow$ $x\sim z$.
    \end{enumerate}
  \end{proof}
\end{proposition}
According to this definition, the equivalence classes of $G$ are the sets
$A\subset G$ satisfying $\Forall x,y\in A$ $xy^{-1}\in H$. To see why this is
important, fix $a\in A$. Then $\Forall x\in A$,
\begin{equation}
  \begin{aligned}
    x\sim a &\Rightarrow xa^{-1}\in H \\
            &\Rightarrow xa^{-1}=h,\ h\in H\\
            &\Rightarrow x=ha,\ h\in H\\
            &\Rightarrow A=Ha.
  \end{aligned}
\end{equation}
In other words, the equivalence classes are the right cosets of $H$! If we
instead defined the equivalence relation by $x\sim y$ $\Leftrightarrow$
$x^{-1}y\in H$, we would have found the equivalence classes to be the left
cosets of $H$.

We can now construct our new group. The group is a collection of cosets, but
it only forms a group if the cosets are special. In particular,
a subgroup $N\le G$ is said to be {\it normal}\index{normal} if 
$\Forall g\in G,$ $gN=Ng$. In this case we write $N\unlhd G$. The 
{\it quotient group}\index{quotient group} $G/N$ is the set of all cosets of 
$N$. We define an operation on $G/N$ by
\begin{equation}
  (xN)(yN)=x(Ny)N=x(yN)N=(xy)(NN)=(xy)N.
\end{equation}
Hence we see why it was so crucial that the subgroup be normal: It allowed
us to move the $y$ past $N$ in the second step, thus guaranteeing $G/N$ is
closed under this operation.
\begin{proposition}{}{}
  $G/N$ forms a group under the above operation.
  \begin{proof}
    This is pretty clearly a group because $G$ is. The identity is $\id N$,
    and each $xN$ has inverse $x^{-1}N$
  \end{proof}
\end{proposition}
\begin{proposition}{}{}
  $G/N$ forms a partition of $G$.
  \begin{proof}
    Let $g\in G$. We have to check that (1) $g$ belongs to some coset of $N$,
    and further that (2) $g$ doesn't belong to two different cosets.
    \begin{enumerate}
      \item Let $n\in N$ and define $g'=n^{-1}g$. Then
            $ng'=nn^{-1}g=g$, so $g\sim g'$.
      \item Let $x,y\in G$. If $g\sim x$ and $g\sim y$ then $x\sim y$ since
            $\sim$ is transitive. 
    \end{enumerate}
  \end{proof}
\end{proposition}
To summarize: If we find a normal subgroup $N$ of $G$, we can make a new group
$G/N$ by partitioning $G$ into disjoint cosets of $N$, which turn out to be
disjoint equivalence classes. These quotient groups are useful and pop up
pretty frequently; indeed the group $\Z/n\Z$, which we
encountered in the first section, is a quotient group. Now you understand the
notation.
\begin{example*}{}{}
\index{modular arithmetic}
  Take $G=(\Z,+)$ and $N=4\Z$. Clearly $4\Z$ is a
  normal subgroup of $\Z$ since it's commutative. To form the 
  quotient group $\Z/4\Z$, we find all cosets of $4\Z$.
  These are
  \begin{itemize}
    \item $4\Z$,
    \item $4\Z+1$=$\{4m+1:m\in\Z\}$,
    \item $4\Z+2$, and
    \item $4\Z+3$.
  \end{itemize}
  To check for instance that $4\Z+1$ is an equivalence class, we just
  need to show that $x,y\in4\Z+1\Rightarrow xy^{-1}\in4\Z$.
  Let $x=4m+1$ and $y=4n+1$ for some $m,n\in\Z$. Then
  $$xy^{-1}=x-y=4m+1-(4n+1)=4(m-n)\in4\Z.$$
  (This is an instance where the notation can be confusing; remember
  $\Z$ is being considered as a group under addition, so applying the
  inverse group operation means subtracting.) To see that there are no other
  cosets, note that $4\Z+4$=$4\Z$.
\end{example*}


\section{Group representations}

It turns out that many groups are isomorphic to sets of linear transformations
on vector spaces, and similarly that many sets of matrices equipped with matrix
multiplication form groups. In many cases these isomorphisms make dealing with
groups less abstract and more manageable; after all, everyone is comfortable
with matrices. In this section we will make these ideas precise, but we will
need to use linear algebra. For a very brief review of the relevant bits, see
the appendix.

The set of all automorphisms of a vector space $V$ is called
the {\it automorphism group}\index{automorphism group} of $V$ or the 
{\it general linear group}\index{general linear group} and is
denoted $GL(V)$. The set of all $n\times n$ matrices with entries from 
a field $F$ is called the {\it general linear group of degree $n$} and 
is denoted by $GL_{n}(F)$.
The reader may be concerned that the nomenclature
of the previous definition is poorly chosen. However if $V$ is a vector field
over the field $F$, the groups $GL(V)$ and $GL_{n}(F)$ are actually just two
ways of viewing the same thing. We shall demonstrate this fact with the
following theorem. 

\begin{theorem}{}{}
  Let $V$ be an $n$-dimensional vector space over $F$. Then 
  $$GL(V)\cong GL_{n}(F).$$
\end{theorem}

Let $G$ be a group, $F$ a field, and $V$ a vector space over
$F$. A {\it linear representation}\index{representation} of $G$ 
is any homomorphism $D:G\to GL(V)$. A representation is said to be 
{\it faithful}\index{representation!faithful} if it is injective. 
The {\it dimension}\index{representation!dimension} of a representation 
is the dimension of $V$.

\begin{example*}{}{}
\leavevmode
\begin{enumerate}
  \item Every group has a {\it trivial}\index{representation!trivial} 
    representation, $D(g)=1$.
  \item 
  $\Z_3$ has a 1D representation
  \begin{equation}
    D\left(\bar{0}\right)=1,~~~~
    D\left(\bar{1}\right)=e^{2\pi i/3},~~~~
    D\left(\bar{2}\right)=e^{4\pi i/3}.
  \end{equation}
  In the original group addition modulo $n$ is the binary operation,
  while the representation takes ordinary multiplication on the reals.
  An example of a 3D representation of $\Z_3$ is
  \begin{equation}\label{eq:reg}
  \begin{gathered}
    D\left(\bar{0}\right)=\left(\begin{array}{ccc}
                           1 & 0 & 0 \\
                           0 & 1 & 0 \\
                           0 & 0 & 1
                          \end{array}\right),~~~~
    D\left(\bar{1}\right)=\left(\begin{array}{ccc}
                           0 & 0 & 1 \\
                           1 & 0 & 0 \\
                           0 & 1 & 0
                          \end{array}\right),\\
    D\left(\bar{2}\right)=\left(\begin{array}{ccc}
                           0 & 1 & 0 \\
                           0 & 0 & 1 \\
                           1 & 0 & 0
                          \end{array}\right).
  \end{gathered}
  \end{equation}
  \end{enumerate}
\end{example*}
The representation of eq.~\eqref{eq:reg} is constructed by 
the following general prescription. Take $V=\R^n$ 
and imagine that the group elements form an orthonormal 
basis; i.e. $\ket{e_i}=\ket{g_i}$. Now define
\begin{equation}
  D(g_1)\ket{g_2}=\ket{g_1g_2}.
\end{equation}
The dimension of this representation is clearly just the order
of the group. One can recover the matrix elements via
\begin{equation}
  [D(g)]_{ij}=\bra{e_i}D(g)\ket{e_j}.
\end{equation}
  This is called the {\it regular}\index{representation!regular} 
representation. Another advantage of using linear spaces to represent groups
is that we can make a change of basis for our convenience. You
will recall this is achieved by similarity transformations
of the form
\begin{equation}\label{eq:simtr}
  D(g)\to D'(g)=S^{-1}D(g)S.
\end{equation}
This transformation clearly preserves the group multiplication
rules because $S$ cancels with its inverse.

Let $D$ be a representation of $G$ over a vector space $V$ with
subspace $W$. $W$ is an {\it invariant subspace}\index{subspace!invariant} 
of $D$ if
$\forall w\in W$
\begin{equation}
  D(g)w\in W.
\end{equation}
If $D$ has an invariant subspace it is said to be {\it reducible};
\index{representation!reducible} otherwise it is {\it irreducible}. 
We will shorten irreducible representation as irrep. If two representations $D$
and $D'$ are related by eq.~\eqref{eq:simtr} they are {\it equivalent}.
\index{representation!equivalent}
If $D$ is equivalent to a representation whose matrix
elements are in block diagonal form
\begin{equation}
  D(g)=\left(\begin{array}{ccc}
             D_1(g) & 0      &       \\
             0      & D_2(g) &       \\
                    &        & \ddots
             \end{array}\right)
\end{equation}
where the $D_i$ are irreps, then $D$ is called {\it completely
reducible}. Sometimes this is written as
\begin{equation}
  D=D_1\oplus D_2\oplus ...
\end{equation}
and we say that $D$ is the {\it direct sum}\index{direct sum} 
of the representations $D_i$.

\begin{theorem}{}{}
  Every representation of a finite group is equivalent to a
  unitary transformation.
%  \begin{proof} 
%  Let $D$ be a representation of a finite group $G$. Define
%  $$
%    S\equiv\sum_{g\in G}D(g)^\dagger D(g).
%  $$
%  $S$ is clearly hermitian and positive semidefinite. By
%  Theorem~\ref{thm:hermit}, it is diagonalizable.
%  \end{proof}
\end{theorem}
\begin{theorem}{}{}
  Every representation of a finite group is completely reducible.
\end{theorem}

% Talk about simple, schur's lemma.

\section{Young tableaux}

Recall from Section~\ref{sec:gpprelim} the permutation group on
$n$ objects, $S_n$. Any element in $S_n$ can be written
as a product of cycles, which are just cyclic permutations of subsets.
Conventional notation writes a cycle as a list of numbers between
parenthesis, indicating the set of elements are are permuted.
\begin{example*}{}{}
  Consider permutations of $V=\{x_1,...,x_n\}$. Then
  \begin{enumerate}
    \item (1) takes $x_1\to x_1$;
    \item (238) takes $x_2\to x_3\to x_8\to x_2$;
    \item a cycle of length $k$ is a {\it k-cycle};
    \item $\id$=(1)(2)...(n); and
    \item $(12)(3)...(n)\in S_n$ interchanges $x_1$ and $x_2$
          while leaving all other elements fixed.
  \end{enumerate}
\end{example*}
A simple $n$D representation of $S_n$ permutes the orthonormal
basis vectors of $\R^n$. If a permutation $\pi$ takes $x_i$
to $x_j$ then
\begin{equation}
  D(\pi)\ket{i}=\ket{j},
\end{equation}
which implies
\begin{equation}
  D(\pi)_{li}=\bra{l}D\ket{i}=\delta_{lj}.
\end{equation}
This is called the {\it defining representation}.
\index{representation!defining}

A set is called a {\it conjugacy class}\index{conjugacy class} if 
$gSg^{-1}=S$. If you have a group element $g_1$, the 
{\it conjugation} of $g_1$ is $gg_1g^{-1}$.
From the definition of normal subgroup in Section~\ref{sec:q} we
see that normal subgroups are conjugacy classes. The conjugacy classes
of of permutations are just the cycle structure; for instance all
interchanges are in the same conjugacy class. One way of seeing this
is by looking at the defining representation. If you conjugate using
an interchange, all this does is switch the basis vectors
$\ket{i}$ and $\ket{j}$, which clearly has no impact on the cycle
structure. Then, since any permutation can be built from interchanges,
it follows that an arbitrary conjugation preserves the cycle
structure. 

\section{Lie algebras and Lie groups}

  A {\it Lie algebra}\index{Lie!algebra} is a vector space $L$ over a field $F$
  together with an operation $[\ ,\ ]:L\times L\to L$ called the 
  {\it Lie bracket}\index{Lie!bracket}that satisfies 
  $\Forall \alpha,\beta\in F$ 
  and $x,y,z\in L$
  \begin{enumerate}
    \item $[\alpha x+\beta y,z]=\alpha[x,z]+\beta[y,z]$ (the Lie bracket is 
          {\it bilinear}),
    \item $[x,x]=0$ ({\it alternating} on $L$), and
    \item $[x,[y,z]]+[z,[x,y]]+[y,[z,x]]=0$ (and satisfies the {\it Jacobi
          identity})\index{Jacobi identity}.
  \end{enumerate}
  The basis elements $T^a$ of $L$ are called the {\it generators}.
  \index{Lie!generator}

\begin{proposition}{}{}
  The Lie bracket is anticommutative.
  \begin{proof}
    $0=[x+y,x+y]=[x,x]+[x,y]+[y,x]+[y,y]=[x,y]+[y,x].$
  \end{proof}
\end{proposition}

Lie Algebras are used in physics to construct Lie groups.
Lie groups are used to collect and analyze continuous symmetries of systems
and structures.
Let us see a way to construct Lie groups in physics. We will call the 
Lie group $G$, and the groups elements $g(\alpha)\in G$ will depend
smoothly on a set of continuous parameters $\alpha$. When we say ``smooth", we
mean that if two group elements are ``close together" in $G$, their parameters
are also close together. The identity is an important element in the group, so
we will parameterize elements with respect to it. We will shorthand
\begin{equation}
  \alpha=\left(\alpha^1,\,\alpha^2,\,...,\,\alpha^N\right),
\end{equation}
where $\alpha^a\in\R$. For our parameterization we set
\begin{equation}
  g(0)=\id.
\end{equation}
Then when we find a representation of the group, it will be
parameterized in the same way, so that
\begin{equation}
  D(0)=\id.
\end{equation}
In a neighborhood of $\id$ we can expand $D$. We find
\begin{equation}\label{eq:Depsilon}
  D(\epsilon\alpha)=\id+i\epsilon\alpha^aT^a+\order{\epsilon^2},
\end{equation}
where
\begin{equation}
  T^a\equiv-i\frac{\partial}{\partial\alpha^a}D(\alpha)\Big|_{\alpha=0}.
\end{equation}
If we can identify the $T^a$ here with the Lie algebra $T^a$, we can see
how it Lie algebra generates the Lie group. The $i$ is included so that
if the representation is unitary, the $T^a$ are Hermitian.

We can move in a fixed direction away from the identity using
eq.~\eqref{eq:Depsilon} by simply raising it to some power. This suggests
defining the representation of the group elements as
\begin{equation}
  D(\alpha)=\lim_{k\to\infty}\left(1+i\alpha^aT^a/k\right)^k
           =\exp(i\alpha^aT^a).
\end{equation}
In the limit, this expression clearly goes to the representation of a group 
element, because $k$ becomes large, which means the term in parentheses is a 
group element, which means the whole product is a group element, since the
product of group elements stays in the group.

Since the exponentials are group elements, it must be that
\begin{equation}
  \exp(i\alpha^aT^a)\exp(i\beta^bT^b)=\exp(i\delta^cT^c)
\end{equation}
for some $\delta$. Since our parameterization is smooth, we can solve for
$\delta$ by Taylor expanding both sides. We write
\begin{equation}
  i\delta^cT^c=\log\big(1+\exp(i\alpha^aT^a)\exp(i\beta^bT^b)-1\big).
\end{equation}
Keeping terms up to only second order in $\alpha$ and $\beta$ we find
\begin{equation}
  i\delta^cT^c=i\alpha^aT^a+i\beta^bT^b-\frac{1}{2}[\alpha^aT^a,\beta^bT^b].
\end{equation}
We can rearrange the above to find
\begin{equation}
  [\alpha^aT^a,\beta^bT^b]=
   -2i(\delta^c-\alpha^c-\beta^c)T^c\equiv i\gamma^cT^c,
\end{equation}
where the bracket here denotes the ordinary commutator.
From the LHS of the above, we can see that the $\gamma^c$ must be some
sum of products of the $\alpha^a$ and $\beta^b$, so we can write
\begin{equation}\label{eq:gammac}
  \gamma^c=\alpha^a\beta^bf^{abc}
\end{equation}
for some constants $f^{abc}$. Hence
\begin{equation}\label{eq:liealg}
  [T^a,T^b]=if^{abc}T^c.
\end{equation}
The $f^{abc}$ are called the {\it structure constants}
\index{structure constants}. We have
\begin{equation}
  f^{abc}=-f^{bac}
\end{equation}
since $[A,B]=-[B,A]$. The structure constants essentially summarize the
group multiplication law. Sometimes we refer to the commutator relation
\eqref{eq:liealg} as the Lie algebra, which makes sense because it
essentially gives you a prescription for how the Lie bracket works.

Something worth noting is that we follow the same steps
as above to prove the Campbell-Baker-Hausdorff formula. In particular
if $X$ and $Y$ are non-commuting matrices, and $\epsilon$ is small,
we can expand
\begin{equation}
  \log\big(1+\exp(\epsilon X)\exp(\epsilon Y)-1\big)
\end{equation}
to second order in $\epsilon$ and then do some rearranging.
\begin{theorem}{Campbell-Baker-Hausdorff formula}{}
  $$
    \exp(\epsilon X)\exp(\epsilon Y)
      =\exp\left(\epsilon X+\epsilon Y+\frac{1}{2}[\epsilon X,\epsilon Y]
                           +\order{\epsilon^3}\right).
  $$
\end{theorem}

\subsection{SU(2)}

The $\SU(2)$ algebra is defined by
\begin{equation}
  [J_i,J_j]=i\epsilon_{ij}J_k.
\end{equation}
In units $\hbar=1$, this is the algebra of the familiar angular momentum
operator from QM. 

\begin{proposition}{}{}
  Let $A,B\in\SU(2)$ with parameterization
  $$
    A=a_0\id+i\vec{a}\cdot\vec{\sigma}, \qquad
    B=b_0\id+i\vec{b}\cdot\vec{\sigma},
  $$
  where the $\sigma_i$ are the usual Pauli matrices. Then
  \begin{equation*}
    AB=\id\left(a_0b_0-\sum\limits_{i=1}a_ib_i\right)
        +i\sum\limits_{i=1}\sigma_i\left(a_0b_i+a_ib_0
          -\sum\limits_{j\neq k}a_jb_k\epsilon_{jki}\right).
  \end{equation*}
  \begin{proof}
    \begin{equation*}
      \begin{aligned}
        AB&=\left(a_0\id+i\sum\limits_ia_i\sigma_i\right)
            \left(b_0\id+i\sum\limits_ib_i\sigma_i\right)\\
          &=\id\left(a_0b_0-\sum\limits_ia_ib_i\right)
            +i\sum\limits_i\sigma_i\left(a_0b_i+a_ib_0\right)
            -\sum\limits_{j\neq k}a_jb_k\sigma_j\sigma_k \\
          &=\id\left(a_0b_0-\sum\limits_ia_ib_i\right)
            +i\sum\limits_i\sigma_i\left(a_0b_i+a_ib_0\right)
            -i\sum\limits_i
                \sum\limits_{j\neq k}a_jb_k\epsilon_{jki}\sigma_i.
      \end{aligned}
    \end{equation*}
  \end{proof}
\end{proposition}
Something worth noting about the parameterization of the above proposition
is that it shows a bijection between $\SU(2)$ and the three-sphere $S^3$.
In this parameterization, $a_\mu$ is a component of a 4D Euclidean vector where
$a_0\equiv a_4$. The vector, which specifies the $\SU(2)$ matrix entirely,
satisfies $a_\mu a_\mu=1$, which means it's a point on the surface of a 4D
hypersphere; i.e. the point is on $S^3$.
Additionally this parameterization is useful for carrying out
matrix multiplications on the computer.

\begin{proposition}{}{su2add}
  Let $A,B\in\SU(2)$. Then
  $$
  \frac{A+B}{\sqrt{\det(A+B)}}\in\SU(2).
  $$
  \begin{proof}
    This follows from $\det(cA)=c^n\det(A)$ for any $n\times n$ matrix $A$.
  \end{proof}
\end{proposition}

\bibliographystyle{unsrtnat}
\bibliography{bibliography}
 
