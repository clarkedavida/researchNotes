\chapter{Physics: Renormalization Group}

These research notes were made from the perspective of a researcher working
in lattice field theory, and one advantage of the lattice, which will be
discussed later in Chapter~\ref{ch:preliminaries}, is that it allows one
to make direct analogies with statistical mechanical systems, and in
particular, to utilize powerful techniques like renormalization group (RG)
analysis.

I could have combined this chapter with the statistical physics chapter, but I
decided that since RG analysis is a field {\it sui generis},
it deserves its own chapter.  This chapter will generally try to follow 
lectures by Frithjof Karsch, but I will also use details from a nice
book I found Bauerschmidt, Brydges, and Slade \cite{Bauerschmidt_2019}.
In that book, in their introduction, they give a nice overview of the idea
behind RG transformations, saying
\begin{quote}
Two paramount features of critical phenomena are scale invariance and 
universality. The renormalization group method exploits the scale 
invariance to explain universality. This is done via a multi-scale analysis, 
in which a system studied at a particular scale is represented by an 
effective Hamiltonian. Scales are analyzed sequentially, leading to a map 
that takes the Hamiltonian at one scale to a Hamiltonian at the next scale. 
Advancing the scale gives rise to a dynamical system defined by this map. 
Scale invariance occurs at a fixed point of the map, and different fixed
points correspond to different universality classes. The analysis of the 
dynamical system at and near the fixed point provides a means to compute 
universal quantities such as critical exponents. 
\end{quote} 
The goal of this chapter is to try to understand this quote in more detail,
and to learn how to carry out and interpret RG calculations that often
appear in the context of lattice field theory.

\section{Spin systems}


\bibliographystyle{unsrtnat}
\bibliography{bibliography}

