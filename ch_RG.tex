\chapter{Physics: Renormalization Group}

These research notes were made from the perspective of a researcher working
in lattice field theory, and one advantage of the lattice, which will be
discussed later in \chref{ch:preliminaries}, is that it allows one
to make direct analogies with statistical mechanical systems, and in
particular, to utilize powerful techniques like renormalization group (RG)
analysis.

This chapter will generally try to follow 
lectures by F. Karsch, as well as the book by Binney et
al.~\cite{binney_theory_1992}, but I will also use details from a nice
book I found Bauerschmidt et al.~\cite{Bauerschmidt_2019}.
In that book, they give a nice overview of the idea
behind RG transformations, saying
\begin{quote}
Two paramount features of critical phenomena are scale invariance and 
universality. The renormalization group method exploits the scale 
invariance to explain universality. This is done via a multi-scale analysis, 
in which a system studied at a particular scale is represented by an 
effective Hamiltonian. Scales are analyzed sequentially, leading to a map 
that takes the Hamiltonian at one scale to a Hamiltonian at the next scale. 
Advancing the scale gives rise to a dynamical system defined by this map. 
Scale invariance occurs at a fixed point of the map, and different fixed
points correspond to different universality classes. The analysis of the 
dynamical system at and near the fixed point provides a means to compute 
universal quantities such as critical exponents. 
\end{quote} 
\clearpage
The goal of this chapter is to try to understand this quote in more detail,
and to learn how to carry out and interpret RG calculations that often
appear in the context of lattice field theory.

\section{Spin systems}

\section{The scaling hypothesis}
All critical exponents\index{critical exponent} can be written as functions 
of the number of dimensions $d$ as well as $\nu$ and $\eta$. Given the reduced 
temperature\index{temperature!reduced}
\begin{equation}
  t\equiv\frac{T-T_c}{T_c}
\end{equation}
and external magnetic field $h$ one gets the 
relationships~\tabref{tab:scaling}.
These are equivalent to the {\it hyperscaling relations}
\index{hyperscaling relation}
\begin{equation}\begin{aligned}
2\beta+\gamma&=2-\alpha,\\
2\beta\delta-\gamma&=2-\alpha,\\
\gamma&=\nu(2-\eta),\\
\nu d&=2-\alpha.
\end{aligned}\end{equation}
These can be derived from the {\it scaling hypothesis}\index{scaling
hypothesis}, which we will now show for a large class of RG transformations.
\begin{table}
\begin{tabularx}{\linewidth}{lCr} \hline\hline
       Exponent & Definition & Value\\[3pt]\hline
$\alpha$ & $C_V\sim|t|^{-\alpha}$  & $2-\nu d$\\[3pt] 
$\beta$ & $m\sim|t|^{\beta}$ & $\frac{1}{2}\nu(d+\eta-2)$\\[3pt]
$\gamma$ & $\chi\sim|t|^{-\gamma}$  & $\nu(2-\eta)$\\[3pt] 
$\delta$ & $m\sim h^{1/\delta}$ &  $\frac{d+2-\eta}{d-2+\eta}$\\[3pt]
$\nu$ & $\xi\sim |t|^{-\nu}$ & \\[3pt]
        \hline\hline
\end{tabularx}
\caption{Relationships among the critical exponents. Table adapted from 
         from Ref.~\cite{binney_theory_1992}. In each case one coupling is
         small while the other is fixed to zero. For $\beta$, $t$ approaches
         zero from below.}
\label{tab:scaling}
\end{table}

Suppose we start with a system of variables $\set{\sigma}$ with Hamiltonian $H$
and $n$ couplings $\vec{k}=(k_1,...,k_n)^t$. After one RG step, we obtain
a new system of spins $\set{\sigma'}$ with $H'$ and $\vec{k}'$. We want
that expectation values remain unchanged after an RG step.
Thinking about block transformations in particular, one can
achieve this by collecting 
all those original configurations $\set{\sigma}$ that lead to the same
renormalized configuration $\set{\sigma}$\footnote{For example in the 2$d$
Ising model, if one blocks three up-spins and one down spin, there are 
four possible ways to get the same effective spin via majority rule.}, which can be
summarized as
\begin{equation}
  e^{-H'\left(\set{\sigma'},~\vec{k}'\right)}
     =e^{-\beta G(\vec{k})}\sum_{\set{\sigma}~
                        {\rm giving}~\set{\sigma'}}
      e^{-H\left(\set{\sigma},~\vec{k}\right)}.
\end{equation}
Here we have expressed an overall normalization as $e^{-\beta G}$, which will
be convenient later. When we sum over all $\set{\sigma'}$ we then get
\begin{equation}
  Z'(\vec{k}')=e^{-\beta G(\vec{k})} Z(\vec{k}),
\end{equation}
or in terms of free energies,
\begin{equation}
  F'(\vec{k}')=F(\vec{k})-G(\vec{k}).
\end{equation}
Now $F$ is extensive, so we can write
\begin{equation}
  F(\vec{k})=Nf(\vec{k})
\end{equation}
where $N$ is the number of spins on the original lattice and $f$ is intensive.
Similarly one can write
\begin{equation}
  F'(\vec{k}')=\frac{N}{b^d}f(\vec{k}')~~~~\text{and}~~~~G(\vec{k})=Ng(\vec{k}),
\end{equation}
where $b$ is the {\it scale factor}\index{scale factor}, which represents the
factor by which the lattice spacing changes. That $F$ and $F'$ are proportional
to the same intensive function with different couplings follows from the fact
that the effective Hamiltonian $H'$ has the same functional form as $H$.
Hence
\begin{equation}\label{eq:widom1}
f(\vec{k}')=b^d\left(f(\vec{k})-g(\vec{k})\right).
\end{equation}


Next, let's think about how to relate the couplings $\vec{k}'$ and $\vec{k}$
near the fixed point $\vec{k}_0$. 
We can Taylor expand $\vec{R}$ in the vicinity of $\vec{k}_0$ to find
\begin{equation}
  \vec{k}_0+\delta\vec{k}'=\vec{R}\left(\vec{k}_0+\delta\vec{k}\right)
                          \approx\vec{R}(\vec{k}_0)
                           +\frac{\partial R_i(\vec{k})}{\partial k_j}
                              \Big|_{\vec{k}=\vec{k}_0}\delta\vec{k}.
\end{equation}
The first-order term defines a matrix $M$.
Since the effect of the renormalization action
$\vec{R}$ on the fixed point is $\vec{R}(\vec{k}_0)=\vec{k}_0$, we can write
\begin{equation}
  \delta\vec{k}'=M\delta\vec{k}.
\end{equation}
Now let's switch to a basis in which $M$ is diagonal. In this basis, our
original system contains new couplings $\vec{x}$, and hence the renormalized
couplings $\vec{x}'$ are given according to the eigenvalues of $M$ as
\begin{equation}
  x_i'=\lambda_i x_i~~~~\text{no summation}.
\end{equation}
Hence we can recast \equatref{eq:widom1} as
\begin{equation}\label{eq:widom1}
f(\lambda_1x_1,\lambda_2x_2,...)=b^d\left(f(x_1,x_2,...)-g(x_1,x_2,...)\right).
\end{equation}


Now $f$ and $g$ are some kinds of functions of $\vec{x}$. Generally speaking
there will be some part that is expressible as a Taylor series in the $\vec{x}$;
this is called the {\it regular} part.\index{regular} There must also be some
part not expressible as a Taylor series in order that we can obtain the power
law behavior one finds in the vicinity of a phase transition; this is
called the {\it singular} part.\index{singular} In order to proceed we make
the following assumption: The singular part is completely contained in
$f$. This is not true of all RG transformations, but it is true
of many; for example it can be shown that this assumption holds for any
RG transformation whose renormalized block variables depend linearly
on the original ones. Under this assumption we can write
\begin{equation}\label{eq:widom2}
  \fsing(\lambda_1x_1,\lambda_2x_2,...)=b^d\fsing(x_1,x_2,...).
\end{equation}


Finally we specialize to our couplings $t$ and $h$. Sufficiently close to the
critical point, these can be identified with two of these eigendirections, say
$x_1=t$ and $x_2=h$. Furthermore the action of the RG transform will not abruptly
change the direction of the flow; this is equivalent to $\lambda_t>0$ and
$\lambda_h>0$. Since these eigenvalues are positive and $b$ is positive we
can introduce $\lambda_i=b^{\,y_i}$. Hence,
\begin{equation}\label{eq:widom3}
  \fsing(b^{\,y_t}t,b^{\,y_h}h,...)=b^d\fsing(t,h,...).
\end{equation}
Since the system is supposed to be scale-invariant, you are free to choose $b$
as you like; e.g. you are free to set the length scale of your blocking
procedure in your RG step. Hence we can choose
\begin{equation}
  b=|t|^{-1/y_t}~~~~\text{or}~~~~b=h^{-1/y_t}.
\end{equation} 
If we restrict our attention to the critical surface $x_i=$ for all $i\geq2$
we obtain finally the scaling hypothesis:
\begin{theorem}{Widom scaling hypothesis}{}
  \begin{equation*}\begin{aligned}
    \fsing(t,h)&=|t|^{d/y_t}\fsing\left(\pm1,|t|^{-y_h/y_t}h\right)\\
    \fsing(t,h)&=h^{d/y_h}\fsing\left(h^{-y_t/y_h},1\right)
  \end{aligned}\end{equation*}
\end{theorem}

One can see from the scaling hypothesis that the order parameter, its
susceptibility, and $C_V$ will scale according to critical exponents
near the critical point. We choose one or the other of the above forms
strategically depending on the observable we're looking at:
\begin{equation}\begin{aligned}
  C_V&=\partial_T^2 F
     \sim\partial_t^2 |t|^{d/y_t}\fsing\left(\pm1,|t|^{-y_h/y_t}h\right)
     \sim|t|^{d/y_t-2}
     \equiv|t|^{-\alpha}\\
  m\big|_{h=0}&=\partial_h F\big|_{h=0}
     \sim\partial_h |t|^{d/y_t}\fsing\left(\pm1,|t|^{-y_h/y_t}h\right)\big|_{h=0}
     \sim|t|^{d/y_t-y_h/y_t}
     \equiv|t|^{\beta}\\
  \chi\big|_{h=0}&=\partial^2_h F\big|_{h=0}
     \sim\partial^2_h |t|^{d/y_t}\fsing\left(\pm1,|t|^{-y_h/y_t}h\right)\big|_{h=0}
     \sim|t|^{d/y_t-2y_h/y_t}
     \equiv|t|^{-\gamma}\\
  m\big|_{t=0}&=\partial_h F\big|_{t=0}
     \sim h^{d/y_h}\fsing\left(h^{-y_t/y_h},1\right)\big|_{t=0}
     \sim h^{d/y_h-y_t/y_h}
     \equiv h^{1/\delta}.
\end{aligned}\end{equation}
One can eliminate the unknown variables $y_h$, $y_t$, and $d$ to obtain
the first two hyperscaling relations.


\section{Classification of observables}


Owing to the powerful connection with statistical physics, RG techniques are of
considerable use in LFT. For example one can look at the scaling of various
observables near a continuous phase transition to figure out the universality
class of that transition. In order to do so, one has to figure out which
observables are analogous to, say, the magnetization. In this context one
distinguishes between {\it energy-like}\index{energy-like} and {\it
magnetization-like}\index{magnetization-like} quantities. Energy-like
observables are those that have the same scaling behavior as the energy density
in spin systems, while magnetization-like quantities scale like the
magnetization.

In order to utilize these tools, one needs to correctly classify the observable
of interest. In the context of global symmetry breaking, magnetization-like
quantities are those explicitly break the symmetry, while energy-like quantities
remain unchanged.

We take as an example the QCD chiral phase transition and try to classify some
observables and couplings with respect to it. Let's see how the fermionic
number operator in a Euclidean metric $\bar{\psi}\gamma_4\psi$ changes under
chiral rotations~\eqref{eq:SUNfc} and \eqref{eq:U1A}:
\begin{equation}
  \bar{\psi}'\gamma_4\psi'
    =\bar{\psi}e^{i\alpha\gamma_5T^a}\gamma_4e^{i\alpha\gamma_5T^a}\gamma_4\psi.
\end{equation}
The $T^a$ carry no Dirac indices, so they will commute with the $\gamma_i$,
meanwhile $\gamma_4$ and $\gamma_5$ anti-commute. Hence
\begin{equation}
  \bar{\psi}'\gamma_4\psi'
    =\bar{\psi}\gamma_4e^{-i\alpha\gamma_5T^a}e^{i\alpha\gamma_5T^a}\gamma_4\psi
    =\bar{\psi}\gamma_4\psi,
\end{equation}
which tells us the number operator is energy-like and that the quark chemical
potential is an energy-like coupling.

\section{Scaling of observables}


The strategy now is to use the Taylor expansions of $f_f$ in the regimes
$z\to\pm\infty$ and $z\approx 0$~\cite{Engels:2011km} to derive the
behavior of $c_T$ for small $H$. When $T<T_c$, using their
notation for the Taylor coefficients and letting $m\geq1$,
we obtain for the $z$-derivatives
\begin{equation}
    f_f^{(m)}=(-z)^{2-\alpha-m}
    \times\sum_{n=0}^\infty c_n^-(-z)^{-n\Delta/2}\prod_{\ell=1}^m\left(\alpha-3+\ell+\frac{n\Delta}{2}\right).
\end{equation}
Meanwhile when $T\approx T_c$ we have 
\begin{equation}
    f_f^{(m)}=\sum_{n=m}^\infty a_n\frac{n!}{(n-m)!}z^{n-m}.
\end{equation}
Finally when $T>T_c$,
\begin{equation}
    f_f^{(m)}=z^{2-\alpha-m}\sum_{n=0}^\infty c_n^+z^{-2n\Delta}\prod_{\ell=1}^m\left(3-\alpha-\ell-2n\Delta\right).
\end{equation}


\bibliographystyle{unsrtnat}
\bibliography{bibliography}

