\chapter{Special Topic: Neutrinos}\label{ap:spec_neutrino}

This chapter is mostly dedicated to a summary of the experiments and 
observations that led to the belief
that neutrinos have mass. Toward the end, we also collect other topics of
interest related to neutrinos. First the solar neutrino problem and early solar
neutrino experiments are discussed. The atmospheric neutrino problem is
discussed. Next the theory behind neutrino oscillations is explained, along
with a possible Standard Model (SM) mechanism giving neutrinos mass. Recent
experiments supporting the neutrino oscillation hypothesis are discussed.
We finish with miscellaneous neutrino topics,
along with future possibilities for neutrino physics research.

Since this chapter was adapted from a project I did as a grad student in 2015,
the experiments are likely not current. Maybe the status of measurements
such as PMNS matrix elements could be updated in the future.
It follows Thomson~\cite{thomson_modern_2013} and a similar project by
my friend A. Yunesi.

\index{solar neutrino problem}
\section{The solar neutrino problem}

\begin{figure}
  \centering
  \includegraphics[height=0.33\textheight,keepaspectratio]
                {pictures/t13_3.pdf}
  \caption{Solar neutrino flux and energy of several processes in the sun.
           Image taken from Thomson Fig. 13.3~\cite{thomson_modern_2013}.}
  \label{fig:solar}
\end{figure}
Roughly $2\times10^{38}$ $\nu_e$ are produced by nuclear fusion in the sun
each second. One of the main processes by which this occurs is the pp-cycle,
\begin{equation}
 \text{p}+\text{p}\to\text{D}+e^++\nu_e.
\end{equation} However experiments detecting
solar neutrinos tend to focus on rarer reactions, such as the
\ce{^7Be} electron capture process, pep-process, and
\ce{^8B} $\beta$-decay, which are, respectively,
\begin{equation}
  \begin{aligned}
    \ce{^7Be}+e^-&\to\ce{^7Li}+\nu_e,
    \\
    \text{p}+e^-+\text{p}&\to\ce{^2H}+\nu_e, 
    \\
    \ce{^8_5B}&\to\ce{^8_4Be}^*+e^++\nu_e.
  \end{aligned}
\end{equation}
As illustrated in \figref{fig:solar}, these processes tend to produce 
neutrinos of higher
energies than the pp-cycle, which allows the neutrinos to overcome kinematic
barriers that would otherwise preclude them from participating in the
interactions used to detect them.
The Standard Solar Model (SSM) allows us to predict how many electron neutrinos
we expect to see. However the results of the first experiments measuring solar
neutrino flux were somewhat unexpected.

\subsection{Radiochemical experiments}
Ray Davis, Jr. and his collaborators started the first experiment to detect
solar neutrinos. The Homestake experiment, based in a mine in South Dakota,
had a 615 ton tank full of dry-cleaning fluid, $\text{C}_2\text{Cl}_4$,
which was chosen for the interaction mentioned below.
Mines are prime real estate for neutrino experiments because being underground
helps isolate them from cosmic rays. The number of captured solar neutrinos
was measured by counting the $\ce{^{37}Ar}$ atoms produced in
the process
\begin{equation}
  \nu_e+\ce{^{37}_{17}Cl}\to\ce{^{37}_{18}Ar}+e^-.
\end{equation}
The minimum energy an electron neutrino needs to participate in the above
process is about 0.81 MeV, which excludes neutrinos produced by the pp-cycle.
Although a huge number of electron neutrinos exit the sun, the SSM
predicted a modest 1.7 interactions per day. The findings
of Homestake were presented in 1968: Merely $0.48\pm0.04$ daily interactions
were found!

Later radiochemical experiments used gallium as a target, and were therefore
sensitive to neutrinos produced in the pp-cycle. The Soviet-American
Gallium Experiment (SAGE) started running in 1989, and appears to still be
collecting data as of March 2015 \cite{gavrin_current_2015}. 
The Gallium Experiment (GALLEX),
which was later succeeded by the Gallium Neutrino Observatory (GNO),
collected data between 1991 and 1997, and was an international project
managed by American, French, German, Italian, Israeli, and Polish scientists.
Both of these experiments confirmed the deficit of solar neutrinos; this
mysterious absence of solar neutrinos became known as the {\it solar
neutrino problem}.

\subsection{Water \v{C}erenkov experiments}
\index{radiation!\v{C}erenkov}
\v{C}erenkov radiation occurs whenever a charged particle passes through a
dielectric medium at a speed greater than the phase speed of light in that
medium. The particle polarizes the medium, then travels away from that
disturbance faster than the electromagnetic field can react, leaving
a wake where those disturbances constructively interfere. Elementary
geometry shows that if the medium has index of refraction $n$, the angle
at which this radiation is emitted is given by $\co=1/n\beta$. Water
\v{C}erenkov detectors work by measuring this angle as leptons propagate
through water. This gives a measure of $\beta$, and hence the energy.
A schematic of a \v{C}erenkov ring is given in \figref{fig:cerenkov} (b).

\begin{figure}
  \centering
  \includegraphics[width=0.75\textwidth,keepaspectratio]
                {pictures/t13_4.pdf}
  \vspace{-10mm}
  \caption{(a) Schematic of the Super-K tank. (b) Electron
           projecting a \v{C}erenkov ring on PMTs. Image taken
           from Thomson Fig. 13.4~\cite{thomson_modern_2013}.}
  \label{fig:cerenkov}
\end{figure}

The Kamioka Nucleon Decay Experiment (or Kamiokande) was
originally a project to study proton decay. In 1988 it was
upgraded, giving it the ability to study solar neutrinos. It was upgraded
again in 1996, donning the name Super Kamiokande (Super-K), with the intent
of obtaining higher statistics than Kamiokande. Super-K is still running today.
Located 1 km underground in the Kamioka area of Gifa Prefecture, Japan,
the Super-K detector consists of a 50 kiloton water tank viewed by 11,146
photo multiplier tubes (PMTs). As charged leptons pass through the water,
they project rings of \v{C}erenkov light on the walls of the detector. The
center-of-mass (CM)
frame is boosted in the direction of the neutrino, which means that the
electron direction is very close to the neutrino direction. This setup
allows experimentalists to not only detect neutrino energies down to
approximately 5 MeV, but also detect particle orientations, which gives Super-K
an advantage over radiochemical experiments. A schematic of this setup
is shown in \figref{fig:cerenkov} (a).
The neutrinos are detected through the elastic scattering (ES) process
\begin{equation}
  \nu_ee^-\to\nu_ee^-. 
\end{equation}
One might expect to also see neutrinos interact
using oxygen, in particular through the charged current (CC) process
\begin{equation}
  \nu_e+\ce{^{16}_8O}\to\ce{^{16}_9F}+e^-.
\end{equation}
But oxygen is more stable than the fluorine isotope, making the process
kinematically forbidden to the relatively low energy solar neutrinos. The 5
MeV threshold, which is a result of radioisotope $\beta$ decay background,
 makes Super-K primarily sensitive to neutrinos produced by
the $\ce{^8B}$ process.

The left plot of \figref{fig:superKdata} is a graph of the electron 
direction with respect to the
direction of the sun. The largest number of events occurred near 
$\co_{\theta_{\rm sun}}=1$,
which provides strong evidence that a large flux of neutrinos comes from the
sun. Just as its predecessors, Super-K experienced an electron neutrino
deficit, finding only $0.474\pm0.03$ of the number of electron neutrinos
predicted by the SSM.

\begin{figure}
  \centering
  \vspace{-20mm}
  \includegraphics[width=0.9\textwidth,keepaspectratio]
                {pictures/t13_6&8.pdf}
  \vspace*{-20mm}
  \caption{Left: Super-K neutrino data plotted against the cosine of the
           angle of the electron with respect to the sun. Right: Constraints
           on neutrino fluxes from SNO data. The bands indicate one standard
           deviation, and the 68\% confidence ellipse is shown. Images taken
           from Thomson Fig. 13.6 and 13.8~\cite{thomson_modern_2013}.}
  \label{fig:superKdata}
\end{figure}

Homestake, Super-K, and other neutrino experiments demonstrated
a dearth of solar electron neutrinos. The Sudbury Neutrino Observatory (SNO)
in Canada took these experiments a step further by measuring the
electron neutrino flux and total neutrino flux from the sun. SNO had a 12m
diameter tank filled with 1 kiloton of $\text{D}_2\text{O}$, which was used
because deuteron has a small binding energy compared to the energies of
neutrinos produced by $\ce{^8B}$. Electron neutrinos can then
be detected through
\begin{equation}
  \nu_e+\text{D}\to e^-+\text{p}+\text{p}
\end{equation}
the CC interaction, and neutrinos of all flavors
interact with the deuteron via the neutral current (NC) interaction
\begin{equation}
 \nu_\ell+\text{D}\to\nu_\ell+\text{n}+\text{p}
\end{equation} 
and ES interactions. The NC
interaction is equally sensitive to all flavors of neutrino, but the ES
interaction is more sensitive to electron neutrinos because $\nu_\mu$ and
$\nu_\tau$ only interact through the NC ES process. In total the interaction
rates obey
\begin{equation}
  \begin{aligned}
    \text{CC rate}&\propto\Phi(\nu_e), \\
    \text{NC rate}&\propto\Phi(\nu_e)+\Phi(\nu_\mu)+\Phi(\nu_\tau), \\
    \text{ES rate}&\propto\Phi(\nu_e)+0.154\big(\Phi(\nu_e)+\Phi(\nu_e)\big).
  \end{aligned}
\end{equation}

For the CC interaction, the
emitted electron is detected by its \v{C}erenkov ring. In the CM frame
the direction of the neutrino relative to the electron is nearly isotropic, and
because the neutrino energy is much smaller than the deuteron mass, the
electron orientation is essentially uncorrelated with the sun's direction.
The ES interaction is also detected through \v{C}erenkov rings, but these
leptons correlate with the sun's direction as in the Super-K. For the NC
interaction, the produced neutron is eventually captured in the process
\begin{equation}
 \text{n}+\ce{^2_1H}\to\ce{^3_1H}+\gamma.
\end{equation}
The photon then produces electrons through subsequent interactions that are
then detected by \v{C}erenkov rings. Ultimately SNO found
\begin{equation}
  \begin{aligned}
    \Phi(\nu_e)&=(1.76\pm0.10)\times10^{-6}~\text{cm}^{-2}\text{s}^{-2} \\
    \Phi(\nu_\mu)+\Phi(\nu_\tau)
               &=(3.41\pm0.63)\times10^{-6}~\text{cm}^{-2}\text{s}^{-2},
  \end{aligned}
\end{equation}
The theoretical prediction of the electron neutrino flux was
\begin{equation}
  \Phi(\nu_e)_{\rm theory}
    =(5.1\pm0.9)\times10^{-6}~\text{cm}^{-2}\text{s}^{-1}, 
\end{equation}
so we see that 
$\Phi(\nu_e)_{\rm theory}=\Phi(\nu_e)+\Phi(\nu_\mu)+\Phi(\nu_\tau)$
within error. Thus SNO showed that the total
neutrino flux was consistent with the theoretical expectation as long as
one allows the neutrino flux from the sun to include muon and tau neutrinos.
But these flavors of neutrinos aren't produced in the sun; therefore SNO also
provided real evidence of neutrino oscillations.
A plot showing the confidence bands of the different processes for data
generated by SNO is shown on the right of \figref{fig:superKdata}.

These early discoveries about neutrinos weren't merely surprising; they
created a new and burgeoning area of study in high energy physics
that lends important
insight to the nature of SM neutrinos. Discoveries about neutrinos tend
to generate a lot of excitement.
In fact in 2002, Ray Davis and Masatoshi Koshiba each received 1/4 of
the Nobel Prize for their involvement in neutrino detection, the former
due to Homestake and the latter due to Super-K. In 2015 
Takaaki Kajita and Arthur B. McDonald split the Nobel Prize for their
stakes in the Super-K and SNO experiments, respectively.

\index{atmospheric neutrino problem}
\section{The atmospheric neutrino problem}
Cosmic rays, which consist mostly of protons and alpha particles, interact with
particles in the earth's atmosphere. Secondary particles emitted in these
interactions will sometimes decay and produce an {\it atmospheric} neutrino,
provided their energy is low enough ($\lesssim2$ GeV). The atmospheric neutrinos
are then produced via
\begin{equation}
  \begin{aligned}
    \text{p}+N&\to\pi^\pm+X, \\
    \pi^\pm&\to\mu^\pm+\nu_\mu(\bar{\nu}_\mu), \\
    \mu^\pm&\to e^\pm+\nu_e(\bar{\nu}_e)+\bar{\nu}_e(\nu_\mu).
  \end{aligned}
\end{equation}
Assuming the neutrino flavor doesn't change on its journey to the detector,
the above sequence of processes implies
\begin{equation}
  \mathcal{R}\equiv\frac{N_{\nu\mu}+N_{\bar{\nu}\mu}}
                           {N_{\nu e}+N_{\bar{\nu}e}}\approx2,
\end{equation}
where $N_i$ indicates the number of particles of type $i$.
It is difficult to determine an exact value for $\mathcal{R}$ because it is
affected by many factors, including solar activity and geomagnetic cut-off.
Therefore to create a quantity independent of external effects, we define
\begin{equation}
  \mathcal{R}'\equiv\frac{\mathcal{R}}{\mathcal{R}_\text{MC}},
\end{equation}
where $\mathcal{R}_\text{MC}$ is obtained using Monte Carlo simulations instead
of observed data. A drawback to $\mathcal{R}'$ is that it cannot distinguish
between an abundance of electrons and a deficit of muons.

In 1986, the first experiment to discover a discrepancy between the number of
observed and expected atmospheric neutrinos was the Irvine-Michigan-Brookhaven
detector (IMB), which was located in a salt mine owned by Morton on 
the shore of Lake
Erie, Ohio.  Shortly thereafter in 1988, Kamiokande confirmed this deficit.
The next two experiments to investigate this phenomenon, Fr\'ejus and NUSEX,
actually were unable to reproduce this finding. As a result many physicists
believed the discrepancy was due to systematic errors originating in our poor
understanding of neutrino interactions with iron and water. The issue was
resolved when the Soudan2 experiment reaffirmed the findings of IMB and
Kamiokande. But the most convincing results came from Super-K, which showed
that the deficit of neutrinos depends on the zenith angle.

\begin{figure}
  \centering
  \includegraphics[width=0.85\textwidth,height=0.25\textheight,keepaspectratio]
                {pictures/dblratio.jpg}
  \vspace*{10mm}
  \caption{Plot summarizing the values of $\mathcal{R}'$ as found by early
           neutrino experiments. Image taken from Fig. 11 on T2K experiment
           webpage~\cite{T2K}.}
  \label{fig:rprime}
\end{figure}

\index{neutrino!oscillation}
\section{Theory of neutrino oscillations}
Thankfully, both the atmospheric and solar neutrino problems are resolved
neatly by neutrino oscillations.
Here we briefly explain the theory behind neutrino oscillations as well as a
model for how neutrinos gain mass. In the following, we use $x$ to denote
the four-vector of a particle's space-time position.

\index{neutrino!eigenstates}
The {\it mass eigenstates} $\nu_1$, $\nu_2$, and $\nu_3$ of the free particle
Hamiltonian are its physical states of definite mass; i.e., they are the
eigenstates of the Hamiltonian. Meanwhile the {\it weak eigenstates} $\nu_e$,
$\nu_\mu$, and $\nu_\tau$ are the eigenstates corresponding to particles
produced in weak interactions. There is no a priori reason to assume that these
are the same set of eigenstates--in general each set forms a basis spanning
the space of physical states. Hence we can relate the bases to each other
through a unitary transformation, represented below as the
Pontecorvo-Maki-Nakagawa-Sakata (PMNS) matrix. \index{PMNS matrix}We write
\begin{equation}
  \colvec{3}{\nu_e}{\nu_\mu}{\nu_\tau}=
  \left(\begin{array}{ccc}
    U_{e1} & U_{e2} & U_{e3} \\
    U_{\mu 1} & U_{\mu 2} & U_{\mu 3} \\
    U_{\tau 1} & U_{\tau 2} & U_{\tau 3}
  \end{array}\right)
  \colvec{3}{\nu_1}{\nu_2}{\nu_3}
\end{equation}
from which we see that the electron neutrino state, for example, is
a superposition of the mass eigenstates
\begin{equation}
  \label{eq:mix1}
  \ket{\nu_e}=U_{e1}^*\ket{\nu_1}+U_{e2}^*\ket{\nu_2}+U_{e3}^*\ket{\nu_3}.
\end{equation}
Once the neutrino weakly interacts, the wavefunction collapses to a (possibly
different) weak eigenstate. This is believed to be the mechanism by which
neutrinos change flavor.

\index{PMNS matrix}
\subsection{The PMNS matrix}

In general the elements of the PMNS matrix are complex,
so it can be parameterized by eighteen real numbers. However since
$U^\dagger U=\id$, we obtain nine constraints among the elements, which leaves
nine parameters. Hence the matrix can be written in terms of three
mixing angles $\theta_{12}$, $\theta_{23}$, and $\theta_{13}$, and six phases.
Five of these phases can be absorbed in the definitions of neutrino
and charged lepton spinors without modifying weak interaction currents. To
see this note that using the PMNS matrix, a typical weak interaction term can
be written as
\begin{equation}
  -i\frac{g_W}{\sqrt{8}}(\bar{e},\bar{\mu},\bar{\tau})\gamma^\mu(1-\gamma_5)
  \left(\begin{array}{ccc}
    U_{e1} & U_{e2} & U_{e3} \\
    U_{\mu 1} & U_{\mu 2} & U_{\mu 3} \\
    U_{\tau 1} & U_{\tau 2} & U_{\tau 3}
  \end{array}\right)
  \colvec{3}{\nu_1}{\nu_2}{\nu_3}. \\
\end{equation}
We can rephase the PMNS matrix by pulling out two diagonal matrices
of three phases each, one on the left and one on the right. The left matrix
of phases can then be incorporated into the charged lepton phases while the
right matrix of phases is incorporated into the neutrino phases. The reason
that this eliminates five parameters rather than six is that one of these
phases is redundant; indeed, multiplying the PMNS matrix by an overall phase
$\theta$ has no physical consequence, and the other six phases can be
defined relative to $\theta$.

Ultimately we are left with our three angles and one phase, which we will
call $\delta$. The PMNS is then usually cast as something close to a product
of three rotations about orthogonal axes,
\begin{equation}
  \begin{aligned}
    U_{\rm PMNS}&=
    \left(\begin{array}{ccc}
      1 & 0 & 0 \\
      0 & c_{23} & s_{23} \\
      0 & -s_{23} & c_{23}
    \end{array}\right)
    \left(\begin{array}{ccc}
      c_{13} & 0 & s_{13}e^{-i\delta} \\
      0 & 1 & 0 \\
      -s_{13}e^{i\delta} & 0 & c_{13}
    \end{array}\right)
    \left(\begin{array}{ccc}
      c_{12} & s_{12} & 0 \\
      -s_{12} & c_{12} & 0 \\
      0 & 0 & 1
    \end{array}\right)\\
    &=
    \left(\begin{array}{ccc}
      c_{12}c_{13} & s_{12}c_{13} & s_{13}e^{-i\delta} \\
      -s_{12}c_{23}-c_{12}s_{23}s_{13}e^{i\delta} 
       & c_{12}c_{23}-s_{12}s_{23}s_{13}e^{i\delta} & s_{23}c_{13} \\
      s_{12}s_{23}-c_{12}c_{23}s_{13}e^{i\delta} 
       & -c_{12}s_{23}-s_{12}c_{23}s_{13}e^{i\delta} & c_{23}c_{13}
    \end{array}\right),
  \end{aligned}
\end{equation}
where $c_{ij}\equiv \cos\theta_{ij}$ and
$s_{ij}\equiv \sin\theta_{ij}$. As a final note, the above computations
were carried out under the assumption that neutrinos correspond to Dirac
spinors. If they correspond to Majorana spinors, two more degrees of freedom
appear, and we have to multiply the PMNS matrix on the right by a diagonal
matrix with three phases, $\alpha_1/2$, $\alpha_2/2$, and 0.

\subsection{Three flavors}
Let us explore the mechanism behind neutrino oscillation in more detail.
The mass eigenstates propagate as plane waves
\begin{equation}
  \label{eq:evolve1}
  \ket{\nu_i(t)}=\ket{\nu_i}e^{-ip_ix},
\end{equation}
where $1\leq i\leq3$. Suppose that at $x=0$ an electron neutrino
is produced in a weak process and let $\phi_i=p_i~x$. Then from \equatref{eq:mix1} 
and \eqref{eq:evolve1}, the wavefunction at an arbitrary
space-time point $x$ is
\begin{equation}
  \begin{aligned}
    \ket{\psi(x)}&=U_{e1}^*\ket{\nu_1}e^{-i\phi_1}
                  +U_{e2}^*\ket{\nu_2}e^{-i\phi_2}
                  +U_{e3}^*\ket{\nu_3}e^{-i\phi_3} \\
                 &=U_{e1}^*(U_{e1}\ket{\nu_e}+U_{\mu 1}\ket{\nu_\mu}
                   +U_{\tau 1}\ket{\nu_\tau})e^{-i\phi_1} \\
                 &~~~~~~
                  +U_{e2}^*(U_{e2}\ket{\nu_e}+U_{\mu 2}\ket{\nu_\mu}
                   +U_{\tau 2}\ket{\nu_\tau})e^{-i\phi_2} \\
                 &~~~~~~
                  +U_{e3}^*(U_{e3}\ket{\nu_e}+U_{\mu 3}\ket{\nu_\mu}
                   +U_{\tau 3}\ket{\nu_\tau})e^{-i\phi_3} \\
                 &=\left(U_{e1}^*U_{e1}e^{-i\phi_1}
                    +U_{e2}^*U_{e2}e^{-i\phi_2}
                    +U_{e3}^*U_{e3}e^{-i\phi_3}\right)\ket{\nu_e} \\
                 &~~~~~~
                   +\left(U_{e1}^*U_{\mu 1}e^{-i\phi_1}
                    +U_{e2}^*U_{\mu 2}e^{-i\phi_2}
                    +U_{e3}^*U_{\mu 3}e^{-i\phi_3}\right)\ket{\nu_\mu} \\
                 &~~~~~~
                   +\left(U_{e1}^*U_{\tau 1}e^{-i\phi_1}
                    +U_{e2}^*U_{\tau 2}e^{-i\phi_2}
                    +U_{e3}^*U_{\tau 3}e^{-i\phi_3}\right)\ket{\nu_\tau}.
  \end{aligned}
\end{equation}
We can extract oscillation probabilities from the above equation by projecting
the state vector on weak eigenstates. For instance the probability of finding
that our initial electron neutrino has oscillated into a muon neutrino at $x$ is
\begin{equation}
  \pr{\nu_e\to\nu_\mu}=\left|U_{e1}^*U_{\mu 1}e^{-i\phi_1}
                    +U_{e2}^*U_{\mu 2}e^{-i\phi_2}
                    +U_{e3}^*U_{\mu 3}e^{-i\phi_3}\right|^2.
\end{equation}
Using the unitarity relations among the PMNS matrix elements, along with the
identity
\begin{equation}
  |z_1+z_2+z_3|^2=|z_1|^2+|z_2|^2+|z_3|^2
                  +2\Re{z_1z_2^*+z_1z_3^*+z_2z_3^*},
\end{equation}
the above oscillation probability simplifies to
\begin{equation}
  \begin{aligned}
    \pr{\nu_e\to\nu_\mu}=&
    \,2\Re{U_{e1}^*U_{\mu 1}U_{e2}U_{\mu 2}^*\left(e^{i(\phi_2-\phi_1)}-1\right)}
     \\&~+
    2\Re{U_{e1}^*U_{\mu 1}U_{e3}U_{\mu 3}^*\left(e^{i(\phi_3-\phi_1)}-1\right)}
     \\&~+
    2\Re{U_{e2}^*U_{\mu 2}U_{e3}U_{\mu 3}^*\left(e^{i(\phi_3-\phi_2)}-1\right)}
     .
  \end{aligned}
\end{equation}
Similarly the electron neutrino survival probability is
\begin{equation}
  \begin{aligned}
    \pr{\nu_e\to\nu_e}=1&+2|U_{e1}|^2|U_{e2}|^2\Re{e^{i(\phi_2-\phi_1)}-1} \\
                        &+2|U_{e1}|^2|U_{e3}|^2\Re{e^{i(\phi_3-\phi_1)}-1} \\
                        &+2|U_{e2}|^2|U_{e3}|^2\Re{e^{i(\phi_3-\phi_2)}-1}.
  \end{aligned}
\end{equation}
To simplify these equations, define phase differences by
\begin{equation}
  \label{eq:phase}
  \Delta_{ji}\equiv\frac{\phi_j-\phi_i}{2}
                  =\frac{(m_j^2-m_i^2)|\vec{x}|}{4E_\nu}.
\end{equation}
The equality follows from the facts that
\begin{equation}
  \phi_j-\phi_i=(E_j-E_i)t-(|\vec{p}_1|-|\vec{p}_2|)|\vec{x}|
\end{equation}
and $t\approx|\vec{x}|$ since $\beta\approx1$.
Using the phase differences, we can recast the electron survival probability as
\begin{equation}
  \label{eq:esurv}
  \begin{aligned}
    \pr{\nu_e\to\nu_e}=1&-4|U_{e1}|^2|U_{e2}|^2\sin^2\Delta_{21} \\
                        &-4|U_{e1}|^2|U_{e3}|^2\sin^2\Delta_{31} \\
                        &-4|U_{e2}|^2|U_{e3}|^2\sin^2\Delta_{32}.
  \end{aligned}
\end{equation}
We see that when the phase differences are near zero, the survival probability
is close to 1. Since the masses of the neutrinos are small compared to their
energies, the mass differences are also small. This means that the survival
probability is nearly unity, except over large distances, which proffers
an explanation for why these oscillations did not manifest in early neutrino
beam experiments. We also see that if neutrinos are massless and this model is
correct, then there can be no oscillations. Since we have good evidence that
neutrino oscillations occur, and because we also have evidence that this model
accurately describes them, it follows that neutrinos have nonzero,
non-degenerate masses.

\subsection{Two flavors}
The above formalism gives the general treatment of neutrino oscillations
of three flavors, but for the present neutrino data, it is essentially correct
to assume only two flavors \cite{kajita_neutrino_2009}. 
For this reason, we will briefly review two flavor oscillations.

When there are two flavors, the relationship between the mass and weak bases
can be parameterized by a mixing angle $\theta$, which is analogous to the
Cabibbo angle. We have
\begin{equation}
  \colvec{2}{\nu_e}{\nu_\mu}=
  \left(\begin{array}{cc}
    \co_\theta & \s_\theta \\
    -\s_\theta & \co_\theta
  \end{array}\right)
  \colvec{2}{\nu_1}{\nu_2}.
\end{equation}
Again assuming the particle starts off as an electron neutrino at $x=0$, and
following the same procedure as before, we find the wavefunction at a general
space-time point to be
\begin{equation}
  \begin{aligned}
    \ket{\psi(x)}&=\co_\theta(\co_\theta\ket{\nu_e}-\s_\theta\ket{\nu_\mu})
                   e^{-i\phi_1}
                  +\s_\theta(\s_\theta\ket{\nu_e}+\co_\theta\ket{\nu_\mu})
                   e^{-i\phi_2} \\
                 &=e^{-i\phi_1}\left((\co_\theta^2
                        +e^{i\Delta\phi_{12}}\s^2_\theta)\ket{\nu_e}
                        -(1-e^{i\Delta\phi_{12}})
                         \co_\theta\s_\theta\ket{\nu_\mu}\right),
  \end{aligned}
\end{equation}
where $\Delta\phi_{12}\equiv\phi_1-\phi_2$. Projecting the state vector on
the muon state, applying \equatref{eq:phase}, and finally expressing
$|\vec{x}|$ in [km], $\Delta m$ in [eV], and $E_\nu$ in [GeV] gives the
familiar oscillation probability
\begin{equation}
  \pr{\nu_e\to\nu_\mu}=\sin^2(2\theta)\sin^2
                          \left(1.27\frac{\Delta m^2[\text{eV}^2]
                          |\vec{x}|[\text{km}]}{E_\nu[\text{GeV}]}\right).
\end{equation}

\subsection{Neutrino masses}
From the previous discussion, we see that measurements of neutrino oscillations
place no constraint on the overall mass scale. No direct measurement has been
made of neutrino masses. Nevertheless, cosmological measurements suggest
that \cite{thomson_modern_2013}
\begin{equation}
  m_{\nu1}+m_{\nu2}+m_{\nu3}\lesssim1~\text{eV}. 
\end{equation}
From what we can tell,
neutrino masses are much smaller than other fermion masses. Now let
$\Delta m_{ji}=m_{\nu j}-m_{\nu i}$.
Recent oscillation experiments have determined $\Delta m_{21}^2$ and
$|\Delta m_{32}^2|$ and shown that $\Delta m_{21}^2\ll|\Delta m_{32}^2|$.
These experiments have not been able to determine the sign of
$\Delta m_{32}^2$, an issue which has led to the definition of two {\it mass
hierarchies}. \index{neutrino!mass hierarchy}
In the {\it normal} hierarchy, $m_{\nu1}<m_{\nu2}<m_{\nu3}$,
while in the {\it inverted} hierarchy, $m_{\nu3}<m_{\nu1}<m_{\nu2}$. In
either case, it's clear that $|\Delta m_{31}^2|\approx|\Delta m_{32}^2|$.

The current explanation for the disparity between the masses of the neutrinos
and other fermions, and for how neutrinos acquire masses in the
first place, is as follows. Right-handed neutrinos $\nu_R$ do not
participate in any SM interactions, so
there is no evidence for or against their existence. Therefore we can introduce
neutrino masses in the same way as quarks. After spontaneous symmetry breaking
of the Higgs, the mass term for the neutrino looks like
\begin{equation}
  \Lagr_D=-m_D(\bar{\nu}_R\nu_L+\bar{\nu}_L\nu_R).
\end{equation}
If this term is the origin of masses, then right handed neutrinos must exist.
Nevertheless this explanation is not totally satisfactory because neutrino
masses are small compared to the masses of other fermions. So we add
the mass term
\begin{equation}
  \Lagr_M=-\frac{1}{2}M\left(\bar{\nu}^c_R\nu_R+\bar{\nu}_R\nu^c_R\right),
\end{equation}
where $\nu^c=CP\nu=i\gamma^2\gamma^0\nu^*$. $\Lagr_M$ does not violate local
gauge invariance. Similar terms are forbidden for charged leptons because
they allow particle-antiparticle interactions, which violate charge
conservation. This is no contradiction for the neutral neutrino, which may
very well correspond to a Majorana spinor, in which case it is its own
antiparticle anyway.

Now suppose we add to the SM Lagrangian
\begin{equation}
  \begin{aligned}
    \Lagr_{\nu~\text{mass}}&=-\frac{1}{2}\left(m_D\bar{\nu}_L\nu_R
                            +m_D\bar{\nu}^c_R\nu^c_L+M\bar{\nu}^c_R\nu_R\right)
                             +~\text{h.c.}\\
        &=\left(\bar{\nu}_L,\bar{\nu}_R^c\right)
          \left(\begin{array}{cc}
            0 & m_D \\
            m_D & M
          \end{array}\right)
          \colvec{2}{\nu_L^c}{\nu_R}+~\text{h.c.},
  \end{aligned}
\end{equation}
where the h.c. indicates the hermitian conjugate of everything that precedes it.
Following the usual procedure we obtain particle masses by diagonalizing the
above mass matrix. We find mass eigenvalues
\begin{equation}
  m=\frac{1}{2}M\left(1\pm\sqrt{1+\frac{4m_D^2}{M^2}}\right)
                \approx\frac{1}{2}M
                   \pm\frac{1}{2}M\left(1+\frac{2m_D^2}{M^2}\right)
\end{equation}
assuming $m_D\ll M$. This process reveals light and heavy neutrino states with
masses
\begin{equation}
  |m_\nu|\approx\frac{m_D}{M}\qquad\text{and}\qquad m_N\approx M.
\end{equation}
The {\it seesaw mechanism} hypothesizes that $m_D$ is of the same order as the
\index{seesaw mechanism}
other fermions, a feature that we like, and that the observed neutrino mass is
so small because the Majorana mass $M$ is large and suppresses it. There is
currently no direct evidence for the seesaw mechanism, but if neutrinos were
discovered to be Majorana particles, the mechanism would be even more
compelling.

\section{Neutrino experiments}
We will now examine some experiments that confirm neutrino oscillations and
have allowed us to measure parameters in the PMNS matrix.
Two subclasses of neutrino experiments are reactor experiments
and accelerator experiments. In reactor experiments, nuclear fission produces
radioisotopes, which then emit electron neutrinos during $\beta$ decays.
Electron antineutrinos are detected through the process
\begin{equation}
  \label{eq:betadecay}
  \bar{\nu}_e+\text{p}\to e^++\text{n},
\end{equation}
but they will not be detected if they oscillate to other flavors, because
the neutrino energy will be too small to produce a muon or tau lepton in
the final state. Thus these experiments can only observe the disappearance
of electron antineutrinos. In beam experiments, a highly relativistic
proton beam is fired at a target, which produces a large number of charged
pions, which subsequently decay via
\begin{equation}
  \begin{aligned}
    \pi^-&\to\mu^-+\bar{\nu}_\mu, \\
    \pi^+&\to\mu^++\nu_\mu.
  \end{aligned}
\end{equation}
The neutrinos and antineutrinos follow the direction of the CM frame boost,
which is in the direction of the incoming pion.

To prepare to analyze more neutrino experiments, we calculate the
electron antineutrino survival probability.
The action of the $T$ operator on the process $\nu_\ell\to\nu_{\ell'}$ is
to interchange the labels $\ell$ and $\ell'$, while $CP$ complex conjugates
the elements of the PMNS matrix. From \equatref{eq:esurv} it follows
that $\pr{\nu_e\to\nu_e}=\pr{\bar{\nu}_e\to\bar{\nu}_e}$. Using the fact
that $|\Delta m_{31}^2|\approx|\Delta m_{32}^2|$ and the unitarity relations,
we can then write
\begin{equation}
  \label{eq:prb}
  \begin{aligned}
    {\rm P}_{\rm surv.}&\equiv\pr{\bar{\nu}_e\to\bar{\nu}_e}\\
      &\approx 1-4|U_{e1}|^2|U_{e2}|^2\sin^2\Delta_{21}
              -4|U_{e3}|^2\left(|U_{e1}|^2+|U_{e2}|^2\right)\sin^2\Delta_{32}
       \\ 
      &= 1-c_{13}^4\sin^2(2\theta_{12})
           \sin^2\left(\frac{\Delta m_{21}^2 |\vec{x}|}{4E_{\bar{\nu}}}
           \right) -\sin^2(2\theta_{13})
           \sin^2\left(\frac{\Delta m_{32}^2 |\vec{x}|}{4E_{\bar{\nu}}}
           \right).
  \end{aligned}
\end{equation}
The second sine factor in the last term has a shorter wavelength than the
second sine factor in the second term because of the relative sizes of the
mass differences. The last term therefore dominates at shorter distances,
while the second term dominates at longer distances. \figref{fig:kamland}
(left) shows a graph of the survival probability assuming a few typical 
parameter values.

\subsection{Reactor experiments}

\begin{figure}
  \centering
  \vspace{-15mm}
  \includegraphics[width=0.9\textwidth,keepaspectratio]
                {pictures/t13_19.pdf}
  \vspace*{-20mm}
  \caption{Left: Daya Bay observed antineutrino rate compared to the
           unoscillated expectation (dotted line). Rates plotted as function
           of flux weighted distance to the reactors (solid line). Right:
           Daya Bay observed near and far e$^+$ energy spectra with the
           background subtracted. Image taken from Thomson Fig. 
           13.19~\cite{thomson_modern_2013}.}
  \label{fig:daya}
\end{figure}

\begin{figure}
  \centering
  \vspace{-20mm}
  \includegraphics[width=0.9\textwidth,keepaspectratio]
                {pictures/t13_18&20.pdf}
  \vspace*{-15mm}
  \caption{Left: The electron antineutrino survival probability assuming
           $\theta_{12}=12$\textdegree, $\theta_{23}=45$\textdegree,
           $\theta_{13}=10$\textdegree, $\Delta m_{21}^2=8\times10^{-5}$
           $\text{eV}^2$, $\Delta m_{32}^2=2.5\time10^{-3}$ $\text{eV}^2$.
           Right: KamLAND data showing measured mean survival probability.
           The histogram is generated from the expected distribution given
           by the oscillation parameters that best fit the data, taking
           background into account. Images taken from Thomson Fig.
           13.18 and 13.20~\cite{thomson_modern_2013}.}
  \label{fig:kamland}
\end{figure}

The Daya Bay experiment is an international reactor experiment based in China.
It consists of
six nuclear reactors and eight antineutrino detectors placed short, varied
distances (less than 2 km) away from the source. Hence the short
wavelength contribution of \equatref{eq:prb} dominates. Placing the
detectors at different distances allows cancellation of some systematic
uncertainty. The detectors are full of 20 tons of
liquid {\it scintillator}, a material that produces light when it interacts
with a particular particle but doesn't interfere with light itself.
The scintillator is doped with gadolinium and viewed by PMTs, which look for
the process \eqref{eq:betadecay}. Ascertaining whether such an event occurred
involves multiple steps. First the annihilation of the positron with an
electron produces two photons. The neutron scatters in the scintillator until
it is captured by a gadolinium nucleus, which takes roughly 100 $\mu$s and
produces photons. Photons from
the capture and the electron-positron annihilation yield Compton scattered
electrons, which then ionize the scintillator and produce scintillation light.
Hence the experiment counts an inverse beta decay event whenever it detects
a pulse of scintillation light followed by a neutron capture pulse
10-100~$\mu$s later.

The results of the Daya Bay experiment are summarized in \figref{fig:daya}. 
The left plot compares the observed number of antineutrino events with the
unoscillated hypothesis. There is an unambiguous deviation
that increases with increasing detector distance.
The plot on the right compares the observed e$^+$ energy spectrum in the far
detectors to the spectrum of the near detectors, scaled to the same integrated
neutrino flux. Again the data are consistent with some electron neutrinos
oscillating to other flavors by the time they reach the far detector.
This energy difference then allows a measurement of $\theta_{13}$, and Daya Bay
found \cite{an_observation_2012}
\begin{equation}
  \sin^2(2\theta_{13})=0.092\pm0.016({\rm stat})\pm0.005({\rm sys}).
\end{equation}

The Kamioka Liquid Scintillator Antineutrino Detector (KamLAND) experiment,
located in the same mine as Super-K, consists of two concentric spheres in
water. The inner sphere is full of liquid scintillator and again surrounded by
PMTs, and neutrinos are detected using the same procedure as in Daya Bay.
However instead of gadolinium, the neutron capture occurs through the
process $\bar{\nu}_e+\text{p}\to\text{D}+\gamma$. The detector is situated
130-240 km away from the reactors, so that the long wavelength contribution of
\equatref{eq:prb} is the dominant term. The KamLAND data, depicted in the right
plot of \figref{fig:kamland}, form a rather nice pattern. 
The plot clearly shows electron
antineutrino survival probabilities that oscillate with $L_0/E_{\bar{\nu}e}$,
where $L_0$ is the flux weighted mean distance from the detector to the
reactors. By comparing this distribution with the long wavelength dominated
survival probability, KamLAND found \cite{abe_precision_2008}
\begin{equation}
  \Delta m_{12}^2=7.58^{+0.14}_{-0.13}({\rm stat})
        \pm0.15({\rm sys})\times10^{-5}~\text{eV}^2.
\end{equation}
KamLAND also determined
\begin{equation}
  \tan^2\theta_{12}=0.56^{+0.10}_{-0.07}({\rm stat})^{+0.10}_{-0.06}({\rm sys}).
\end{equation}

\subsection{Accelerator experiments}
Long baseline accelerator experiments typically have two detectors, one
situated near the source, which reveals the unoscillated neutrino energy
spectrum, and one far away, which reveals the oscillated spectrum.
Many systematic uncertainties cancel when both a near and far detector are
used.
An important accelerator experiment is the Main Injector Neutrino Oscillation
Search (MINOS) at Fermilab. The experiment measures oscillations
of a pure muon neutrino beam. It started collecting data in 2005 and
was upgraded to MINOS+ in 2013. The near detector is situated 1 km
from the source, while the far detector is located in a mine in Minnesota
735 km away. The detectors are made of iron plates with relatively
thin layers of
plastic scintillator. Scintillation light is passed from the scintillator to
PMTs using optical fibers, and the detector is magnetized to measure
muon momentum.

\begin{figure}
  \centering
  \includegraphics[width=0.90\textwidth,height=0.90\textheight,keepaspectratio]
                {pictures/t13_22.pdf}
  \vspace*{-20mm}
  \caption{Left: MINOS far detector energy spectrum and unoscillated
           prediction (dashed). Right: Muon neutrino survival probability
           as measured from the left figure. Image taken from Thomson
           Fig. 13.22~\cite{thomson_modern_2013}.}
  \label{fig:minos}
\end{figure}

We can calculate the muon neutrino survival probability for MINOS
using the same arguments as the beginning of this section. We can write
\begin{equation}
  \label{eq:sexyprob}
  \begin{aligned}
    \pr{\nu_\mu\to\nu_\mu}
      &\approx 1-4|U_{\mu1}|^2|U_{\mu2}|^2\sin^2\Delta_{21}
              -4|U_{\mu3}|^2\left(1-|U_{\mu3}|^2\right)\sin^2\Delta_{32} \\
      &\approx 1-\left(\sin^2(2\theta_{23})c_{13}^4
                       +\sin^2(2\theta_{13})s_{23}^2)\right)
                        \sin^2\Delta_{32},
  \end{aligned}
\end{equation}
where in the last step we have utilized the fact that the long wavelength
component can be ignored for MINOS. \figref{fig:minos} shows the 
results of the experiment. The left plot compares the observed far 
detector energy spectrum
with the unoscillated prediction, which gives a direct measurement of the
muon neutrino survival probability, shown on the right. According to 
\equatref{eq:sexyprob}, the minimum of the right plot yields
\cite{adamson_measurement_2011}
\begin{equation}
  |\Delta m_{32}^2|=(2.32^{+0.12}_{-0.08})\times10^{-3}~\text{eV}^2,
\end{equation}
and the best fit for $\theta_{23}$ yields the constraint
\begin{equation}
  \sin^2(2\theta_{23})\gtrsim0.90\
\end{equation}
at the 90\% confidence level.

\section{Neutrinoless double beta decay}

\begin{figure}[t]\label{fig:ndbd}
\centering
\includegraphics[width=0.6\textwidth]{figs/2560px-Double_beta_decay_feynman.svg.png}
\caption{Example neutrinoless double beta decay diagram. Image taken
         from Wikipedia \cite{wiki:ndbd}.}
\end{figure}

Neutrinoless double beta decay ($0\nu\beta\beta$) is a proposed decay
process that would verify that neutrinos are Majorana fermions, if
it were discovered. Two neutrons can become two protons by emitting
two $e^-$ and two $\nu_e$; finding $\nu_e$ in the final state is
an example of neutrinoful double beta decay. If $\nu_e$ is a
Majorana fermion, i.e. if it is its own antiparticle, then the
emitted $\nu_e$ from one interaction can be absorbed by the other.
This is the neutrinoless double beta decay; its Feynman diagram
is shown in \figref{fig:ndbd}.


\section{Conclusions, outlook}
As has been demonstrated, the experimental evidence supporting the existence of
neutrino oscillations is very strong. The three flavor oscillation model seems
to predict the data quite well, and from the above experiments, we've (more or
less) determined the values of the parameters of the PMNS matrix.
The success of the model has been a triumph for particle
physicists, but there is still more we need to learn. For example we don't yet
know whether the neutrino is a Majorana spinor, which could shed light on the
viability of the seesaw mechanism. We also have not yet found the
phase $\delta$ of the PMNS matrix. The SM could accommodate another light
neutrino species. And of course there is still room to pin down the exact
value for $\theta_{23}$. The sheer number of neutrino experiments currently
performed and in progress, along with the 2015 Nobel Prize, paints an
optimistic picture for the future of neutrino experiments.

\bibliographystyle{unsrtnat}
\bibliography{bibliography}
