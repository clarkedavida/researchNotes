\chapter{LFT: Fermions}
We will now introduce Fermions on the lattice. This presentation follows
Chapter 5 of~\cite{gattringer_quantum_2010}. We will start in the continuum
and introduce a naive discretization. Then we will go into detail, point
out a problem with the naive discretization, and fix it. We will omit
space-time dependence sometimes for the sake of brevity, but it should
be clear which objects have space-time dependence anyway. Primarily
we are interested in studying QCD, so $N_c=3$. This chapter uses Dirac
algebra extensively, but the algebra is different when the metric
is Euclidean. For details see Appendix~\ref{ap:spec_math}.

In the continuum theory, the free fermion propagator is
\begin{equation}
  S_F=\int\dd[4]{x}\bar{\psi}(\slashed{\partial}+m)\psi.
\end{equation}
Using the rules~\eqref{eq:dertodiff} and \eqref{eq:inttosum}, one naively
discretizes this action on the lattice as
\begin{equation}\label{eq:naivefermact}
  S_F=a^4\sum_x\bar{\psi}(x)\left(\sum_{\mu=1}^4\gamma_\mu
       \frac{\psi(x+a\hat{\mu})-\psi(x-a\hat{\mu})}{2a}
       +m\psi(x)\right).
\end{equation}
The fermionic partition function is
\begin{equation}
  Z_F=\int\DD{\psi}\DD{\bar{\psi}}e^{-S_F},
\end{equation}
with integration measure
\begin{equation}
  \int\DD{\psi}\DD{\bar{\psi}}
   =\prod_{x,f,\alpha,c}\dd{\bar{\psi}^f_{\alpha c}(x)}
                        \dd{\psi^f_{\alpha c}(x)},
\end{equation}
where $f$ runs over flavors, $\alpha$ runs over Dirac indices, and $c$
runs over colors.

In addition, we want the fermionic action to be gauge invariant. The
gauge transformation for fermion fields
\begin{equation}
  \psi(x)\to U(x)\psi(x),~~~~~\bar{\psi}(x)\to\bar{\psi}(x)U(x)^\dagger
\end{equation}
with $U\in\SU(3)$ reminds us of the gauge transformation for scalar 
fields given in Chapter~\ref{ch:preliminaries}. Working in the continuum 
we introduce, just as before, a gauge field $A_\mu(x)$ so that the 
fermionic action becomes
\begin{equation}
  S_F=\int\dd[4]{x}\bar{\psi}\big(\gamma_\mu(\partial_\mu+A_\mu)+m\big)\psi.
\end{equation}
The gauge-transformed fermionic action is
\begin{equation}
  S'_F=\int\dd[4]{x}\bar{\psi}\,U^\dagger
        \big(\gamma_\mu(\partial_\mu+A'_\mu)+m\big)U\psi,
\end{equation}
so that the requirement of gauge invariance leads us to conclude
\begin{equation}
  (\partial_\mu+A_\mu)\psi=U^\dagger(\partial_\mu+A'_\mu)U\psi.
\end{equation}
Solving for $A'_\mu$ we find
\begin{equation}
  A'_\mu=(\partial_\mu U)U^\dagger-UA_\mu U^\dagger,
\end{equation}
just as we did before.

On the lattice, link variables transform as
\begin{equation}
  U_\mu(x)\to U'_\mu(x)=U(x)U_\mu(x)U^\dagger(x+a\hat{\mu}),
\end{equation}
which ensures that closed loops of link variables are gauge invariant.
In addition the discretized fermionic action becomes gauge invariant
when it is written as
\begin{equation}\label{eq:naivefermactgauge}
  S_F=a^4\sum_x\bar{\psi}(x)\left(\sum_{\mu=1}^4\gamma_\mu
       \frac{U_\mu(x)\psi(x+a\hat{\mu})-U_{-\mu}(x)\psi(x-a\hat{\mu})}{2a}
       +m\psi(x)\right),
\end{equation}
where
\begin{equation}
  U_{-\mu}(x)=U_\mu(x-a\hat{\mu})^\dagger
\end{equation}
transforms as
\begin{equation}
  U_{-\mu}(x)\to U'_{-\mu}(x)=U(x)U_{-\mu}(x)U^\dagger(x-a\hat{\mu}).
\end{equation}
Since eq.~\eqref{eq:naivefermactgauge} is the gauge invariant action,
this will be what we investigate for fermion doubling.

\section{Grassmann numbers}\index{Grassman algebra}
Fermions are objects that by definition anticommute with each other. 
With this in mind, we introduce Grassmann numbers.
We consider a set of numbers $\eta_i$, $1\leq i\leq N$ obeying
\begin{equation}
  \eta_i\eta_j=-\eta_j\eta_i
\end{equation}
for all $i$ and $j$. These are called {\it Grassmann numbers}.
It follows that
\begin{equation}\label{eq:gnilpotent}
  \eta_i^2=0.
\end{equation}
If a power series of a function $f$ of the Grassmann numbers exists, then
eq.~\eqref{eq:gnilpotent} guarantees that the series terminates almost
immediately. In general we would write
\begin{equation}\label{eq:grasspoly}
  f(\eta)=a+\sum_ia_i\eta_i+\sum_{i<j}a_{ij}\eta_i\eta_j
           +...+a_{1...N}\eta_1...\eta_N
\end{equation}
with $a,\,a_i,\,a_{ij},...,\,a_{1...N}\in\mathbb{C}$.
We refer to eq.~\eqref{eq:grasspoly} as a {\it Grassmann polynomial}.
Grassmann polynomials are closed under addition and multiplication,
and are thus said to form a {\it Grassmann algebra}. The $\eta_i$
are the {\it generators} of the Grassmann algebra.

To learn how differentiation ought to work, we use the simple example
$N=2$. Then
\begin{equation}
  f(\eta)=a+a_1\eta_1+a_{12}\eta_1\eta_2.
\end{equation}
A definition of the derivative that follows our intuition is
\begin{equation}
  \frac{\partial f}{\partial\eta_1}=a_1+a_{12}\eta_2.
\end{equation}
However, the defining characteristic of Grassmann numbers is that they
anticommute. Therefore we could also have expanded $f$ as
\begin{equation}
  f(\eta)=a+a_1\eta_1-a_{12}\eta_2\eta_1.
\end{equation}
In order for the derivative $f$ to make sense we must therefore require
\begin{equation}
  \frac{\partial}{\partial\eta_1}\eta_2=
  -\eta_2\frac{\partial}{\partial\eta_1}.
\end{equation}
Similarly if we take another derivative, this time with respect to $\eta_2$,
we find that partial derivatives with respect to different Grassmann variables
must also anticommute to maintain consistency. Altogether, the differentiation
rules for Grassmann variables become
\begin{equation}\begin{aligned}
  \frac{\partial}{\partial\eta_i}1&=0;\\
  \frac{\partial}{\partial\eta_i}\eta_i&=1;\\
  \frac{\partial}{\partial\eta_i}\eta_j&=
  -\eta_j\frac{\partial}{\partial\eta_i};\\
  \frac{\partial^2}{\partial\eta_i\partial\eta_j}&=
  -\frac{\partial^2}{\partial\eta_j\partial\eta_i}.
\end{aligned}\end{equation}

Next we move on to integration. We will construct integrals that work like
integrals over subsets $\Omega\in\mathbb{R}^N$ for which 
the integrand vanishes at the boundary $\partial\Omega$. 
Thus we demand that the Grassmann integral is a complex number,
\begin{equation}
  \int\dd[N]{\eta}f\in\mathbb{C};
\end{equation}
that with $\lambda_1,\lambda_2\in\mathbb{C}$ it is linear,
\begin{equation}\label{eq:grasslin}
  \int\dd[N]{\eta}\left(\lambda_1f_1+\lambda_2f_2\right)
       =\lambda_1\int\dd[N]{\eta}f_1
         +\lambda_2\int\dd[N]{\eta}f_2;
\end{equation}
and that its integrand vanishes at the boundary,
\begin{equation}\label{eq:grassBC}
  \int\dd[N]{\eta}\frac{\partial}{\partial\eta_i}f=0.
\end{equation}
If some function $f$ of $N-1$ Grassmann numbers can be written as the 
derivative of another function $g$ of $N$ Grassmann numbers, it follows
from eq.~\eqref{eq:grassBC} that the integrand vanishes.
An integral of a Grassmann polynomial of $N$ variables is therefore
proportional to the coefficient $a_{1...N}$, since
$\eta_1...\eta_N$ cannot be written as a derivative of $N$ variables.
In particular if we demand a normalization
\begin{equation}\label{eq:grassnorm}
  \int\dd[N]{\eta}\eta_1...\eta_N=1,
\end{equation}
we obtain the rule
\begin{equation}
  \int\dd[N]{\eta}f=a_{1...N}.
\end{equation}
Finally to make Grassmann integration more analogous to the integration that
we're used to, we define
\begin{equation}
  \dd[N]{\eta}\equiv\dd{\eta_N}...\dd{\eta_1}.
\end{equation}
Altogether then, the rules of Grassmann integration can be summarized by
eq.~\eqref{eq:grasslin} along with
\begin{equation}\label{eq:grassintrules}\begin{aligned}
       \int\dd{\eta_i}1&=0;\\
  \int\dd{\eta_i}\eta_i&=1;\\
  \dd{\eta_i}\dd{\eta_j}&=-\dd{\eta_j}\dd{\eta_i}.
\end{aligned}\end{equation}
Using these properties we can define integration over subsets of Grassmann
variables. Interestingly, the measures obey the same algebraic properties as
the derivatives.

Next let us discuss how Grassmann integrals behave under linear
transformations. In general we can write such a change of variables as
\begin{equation}\label{eq:grassxform}
  \eta'_i=M_{ij}\eta_j,
\end{equation}
where $M$ is a complex $N\times N$ matrix. More succinctly, one writes 
$\eta'=M\,\eta$. Applying this change of variables to the normalization
eq.~\eqref{eq:grassnorm} we obtain
\begin{equation}\begin{aligned}
  \int\dd[N]{\eta}\eta_1...\eta_N
     &=1\\
     &=\int\dd[N]{\eta'}\eta'_1...\eta'_N\\
     &=\int\dd[N]{\eta'}M_{1\,i_1}...M_{N\,i_N}\,\eta_{i_1}...\eta_{i_N}\\
     &=\int\dd[N]{\eta'}M_{1\,i_1}...M_{N\,i_N}\,\epsilon_{i_1...i_N}
               \eta_1...\eta_N\\
     &=\int\dd[N]{\eta'}\det M\,\eta_1...\eta_N.
\end{aligned}\end{equation}
It follows that under the transformation~\eqref{eq:grassxform}, the measure
transforms according to the rule
\begin{equation}
  \dd[N]{\eta}=\det M\dd[N]{\eta'},
\end{equation}
which is in some sense the ``opposite" of the usual transformation rule, where
the $\det M$ would have been on the LHS.

The stuff between the parentheses in the naive action~\eqref{eq:naivefermact}
can be viewed as an operator. Keeping this in mind, we're going to show
that the fermionic partition function can be viewed as a determinant.
The setup is as follows: We have a Grassmann algebra with 2N generators
$\bar{\eta}_i$ and $\eta_i$ that all anticommute with each other and
an $N\times N$ linear transformation $M$.
\begin{theorem}{Matthews-Salam formula}{}
\index{Matthews-Salam formula}
$$
  \int\prod_{i=1}^N\dd{\eta_i}\dd{\bar{\eta}_i}e^{\bar{\eta}M\eta}=\det M.
$$
\begin{proof} Let $\eta'=M\eta$. Then from eq.~\eqref{eq:grassxform}
  we have
  \begin{equation*}\begin{aligned}
  \int \prod_{i=1}^N\dd{\eta_i}\dd{\bar{\eta}_i}e^{\bar{\eta}M\eta}
    &=\det M\int \prod_{i=1}^N\dd{\eta'_i}\dd{\bar{\eta}_i}
       \exp\left(\sum_{j=1}^N\bar{\eta}_j\eta'_j\right)\\
    &=\det M\int \prod_{i=1}^N\dd{\eta'_i}\dd{\bar{\eta}_i}
       \exp\left(\bar{\eta}_i\eta'_i\right)\\
    &=\det M\int \prod_{i=1}^N\dd{\eta'_i}\dd{\bar{\eta}_i}
        (1+\bar{\eta}_i\eta'_i)\\
    &=\det M.
  \end{aligned}\end{equation*}
  The second line follows since pairs $\bar{\eta}_i\eta_i$ commute with
  each other, the third line is a power series expansion of the second
  line, and the last line follows from eq.~\eqref{eq:grassintrules}.
\end{proof}
\end{theorem}
If we replace $M$ with the Dirac operator, we see that the fermionic
partition function is just the determinant of the Dirac operator.
By the way, the ordering of the differentials in the measure should
be $\dd{\eta}\dd{\bar{\eta}}$. This is because the $\bar{\eta}$ variables
in the integrand will always come on the left. By placing the
$\dd{\bar{\eta}}$ differentials on the right, it will always hit the
$\bar{\eta}$ variables first.

A generalization of this gives us the generating functional
for fermions. Here we consider $4N$ Grassmann numbers $\bar{\eta}_i$,
$\eta_i$, $\bar{\theta}_i$, and $\theta_i$. The $\bar{\theta}_i$
and the $\theta_i$ are the source terms. We define
\begin{equation}\label{eq:deffermgf}
  W[\theta,\bar{\theta}]\equiv\int \prod_{i=1}^N\dd{\bar{\eta}_i}\dd{\eta_i}
   e^{\bar{\eta}M\eta+\bar{\theta}\eta+\bar{\eta}\theta}
\end{equation}
\begin{theorem}{}{fermgf}
$$
   W[\theta,\bar{\theta}]=e^{-\bar{\theta}M^{-1}\theta}\det M.
$$
\begin{proof} Using the definition of the generating functional and
completing the square, we have 
$$
  \int \prod_{i=1}^N\dd{\bar{\eta}_i}\dd{\eta_i}
   e^{\bar{\eta}M\eta+\bar{\theta}\eta+\bar{\eta}\theta}
   =\int \prod_{i=1}^N\dd{\bar{\eta}_i}\dd{\eta_i}
   e^{(\bar{\eta}+\bar{\theta}M^{-1})\,M\,
             (\eta+M^{-1}\theta)-\bar{\theta}M^{-1}\theta}.
$$
Now we define new variables $\bar{\eta}'=\bar{\eta}+\bar{\theta}M^{-1}$
and $\eta'=\eta+M^{-1}\theta$. From eq.~\eqref{eq:grassintrules} it follows
that
$$
  \int \prod_{i=1}^N\dd{\bar{\eta}_i}\dd{\eta_i}=
  \int \prod_{i=1}^N\dd{\bar{\eta}'_i}\dd{\eta'_i}.
$$
Therefore by the Matthews-Salam formula,
\begin{equation*}\begin{aligned}
   \int \prod_{i=1}^N\dd{\bar{\eta}_i}\dd{\eta_i}
   e^{(\bar{\eta}+\bar{\theta}M^{-1})\,M\,
             (\eta+M^{-1}\theta)-\bar{\theta}M^{-1}\theta}
   &=\int \prod_{i=1}^N\dd{\bar{\eta}'_i}\dd{\eta'_i}
    e^{\bar{\eta}'M\eta'-\bar{\theta}M^{-1}\theta}\\
   &=e^{-\bar{\theta}M^{-1}\theta}\det M.
\end{aligned}\end{equation*}
\end{proof}
\end{theorem}
With the knowledge we now have, we can derive Wick's theorem, which lets us
calculate fermionic expectation values. First we define 
\begin{equation}\label{eq:deffermev}
  \ev{\eta_{i_1}\bar{\eta}_{j_1}...\eta_{i_n}\bar{\eta}_{j_n}}_F
  \equiv\frac{1}{Z_F}\int\prod_{k=1}^N\dd{\eta_k}\dd{\bar{\eta}_k}
        \bar{\eta}_{j_1}...\eta_{i_n}\bar{\eta}_{j_n}
        e^{-\bar{\eta}M\eta}.
\end{equation}
\begin{theorem}{Wick's theorem}{}
\index{Wick's theorem}
$$
  \ev{\eta_{i_1}\bar{\eta}_{j_1}...\eta_{i_n}\bar{\eta}_{j_n}}_F
  =(-1)^n\sum_{P_{1,...,n}}\sign P\,
   M^{-1}_{i_1,j_{P_1}}\,...\,M^{-1}_{i_n,j_{P_n}}.
$$
\begin{proof} From the definitions~\eqref{eq:deffermgf} and
  \eqref{eq:deffermev} we have
  $$
    \ev{\eta_{i_1}\bar{\eta}_{j_1}...\eta_{i_n}\bar{\eta}_{j_n}}_F
    =\frac{1}{Z_F}
    \frac{\partial}{\partial\theta_{j_1}}
    \frac{\partial}{\partial\bar{\theta}_{i_1}}\,...\,
    \frac{\partial}{\partial\theta_{j_n}}
    \frac{\partial}{\partial\bar{\theta}_{i_n}}W[\theta,\bar{\theta}]
    \Big|_{\theta=\bar{\theta}=0}.
  $$
  Let us now apply Theorem~\ref{thm:fermgf} to the RHS and carry
  out the derivatives. The $\det M$ will cancel with $Z_F$ because
  of the Matthews-Salam formula. 
\end{proof}
\end{theorem}

\section{Fermion doubling}
Let us now discuss one of the important problems with the naive fermionic
action. First we introduce some results about Fourier transformations on the
lattice. Define
\begin{equation}
V\equiv N_1N_2N_3N_4
\end{equation}
with $N_\mu$ even.\index{BCs!toroidal}
We generalize to {\it toroidal BCs}, i.e.
\begin{equation}
  f(x+aN_\mu\hat{\mu})=e^{2\pi i\theta_\mu}f(x).
\end{equation}
\index{BCs!anti-periodic}
Periodic BCs then have $\theta_\mu=0$ and {\it anti-periodic BCs} have
$\theta_\mu=1/2$. The momentum space becomes
\begin{equation}
  p_\mu=\frac{2\pi}{aN_\mu}(k_\mu+\theta_\mu),~~~~~~
   -\frac{N_\mu}{2}<k_\mu\leq\frac{N_\mu}{2},
\end{equation}
which reduces to the first Brillouin zone when periodic BCs are employed. By
including the boundary phases in the momentum definition, plane waves
\begin{equation}
  \exp(ip_\mu x_\mu)
\end{equation}
will also obey the BCs. Now we derive a basic formula for Fourier
transformations on the lattice. Let $N$ be even and $\ell$ be an integer
$0\leq\ell\leq N-1$.
\begin{proposition}{}{}
$$
  \frac{1}{N}\sum_{j=-N/2+1}^{N/2}\exp(\frac{2\pi i\ell}{N})^j=\delta_{\ell 0}.
$$
\begin{proof} Since $N$ is even there are $N$ terms in the above sum,
  so we find the LHS to be 1. For $\ell\neq 0$ let
  $m\equiv j+N/2+1$ and define
  $$
    q\equiv\exp(\frac{2\pi i\ell}{N}).
  $$
  Then
  \begin{equation*}\begin{aligned}
    \frac{1}{N}\sum_{j=-N/2+1}^{N/2}\exp(\frac{2\pi i\ell}{N})
      \propto\sum_{m=0}^{N-1}q^m
      =\frac{1-q^N}{1-q}=0
  \end{aligned}\end{equation*}
  since $q^N=1$.
\end{proof}
\end{proposition}
Applying this formula in each space-time direction, we arrive at
\begin{equation}
  \frac{1}{V}\sum_p\exp\big(ip_\mu(x-x')_\mu\big)
  =\delta(x-x')
  =\delta_{n_1n'_1}\delta_{n_2n'_2}\delta_{n_3n'_3}\delta_{n_4n'_4},
\end{equation}
where $x_\mu=an_\mu$, and
\begin{equation}
  \frac{1}{V}\sum_x\exp\big(i(p-p')_\mu x_\mu\big)
  =\delta(p-p')
  =\delta_{k_1k'_1}\delta_{k_2k'_2}\delta_{k_3k'_3}\delta_{k_4k'_4}.
\end{equation}
We define the Fourier transform as
\begin{equation}\label{eq:latft}
  \tilde{f}(p)=\frac{1}{\sqrt{V}}\sum_xf(x)\exp(-ip_\mu x_\mu).
\end{equation}
The inverse transform
\begin{equation}\label{eq:latift}
  f(x)=\frac{1}{\sqrt{V}}\sum_p\tilde{f}(p)\exp(ip_\mu x_\mu)
\end{equation}
can be verified by plugging the Fourier transform into the LHS.

Using these results about Fourier transformations on the lattice, we
can begin to investigate fermion doubling. We shall do this for
a single flavor for notational convenience; the result clearly
generalizes to a summation over flavors. 
The gauge-invariant, naive
fermion action~\eqref{eq:naivefermactgauge} is bilinear in $\psi$
and $\bar{\psi}$, so it can be written as
\begin{equation}
  S_F=a^4\sum_{x,y}\sum_{\alpha,\beta,c_1,c_2}\bar{\psi}(x)_{\alpha c_1}
      \,D(x|y)_{\alpha \beta c_1 c_2}\,\psi(y)_{\beta c_2},
\end{equation}
where $\alpha$ and $\beta$ are Dirac indices, $c_1$ and $c_2$ are
color indices, and the Dirac operator on the lattice is
\begin{equation}
  D(x|y)_{\alpha\beta c_1c_2}\equiv\sum_{\mu=1}^4(\gamma_\mu)_{\alpha\beta}
   \frac{U_\mu(x)_{c_1c_2}\delta_{x+a\hat{\mu},y}
         -U_{-\mu}(x)_{c_1c_2}\delta_{x-a\hat{\mu},y}}{2a}
   +m\delta_{\alpha\beta}\delta_{c_1c_2}\delta_{xy}.
\end{equation}
In this form, we can apply the Matthews-Salam formula or Wick's
theorem by identifying $M=a^4D$. For simplicity, we set $U_\mu(x)=\id$
for all links, just so we can see clearly how the doubling arises.
Since we have free fermions, we may as well suppress color indices,
and we will also use vector/matrix notation in Dirac space, so that
Dirac indices are also suppressed. We find
\begin{equation}\begin{aligned}
  \tilde{D}(p|q)&=\frac{1}{V}\sum_{x,y}e^{-ipx}D(x|y)e^{-iqy}\\
     &=\frac{1}{V}\sum_{x,y}e^{-ipx}\left(
      \sum_\mu\gamma_\mu\frac{\delta_{x+a\hat{\mu},y}
                             -\delta_{x-a\hat{\mu},y}}{2a}+m\delta_{x,y}\id
      \right)e^{-iqy}\\
     &=\frac{1}{V}\sum_xe^{-i(p+q)x}\left(
      \sum_\mu\gamma_\mu\frac{e^{iq_\mu a}-e^{-iq_\mu a}}{2a}+m\id\right)\\
     &=\delta(p+q)\left(\frac{i}{a}\sum_\mu
                    \gamma_\mu\sin(q_\mu a)+m\id\right)\\
     &\equiv\delta(p+q)\tilde{D}(q).
\end{aligned}\end{equation}
In Gattringer and Lang~\cite{gattringer_quantum_2010} they take the transform
of the $y$ part to have opposite sign, which they say makes the similarity
transformation unitary. For the purpose of seeing fermion doubling
the sign does not make a difference; it only changes the delta
function from $\delta(p-q)$ to $\delta(p+q)$. Therefore I decided to stick
with the convention~\eqref{eq:latft}. If we want to use Wick's theorem,
we will need to calculate the inverse of the Dirac operator. We find
\begin{equation}\label{eq:latprop}
  \tilde{D}(p)^{-1}=\frac{m\id-ia^{-1}\sum_\mu\gamma_\mu\sin(p_\mu a)}
                         {m^2+a^{-2}\sum_\mu\sin(p_\mu a)^2},
\end{equation}
which can be verified by multiplying both sides by $\tilde{D}(p)$. This
is the propagator for free fermions in momentum space. At $m=0$ we find
as $a\to0$
\begin{equation}
  \tilde{D}(p)^{-1}\to\frac{-i\sum_\mu\gamma_\mu p_\mu}{p^2}.
\end{equation}
This propagator has a pole at $p=0$. Poles in propagators correspond
to physical particles, so in the continuum theory, the propagator
corresponds to a single particle satisfying the Dirac equation. On
the lattice if the fermions are massless, eq.~\eqref{eq:latprop} has
a pole anywhere $p_\mu a=\pi$. In the first Brillouin zone, this
happens 15 places besides $p_\mu=0$. Hence on the lattice at finite
spacing, the propagator has 15 unphysical poles that nevertheless
correspond to some fermions. This is called {\it fermion doubling},
\index{fermion doubling}
and we call the 15 unwanted particles the {\it doublers}.

This proliferation of extra particles can cause problems in the
interacting theory. The additional states can be pair produced
through interactions of the fermion field~\cite{montvay_quantum_1994}.
For instance even if all particles on the external lines of a diagram
are the real particles, the doublers can appear in virtual loops.
Therefore one typically wants to remove the doublers.
One way to remove the doublers is to introduce an extra term that
cancels the doublers on the lattice, and still reduces to the
correct continuum value. With this formulation, the lattice
momentum space Dirac operator is
\begin{equation}
  \tilde{D}(p)=m\id+\frac{i}{a}\sum_\mu\gamma_\mu\sin(p_\mu a)
                   +\frac{1}{a}\sum_\mu\id\big(1-\cos(p_\mu a)\big).
\end{equation} 
This second term is called the {\it Wilson term}. This formulation
\index{Wilson!fermions}
of fermions is called {\it Wilson fermions}. The complete Dirac
operator using this formulation becomes 
\begin{equation}
  D(x|y)^f_{\alpha\beta c_1c_2}
  =\left(m^f+\frac{4}{a}\right)\delta_{\alpha\beta}
                               \delta_{c_1c_2}
                               \delta_{xy}
   -\frac{1}{2a}\sum_{\mu=\pm 1}^{\pm 4}(\id-\gamma_\mu)_{\alpha\beta}
      U_\mu(x)_{c_1 c_2}\delta_{x+a\hat{\mu},y},
\end{equation}
where
\begin{equation}
  \gamma_{-\mu}\equiv-\gamma_{\mu}.
\end{equation}

%\section{Questions}
%\begin{enumerate}
%  \item When can we find a power series expansion for a function of
%        Grassmann numbers?
%  \item Why do we want the integrand to vanish at the boundary?
%  \item Why do we know $\gamma_\mu$ and $U(x)\in\SU(N)$ commute?
%\end{enumerate}

\bibliographystyle{unsrtnat}
\bibliography{bibliography}
