\chapter{Special Topic: Bell's Theorem}\label{ap:bell}

This chapter gives a brief introduction to Bell's theorem and its role in
interpreting quantum mechanics (QM). The presentation in the first two sections
is based mostly off the one given by Griffiths 
\cite{griffiths_introduction_2005}. In the last section, I wanted to focus
on one of the assumptions necessary to derive Bell's inequality because
it's fun to think about philosophically in the context of free will.

Let's recall what happens when you make a measurement in a quantum system.
Generically speaking, you have a system whose state is given by a wave
function. This wave function doesn't tell you exactly what the
outcome of a measurement will be, but rather is related to a statistical
distribution of possible outcomes. Only after you make a measurement do
you know the unique state definitively, and we say the wave function collapses.

But one may wonder whether this distribution merely stands in for some
hidden knowledge that we don't have. For example when one throws a die
in real life, one could, at least in principle, predict the outcome of
the die throw, using classical mechanics, given some initial conditions.
In practice we just say each face is equally likely and leave it at that.
In QM, this viewpoint that there must be some attribute
of the system that tells us what the measurement should be is called
the {\it realist} viewpoint. The viewpoint that the act of measurement
somehow ``creates" this attribute is the {\it orthodox} viewpoint.
Of course one may feel that answering such a question falls outside the
scope of physics entirely; this is the {\it agnostic} viewpoint.

The orthodox view is the textbook interpretation of quantum
measurements. Formally, the observable of interest does not have a
well-defined value before measurement. This has lead to a powerful,
consistent, predictive, successful theory. But it might make one 
uncomfortable\footnote{It is important to emphasize that discomfort
with a theory says nothing about its truth. But it can be enough of
a motivation to at least consider other alternatives.} to
think that the universe is fundamentally truly stochastic, and it
may also seem strange that the act of measuring has such a special
position in the theory. A more modern motivation to re-examine this
viewpoint is that there are many serious, unsolved problems\footnote{In
particular I have in mind that quantum field theories and gravity are
not yet unified.} in physics today,
and one may wonder whether these problems could be treated in a
more complete theory that also explains these ambiguities.

\section{The EPR paradox}
Actually physicists have been uncomfortable with QM for a long time.
In 1935 Einstein, Podolsky, and Rosen
\cite{einstein_can_1935} published the {\it EPR paradox},
which was designed to show that the realist viewpoint must be
the correct one. The following variant of the EPR paradox seems to be
rather popular, and is due to Bohm \cite{bohm_quantum_1951}.
Consider the decay of a neutral pion to an electron and positron:
\begin{equation}
  \pi^0\to e^-+e^+.
\end{equation}
If the pion was at rest, the $e^+$ and $e^-$ will fly off in opposite
directions with the same speed. Since the pion is spin-0, we know
by conservation of angular momentum that the RHS is in the
singlet configuration, i.e.
\begin{equation}
  \ket{\psi}\sim
   \ket{\uparrow\downarrow}-\ket{\downarrow\uparrow},
\end{equation}
where the first and second components of the ket indicate the spins of
the electron and positron, respectively.
Since the system is in the singlet configuration, if an experimenter
measures the spin of the positron, they know the spin of the electron
immediately, and vice-versa, regardless of how separated they are.
To reference this phenomenon, one sometimes speaks of {\it entanglement}.

This is at least superficially surprising from the orthodox perspective.
The reason is that special relativity tells us no influence can travel
faster than light, i.e. our theory must be {\it local}; 
otherwise it would violate causality. On the other
hand, it must be that wave function collapse is instantaneous; otherwise
one would, at least momentarily, violate angular momentum conservation.
Finally, entangled particles are experimentally known to have perfectly
correlated spins. 

Einstein, Podolsky, and Rosen therefore conclude
that the orthodox opinion is untenable. One can then suppose there is
some kind of missing information, or {\it hidden variable}, which
is in these discussions often labeled $\lambda$. The hidden variable
could be a single number, but it could also be a collection
of numbers.

\section{Bell's inequality}
In 1964, Bell proved that, given some very mild, highly intuitive assumptions,
all local hidden variable theories are incompatible with QM
\cite{bell_einstein_1964}. 

\section{Superdeterminism}

\bibliographystyle{unsrtnat}
\bibliography{bibliography}

