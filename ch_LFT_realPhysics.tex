\chapter{LFT: Real physics on the lattice}

We are now in a position to start tackling some real-world physics using the
methods of LFT. Nowadays lattice calculations inform many branches of particle
physics where ab initio, non-perturbative investigations are useful, for example
when computing CKM matrix elements, when exploring the QCD phase diagram at
low-to-moderate temperature near zero quark chemical potential, hadron
spectroscopy, or learning about the muon anomalous magnetic moment, just to name
a few applications.

\section{QCD thermodynamics}  

This chapter focuses on the phenomenology I worked on as a postdoc in
Bielefeld, i.e. the phase diagram of QCD. There are many useful reviews
on the market, for instance Ref.~\cite{ding_thermodynamics_2015}.

In the following we will focus on $N_f=2+1$ QCD using HISQ fermions. We are
interested in nonzero $\mu$, which is not accessible directly by the lattice.
One possible strategy is to expand in small $\muh\equiv\mu/T$. For the
pressure, we find\footnote{Note that this combination on the LHS of 
eq.~\eqref{eq:pxpac} is unitless. This is because area has units
[MeV$^{-2}$] and force has units [MeV$^2$].}
\begin{equation}\label{eq:pxpac}
\frac{P}{T^4}=\sum_{ijk}\frac{1}{i!j!k!}\,\chi_{ijk}^{(u)(d)(s)}(T)\,
               \muh_u^i\,\muh_d^j\,\muh_s^k,
\end{equation}
where we have defined the {\it generalized
susceptibility}\index{generalized susceptibility}
\begin{equation}
\chi_{ijk}^{(u)(d)(s)}(T)
  \equiv\frac{\partial^{\,i+j+k}\,P/T^4}{\partial\muh_u^i\partial\muh_d^j
                                       \partial\muh_s^k}\Big|_{\muh=0}.
\end{equation}
Some example generalized susceptibilities could be
\begin{equation}
  \chi_2^u=\frac{\partial^2}{\partial\muh_u^2}\frac{P}{T^4}
  ~~~~\text{or}~~~~
  \chi_{11}^{us}=\frac{\partial^2}{\partial\muh_u\partial\muh_s}\frac{P}{T^4}.
\end{equation}

Another set of useful relations is between chemical potentials in the flavor
$(u,d,s)$ basis and the conserved charge $(B,Q,S)$ basis\footnote{Yet a further
possible basis of conserved charges is the $(B,I,S)$ basis with isospin $I\equiv I_3$.
For the purpose of distinguishing between $u$ and $d$ quarks only, one can use
either $I$ or $Q$.
To learn more in detail about isospin, see Section~\ref{sec:isohyper}.}, i.e. the basis
of baryon number, electric charge, and strangeness. These relations 
are\footnote{A mnemonic to remember these relations is to ask what each quark
contributes to a system. For example a strange quark makes 1/3 of a baryon,
adds charge -1/3, and subtracts 1 from the strangeness.}
\begin{equation}\begin{aligned}\label{eq:quark-conservedCharge}
  \mu_u &= \frac{1}{3}\mu_B + \frac{2}{3}\mu_Q\\
  \mu_d &= \frac{1}{3}\mu_B - \frac{1}{3}\mu_Q\\
  \mu_s &= \frac{1}{3}\mu_B - \frac{1}{3}\mu_Q - \mu_S.
\end{aligned}\end{equation}
These relations are needed whenever one is interested in {\it conserved
charge fluctuations},\index{conserved charge fluctuation} i.e. susceptibilities like
\begin{equation}
\chi_{ijk}^{(B)(Q)(S)}(T)
  \equiv\frac{\partial^{\,i+j+k}\,P/T^4}{\partial\muh_B^i\partial\muh_Q^j
                                       \partial\muh_S^k}\Big|_{\muh=0}.
\end{equation}
Since none of the conserved charges appears directly in the action, one
instead must decompose a conserved charge derivative into quark derivatives
using eq.~\eqref{eq:quark-conservedCharge}. For example 
one finds
\begin{equation}
  \frac{\partial}{\partial\mu_B}
   =\sum_f\frac{\partial\mu_f}{\partial\mu_B}\frac{\partial}{\partial\mu_f}
   =\frac{1}{3}\sum_f\frac{\partial}{\partial\mu_f}.
\end{equation}

Now we will connect this to the partition function $Z$. We have
\begin{equation}
  \frac{P}{T^4}=\frac{1}{VT^3}\log Z,
\end{equation}
where as usual $V=(aN_s)^3$ and $T=1/aN_\tau$.
When there are $N_f$ flavors and when we use the HISQ action we have,
according to eq.~\eqref{eq:HISQdist},
\begin{equation}
  Z=\int\DD U\prod_f (\det D_f)^{1/4}\,e^{-S_G}
\end{equation}
and compute expectation values of the observable $X$ as
\begin{equation}\label{eq:HISQev}
  \ev{X}=\frac{1}{Z}\int\DD U\prod_f(\det D_f)^{1/4}\,e^{-S_G}X.
\end{equation}

Recall that in the grand canonical ensemble, a particle number
$N$ enters the Boltzmann factor as $\mu N$; so a particle number density 
for a quark flavor $f$ is extracted as\footnote{Remember that there is also
an overall $1/T$ factor in the exponent in natural units.}
\begin{equation}\label{eq:nfdensity}
  \frac{1}{V}\,\partial_{\muh_f}\log Z
  =\frac{1}{VZ}\,\partial_{\muh} Z
  =\ev{n_f}.
\end{equation}
In the staggered formulation, all the dependence on $\muh_f$ will be
hidden in $D_f$, so for $f\neq g$ we get
\begin{equation}
  \partial_{\muh_f}D_g = 0.
\end{equation}

\subsection{Derivative formulas}

We will now derive some formulas which are useful for calculations in QCD
thermodynamics. You can find even more useful formulas for a system of 
$N_f$ identical
fermion flavors in the appendix of Ref.~\cite{allton_thermodynamics_2005}.
For these calculations we will often need the following formula for an
invertible matrix $M$:
\begin{theorem}{The exp-trace-log (ETL) formula}{}\label{thm:exptrlog} 
  $$\det M = \exp \tr \log M$$
\end{theorem}
Let $M$ depend on $\alpha\in\C$ and let $y\in\R$. Then from the ETL formula
\begin{equation}\begin{aligned}
  \partial_\alpha(\det M)^y &= y(\det M)^{y-1}\partial_\alpha\det M\\
      &= y(\det M)^{y-1}\exp\left[\tr\log M\right]\partial_\alpha\tr\log M\\
      &= y(\det M)^y \tr M^{-1}\partial_\alpha M.
\end{aligned}\end{equation}
This formula is useful in evaluating derivatives of roots of the fermion
matrix determinant. A final formula useful for matrix derivatives is
\begin{equation}
  \partial_\alpha M^{-1} = - M^{-1}\left(\partial_\alpha M\right) M^{-1}.
\end{equation}

Our goal is to eventually take derivatives of expectation values, so from
eq.~\eqref{eq:HISQev} we will need $\muh$-derivatives of the partition 
function. For the following few formulas 
I will assume arbitrarily many non-degenerate
fermion flavors\footnote{I find this helpful to get prefactors that appear
in terms in, e.g., $\chi_2^u$ correct.}. Assuming $S_G$ has no 
$\muh$-dependence, we find
\begin{equation}
  \partial_{\muh}Z = \frac{1}{4}\sum_f 
                      Z\ev{\tr D_f^{-1}\partial_{\muh}D_f}.
\end{equation}
Hence 
\begin{equation}\label{eq:dZinv}
  \partial_{\muh}Z^{-1} = -Z^{-2}\partial_{\muh}Z
                        =\frac{1}{4}\sum_fZ\ev{\tr D_f^{-1}\partial_{\muh}D_f}
\end{equation}
and
\begin{equation}\label{eq:dlogZ}
  \partial_{\muh}\log Z=\frac{1}{4}\sum_f\ev{\tr D_f^{-1}\partial_{\muh}D_f}.
\end{equation}
Comparing eq.~\eqref{eq:dlogZ} with eq.~\eqref{eq:nfdensity} we find in
the staggered formulation
\begin{equation}
  n_f=\frac{1}{4V}\tr D_f^{-1}\partial_{\muh_f}D_f.
\end{equation} 
Finally using eq.~\eqref{eq:dZinv} one quickly derives the following useful formula
for an observable $O$ calculated on HISQ configurations:
\begin{proposition}{}{}\label{prop:dO}
\begin{equation*} \partial_{\muh}\ev{O}
       =   \ev{\partial_{\muh}O}
        -\frac{1}{4}\sum_f\ev{\tr D_f^{-1}\partial_{\muh}D_f}\ev{O}
         +\frac{1}{4}\sum_f\ev{O\tr D_f^{-1}\partial_{\muh}D_f}
\end{equation*}
\end{proposition} 
Note that if $\muh=\muh_f$ with $N_f$ degenerate flavors, 
Proposition~\ref{prop:dO} reduces to eq. (A3)
in Ref.~\cite{allton_thermodynamics_2005}. Using this Proposition along
with eq.~\eqref{eq:dlogZ}, one 
can start expressing generalized susceptibilities in terms of the fermion
matrix. For example one finds for any flavor $f$
\begin{equation}\begin{aligned}
  VT^3\,\chi_2^f =\partial_{\muh_f}^{\,2}\log Z
                 =\, & \frac{1}{16}\ev{\left(\tr D_f^{-1}D_f'\right)^2}
                    -\frac{1}{16}\ev{\tr D_f^{-1}D_f'}^2\\
                   & -\frac{1}{4}\ev{\tr \left(D_f^{-1} D_f'\right)^2}
                    +\frac{1}{4}\ev{\tr D_f^{-1}D_f''},
\end{aligned}\end{equation}
where a prime on $D_f$ indicates a derivative w.r.t. the corresponding chemical
potential $\muh_f$. Similarly for two flavors $f\neq g$ 
\begin{equation}\begin{aligned}
  VT^3\,\chi_{11}^{fg} =\partial_{\muh_f}\partial_{\muh_g}\log Z
                 = \,& \frac{1}{16}\ev{\tr D_f^{-1}D_f'\tr D_g^{-1}D_g'} \\
                   & -\frac{1}{16}\ev{\tr D_f^{-1}D_f'}\ev{\tr D_g^{-1}D_g'}.
\end{aligned}\end{equation}
Using an $N_f=2+1$ HISQ action with $m_l\equiv m_u=m_d$ along with 
eq.~\eqref{eq:quark-conservedCharge}, one can figure out how to 
express\footnote{Note that when carrying out calculations like these, to avoid making mistakes,
one should always work with non-degenerate flavors, then allow the flavors to be
degenerate at the end. This way one correctly distinguishes between e.g.
$\chi_{11}^{ll}$ and $\chi_2^l$.} conserved charge fluctuations in terms of 
generalized susceptibilities, such as
\begin{equation}
  \chi_2^B=\frac{1}{9}\left(2\chi_2^l+\chi_2^s+2\chi_{11}^{ll}+4\chi_{11}^{ls}\right)
\end{equation}
and
\begin{equation}
  \chi_2^Q=\frac{1}{9}\left(5\chi_2^l+\chi_2^s-4\chi_{11}^{ll}-2\chi_{11}^{ls}\right).
\end{equation}


%\subsection{Practical facts about thermodynamic quantities}

\bibliographystyle{unsrtnat}
\bibliography{bibliography}
