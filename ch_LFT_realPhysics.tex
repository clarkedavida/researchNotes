\chapter{LFT: Real physics on the lattice}

We are now in a position to start tackling some real-world physics using the
methods of LFT. Nowadays lattice calculations inform many branches of particle
physics where ab initio, non-perturbative investigations are useful, for example
when computing CKM matrix elements, when exploring the QCD phase diagram at
low-to-moderate temperature near zero quark chemical potential, hadron
spectroscopy, or learning about the muon anomalous magnetic moment, just to name
a few applications.

\section{QCD thermodynamics}  

This chapter focuses on the phenomenology I worked on as a postdoc in
Bielefeld, i.e. the phase diagram of QCD. There are many useful reviews
on the market, for instance Ref.~\cite{ding_thermodynamics_2015}.

In the following we will focus on $N_f=2+1$ QCD using HISQ fermions. We are
interested in nonzero $\mu$, which is not accessibly directly by the lattice.
One possible strategy is to expand in small $\muh\equiv\mu/T$. For the
pressure, we find\footnote{Note that this combination on the LHS of 
eq.~\eqref{eq:pxpac} is unitless. This is because area has units
[MeV$^{-2}$] and force has units [MeV$^2$].}
\begin{equation}\label{eq:pxpac}
\frac{P}{T^4}=\sum_{ijk}\frac{1}{i!j!k!}\,\chi_{ijk}^{(u)(d)(s)}(T)\,
               \muh_u^i\,\muh_d^j\,\muh_s^k,
\end{equation}
where we have defined the {\it generalized susceptibility}
\begin{equation}
\chi_{ijk}^{(u)(d)(s)}(T)
  \equiv\frac{\partial^{\,i+j+k}\,P/T^4}{\partial\muh_u^i\partial\muh_d^j
                                       \partial\muh_s^k}\Big|_{\muh=0}.
\end{equation}
Some example generalized susceptibilities could be
\begin{equation}
  \chi_2^u=\frac{\partial^2}{\partial\muh_u^2}\frac{P}{T^4}
  ~~~~\text{or}~~~~
  \chi_{11}^{us}=\frac{\partial^2}{\partial\muh_u\partial\muh_s}\frac{P}{T^4}.
\end{equation}

Another set of useful relations is between chemical potentials in the flavor
$(u,d,s)$ basis and the conserved charge $(B,Q,S)$ basis, i.e. the basis
of baryon number, electric charge, and strangeness. These relations 
are\footnote{A mnemonic to remember these relations is to ask what each quark
contributes to a system. For example a strange quark makes 1/3 of a baryon,
adds charge -1/3, and subtracts 1 from the strangeness.}
\begin{equation}\begin{aligned}\label{eq:quark-conservedCharge}
  \mu_u &= \frac{1}{3}\mu_B + \frac{2}{3}\mu_Q\\
  \mu_d &= \frac{1}{3}\mu_B - \frac{1}{3}\mu_Q\\
  \mu_s &= \frac{1}{3}\mu_B - \frac{1}{3}\mu_Q - \mu_S.
\end{aligned}\end{equation}
These relations are needed whenever one is interested in {\it conserved
charge fluctuations}, i.e. susceptibilities like
\begin{equation}
\chi_{ijk}^{(B)(Q)(S)}(T)
  \equiv\frac{\partial^{\,i+j+k}\,P/T^4}{\partial\muh_B^i\partial\muh_Q^j
                                       \partial\muh_S^k}\Big|_{\muh=0}.
\end{equation}
Since none of the conserved charges appears directly in the action, one
instead must decompose a conserved charge derivative into quark derivatives
using eq.~\eqref{eq:quark-conservedCharge}. For example for any quark
flavor $f$ one finds
\begin{equation}
  \frac{\partial}{\partial\mu_B}
   =\frac{\partial\mu_f}{\partial\mu_B}\frac{\partial}{\partial\mu_f}
   =\frac{1}{3}\frac{\partial}{\partial\mu_f}.
\end{equation}

Now we will connect this to the partition function $Z$. We have
\begin{equation}
  \frac{P}{T^4}=\frac{1}{VT^3}\log Z,
\end{equation}
where as usual $V=(aN_s)^3$ and $T=1/aN_\tau$.
Since we are working with $N_f=2+1$ and the HISQ action we have,
according to eq.~\eqref{eq:HISQdist},
\begin{equation}
  Z=\int\DD U(\det D_l)^{1/2}\,(\det D_s)^{1/4}\,e^{S_G}
\end{equation}
and compute expectation values of the observable $X$ as
\begin{equation}\label{eq:HISQev}
  \ev{X}=\frac{1}{Z}\int\DD U(\det D_l)^{1/2}\,(\det D_s)^{1/4}\,e^{S_G}X.
\end{equation}

Recall that in the grand canonical ensemble, a particle number
$N$ enters the Boltzmann factor as $\mu N$; so a particle number density 
is extracted as\footnote{Remember that there is also
an overall $1/T$ factor in the exponent in natural units.}
\begin{equation}
  n = \frac{1}{V}\,\partial_{\muh}\log Z
\end{equation}

\subsection{Derivative formulas}

We will now derive some formulas which are useful for calculations in QCD
thermodynamics. You can find even more useful formulas for a system of 
$N_f$ identical
fermion flavors in the appendix of Ref.~\cite{allton_thermodynamics_2005}.
For these calculations we will often need the following formula for a
matrix $A$:
\begin{theorem}{The exp-trace-log (ETL) formula}{}\label{thm:exptrlog} 
  $$\det A = \exp \tr \log A$$
\end{theorem}
\begin{corollary}{}{} Let $M$ be any matrix with $\alpha$ a parameter that 
$M$ depends on and $y\in\R$. Then
$$
  \partial_\alpha(\det M)^y = y(\det M)^y \tr M^{-1}\partial_\alpha M
$$
\begin{proof}
\begin{equation*}\begin{aligned}
  \partial_\alpha(\det M)^y &= y(\det M)^{y-1}&&\partial_\alpha\det M\\
      &=  &&\exp\left[\tr\log M\right]\partial_\alpha\tr\log M
\end{aligned}\end{equation*}
Then just apply the ETL formula. 
\end{proof}
\end{corollary}

Our goal is to eventually take derivatives of expectation values, so from
eq.~\eqref{eq:HISQev} we will need $\muh$-derivatives of the partition 
function. Assuming $S_G$ has no $\muh$-dependence, we find
\begin{equation}
  \partial_{\muh}Z =   \frac{1}{2}\,Z\ev{\tr M_l^{-1}\partial_{\muh}M_l}
                     + \frac{1}{4}\,Z\ev{\tr M_s^{-1}\partial_{\muh}M_s},
\end{equation}
and hence 
\begin{equation}
  \partial_{\muh}Z^{-1} = -Z^{-2}\partial_{\muh}Z
                        =   \frac{1}{2}\,Z\ev{\tr M_l^{-1}\partial_{\muh}M_l}
                          + \frac{1}{4}\,Z\ev{\tr M_s^{-1}\partial_{\muh}M_s}.
\end{equation}

\subsection{Practical facts about thermodynamic quantities}

\bibliographystyle{unsrtnat}
\bibliography{bibliography}
