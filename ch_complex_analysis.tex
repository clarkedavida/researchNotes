\chapter{Math: Complex Analysis} 

Here we review some basics of complex analysis that often appear in physics
calculation. Some useful resources for this include
Ref.~\cite{brown_complex_2003,weber_essentials_2012}.


\section{Preliminaries}

We are interested in functions $f:\C\to\C$. Derivatives are defined similarly as
with real, $1D$ functions. The limit
\begin{equation}\label{eq:cderiv}
  \dv{f}{z}\equiv\lim_{w\to z}\frac{f(w)-f(z)}{w-z}
\end{equation}
can be approached in infinitely many ways in $\C$. We call \equatref{eq:cderiv} the
the derivative of $f$ when this limit exists and is independent of the chosen
path\footnote{This analogous to how the derivative can only be well defined in
$\R$ if the limit is the same whether one approaches from the left or right.}.
The function is said to be {\it holomorphic}. If it can be expanded as a
convergent Taylor series, it is {\it analytic}.
\index{analytic}\index{holomorphic}

Following the spirit of the above discussion, many of the facts we will learn
about complex analysis can be inherited from facts about real analysis. With
that in mind, for the rest of the chapter we will introduce for the complex
function $f$ the notation
\begin{equation}
  f(z)=f(x+iy)=u(x,y)+iv(x,y),
\end{equation}
where $x,y\in\R$ and $u,v:\R\to\R$. We start with a fact about when complex
functions are holomorphic.
\index{Cauchy-Riemann equations}
\begin{theorem}{Cauchy-Riemann (CR) equations}{}
$f$ is holomorphic if and only if
$$
  \pdv{u}{x}=\pdv{v}{y},~~~~~~~~\pdv{u}{y}=-\pdv{v}{x}.
$$
\begin{proof}
If the derivative exists, then by definition
it exists along a pure imaginary, vertical path approaching $z$, as well
as a pure real, horizontal path approaching $z$.
Then using \equatref{eq:cderiv} one obtains
$$
\pdv{u}{x}+i\pdv{v}{x}=\dv{f}{z}=-i\pdv{u}{y}+\pdv{v}{y},
$$
from which the result immediately follows.
\end{proof}
\end{theorem}

Since the definition in $\C$ is analogous to that in $\R$, the derivative obeys
the product and chain rules. The CR equations are a convenient way
to determine whether a complex function is holomorphic. For instance you can
use them to show $|z|^2$ is not holomorphic. 
It is also useful to note that $\exp(z)$ is holomorphic in $\C$.
It also follows from the CR equations that both $u$ and $v$ satisfy
\index{Laplace's equation}
Laplace's equation, i.e.
\begin{equation}
\left(\partial_x^2+ \partial_y^2\right)u = \left(\partial_x^2
+\partial_y^2\right))v = 0.
\end{equation}
Hence if you specify $f$ on the boundary of some region, this fixes the solution
in the region uniquely.

Before moving on to integration, it is useful to discuss {\it isolated singularities} of
\index{singularity!isolated}
complex functions, i.e. isolated\footnote{``Isolated" in this context means that if
$z_0$ is a singularity, there exists a neighborhood of $z_0$ free of any other
singularities. This can be contrasted with e.g. $f(z)=z^{1/2}$, which is
singular along the entire negative real axis.} 
points where these functions are not well defined
or well behaved. As we will see later, these singularities can limit Taylor
expansions, and they contribute to complex integrals. 
Let the region $B\subset\C$ be open, let $z_0\in B$ be a singularity of $f$, 
and let $f$ be holomorphic in $B-z_0$. Complex singularities
can be classified as follows:
\begin{itemize}
  \index{singularity!removable}
  \item $z_0$ is a {\it removable singularity} of $f$ if there exists a
        holomorphic function $g$ defined on $B$ such that $f(z) = g(z)$ for all
        $z\in B-z_0$.
  \index{singularity!pole}
  \item $z_0$ is a {\it pole} of $f$ if there exists a holomorphic function $g$
        defined on $B$ with $g(z_0)\neq0$ and an $n\in\N$ such that 
        $f(z) = g(z) / (z-z_0)^n$ for all $z\in B-z_0$. The smallest such $n$ 
        is called the {\it order} of the pole. A {\it meromorphic} function
        \index{meromorphic} is analytic everywhere in $\C$ except at a finite
        number of poles.
  \index{singularity!essential}
  \item $z_0$ is an {\it essential singularity} if it is neither removable nor a
        pole.
\end{itemize}

Complex integrals are in general path-dependent.
This integral can be defined using a Riemann sum, but in practice one usually
parameterizes the curve and computes the integral that way.
It follows that the integral is linear, and that the sign of an integral
flips when the orientation of the contour changes.
The following Proposition establishes an integral that appears frequently
in complex analysis. Note that the integrand has a pole at $z_0$.
\begin{proposition}{}{circleIntegral}
Let $C_R(z_0)$ denote the circle of radius $R$ centered at $z_0\in\C$
and let $n\in\Z$. Then
$$
\oint_{C_R(z_0)}\dd{z}(z-z_0)^n=
\begin{cases}
 2\pi i & n=-1\\
 0      & \text{otherwise}.
\end{cases}
$$
\begin{proof}
We can parameterize the contour with $z(\theta)=z_0+Re^{i\theta}$
with $\theta\in[0,2\pi)$. The integral evaluates to
\begin{equation*}\begin{aligned}
\oint_{C_R(z_0)}\dd{z}(z-z_0)^n&=
i\int_0^{2\pi}\dd{\theta} R^{n+1}e^{i(n+1)\theta}\\
&=
\begin{cases}
 2\pi i & n=-1\\
 \frac{R^{n+1}}{n+1}e^{i(n+1)\theta}\,\big|_0^{2\pi} & \text{otherwise}. 
\end{cases}
\end{aligned}\end{equation*}
\end{proof}
\end{proposition}

Later this chapter we will construct paths that have circles around singular
points, and \propref{prp:circleIntegral} is extremely useful in those contexts.
For example it will later be used to prove the residue \index{residue theorem}
theorem, and we can see that the $2\pi i$ factor that appears there can be traced
back to parameterizing this closed path with an angle.  

We now move on to some more general facts about complex integrals.
A region $B\subset\C$ is {\it path connected} if any two points in the region
can be joined by a path. A {\it simply connected} region is a path connected
\index{path connected}\index{simply connected}
region and any loop path can be contracted to a point\footnote{In other words,
there are no holes.}. It turns out that a wide class of complex integrals
over such regions evaluate to zero.

\index{Cauchy!integral theorem}
\begin{theorem}{Cauchy integral theorem}{}
Let $B\subset\C$ be a bounded and simply connected region. Let $f:B\to\C$ be
holomorphic in $B$ and $C$ be a closed curve fully contained in $B$. Then
$$
  \oint_C\dd{z}f(z)=0.
$$
\begin{proof}
We have
\begin{equation*}\begin{aligned}
\oint_C\dd{z}f(z)=&\oint_C\left(\dd{x}u(x,y)-\dd{y}v(x,y)\right)\\
                 &+i\oint_C\left(\dd{y}u(x,y)+\dd{x}v(x,y)\right).
\end{aligned}\end{equation*}
Let $A$ be the area enclosed by $C$. By Stokes' theorem, we get
\begin{equation*}
\oint_C\dd{z}f(z)=\iint_A\dd{x}\dd{y}\left(-\pdv{v}{x}-\pdv{u}{y}\right)
                   +i\iint_A\dd{x}\dd{y}\left(\pdv{u}{x}-\pdv{v}{y}\right).
\end{equation*}
Since $f$ is holomorphic, the integrands vanish by the CR equations.
\end{proof}
\end{theorem}

\index{Cauchy!integral formula}
\begin{theorem}{Cauchy integral formula}{}
Let $B\subset\C$ be bounded and simply connected region. Let $f:B\to\C$ be
holomorphic in $B$ and continuous on the boundary $\partial B$.
Then $\Forall n\in\N$ we get
$$
  \oint_{\partial B}\dd{z}\frac{f(z)}{(z-z_0)^{n+1}}=
\begin{cases}
\frac{2\pi i}{n!}f^{(n)}(z_0) & z_0\in B \\
 0            & \text{otherwise}.
\end{cases}
$$
\begin{proof} When $z_0\notin B$, we get 0 by the Cauchy integral theorem.
Therefore we consider for the remainder of the proof $z_0\in B$.
We start with the $n=0$ case, then derive general $n$ from that.
Let $z-z_0=\epsilon e^{i\phi}$.
\begin{equation*}\begin{aligned}
  \oint_{\partial B}\dd{z}\frac{f(z)}{z-z_0}
&=-\lim_{\epsilon\to 0}\int_{2\pi}^0\dd{\phi}
     i\epsilon e^{i\phi}\frac{f(z_0+\epsilon e^{i\phi})}{\epsilon e^{i\phi}}\\
&=i\lim_{\epsilon\to 0}\int_{0}^{2\pi}\dd{\phi}f(z_0+\epsilon e^{i\phi})\\
&=i\int_{0}^{2\pi}\dd{\phi}f(z_0)\\
&=2\pi if(z_0).\\
\end{aligned}\end{equation*}
From here, all one has to do is take partial derivatives of this result 
w.r.t. $z_0$.
\end{proof} 
\end{theorem}

\section{The residue theorem}\index{residue theorem}

Let $f$ be holomorphic in the annulus ${z\in\C : 0\leq r<|z-z_0|<R}$
where $r,R>0$.
Define $\Forall\rho : r<\rho<R$ and $\Forall k\in\Z$
\begin{equation}
  a_k\equiv\oint\dd{z}\frac{f(z)}{(z-z_0)^{k+1}}.
\end{equation}
Then the {\it Laurent expansion}\index{Laurent expansion} of $f$ is
\begin{equation}
  f(z)=\sum_{k=-\infty}^\infty a_k(z-z_0)^k.
\end{equation}
Note that if $f$ permits a Taylor expansion, this Taylor series is the Laurent
series, with $a_k=0$ whenever $k<0$.



\section{Pad\'e approximants}\index{Pad\'e approximant}

To define the Pad\'e approximant, we introduce\footnote{One can always force
the first coefficient in the denominator to be 1 by factoring it out of all
the other coefficients in both the numerator and denominator.} first the {\it rational
function of order} $[m,n]$, $m,n\in\N$ as \index{function!rational}
\begin{equation}
  R_n^m(x)\equiv\frac{\sum_{i=0}^m a_ix^i}{1+\sum_{j=1}^nb_jx^j}.
\end{equation}


\bibliographystyle{unsrtnat}
\bibliography{bibliography}
