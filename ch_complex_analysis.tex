\chapter{Math: Complex Analysis} 

Here we review some basics of complex analysis that often appear in physics
calculation. Some useful resources for this include
Ref.~\cite{brown_complex_2003,weber_essentials_2012}.

\section{Preliminaries}\label{sec:complexPrelim}

We are interested in functions $f:\C\to\C$. Derivatives are defined similarly as
with real, $1D$ functions. The limit
\begin{equation}\label{eq:cderiv}
  \dv{f}{z}\equiv\lim_{w\to z}\frac{f(w)-f(z)}{w-z}
\end{equation}
can be approached in infinitely many ways in $\C$. We call \equatref{eq:cderiv} the
the derivative of $f$ when this limit exists and is independent of the chosen
path\footnote{This analogous to how the derivative can only be well defined in
$\R$ if the limit is the same whether one approaches from the left or right.}.
The function is said to be {\it holomorphic}. If it can be expanded as a
convergent Taylor series, it is {\it analytic}.
\index{analytic}\index{holomorphic}

Following the spirit of the above discussion, many of the facts we will learn
about complex analysis can be inherited from facts about real analysis. With
that in mind, for the rest of the chapter we will introduce for the complex
function $f$ the notation
\begin{equation}
  f(z)=f(x+iy)=u(x,y)+iv(x,y),
\end{equation}
where $x,y\in\R$ and $u,v:\R\to\R$. We start with a fact about when complex
functions are holomorphic.
\index{Cauchy-Riemann equations}
\begin{theorem}{Cauchy-Riemann (CR) equations}{}
$f$ is holomorphic if and only if
$$
  \pdv{u}{x}=\pdv{v}{y},~~~~~~~~\pdv{u}{y}=-\pdv{v}{x}.
$$
\begin{proof}
If the derivative exists, then by definition
it exists along a pure imaginary, vertical path approaching $z$, as well
as a pure real, horizontal path approaching $z$.
Then using \equatref{eq:cderiv} one obtains
$$
\pdv{u}{x}+i\pdv{v}{x}=\dv{f}{z}=-i\pdv{u}{y}+\pdv{v}{y},
$$
from which the result immediately follows.
\end{proof}
\end{theorem}

Since the definition in $\C$ is analogous to that in $\R$, the derivative obeys
the product and chain rules. The CR equations are a convenient way
to determine whether a complex function is holomorphic. For instance you can
use them to show $|z|^2$ is not holomorphic. 
It is also useful to note that $\exp(z)$ is holomorphic in $\C$.
It also follows from the CR equations that both $u$ and $v$ satisfy
\index{Laplace's equation}
Laplace's equation, i.e.
\begin{equation}
\left(\partial_x^2+ \partial_y^2\right)u = \left(\partial_x^2
+\partial_y^2\right)v = 0.
\end{equation}
Hence if you specify $f$ on the boundary of some region, this fixes the solution
in the region uniquely.

Before moving on to integration, it is useful to discuss {\it isolated singularities} of
\index{singularity!isolated}
complex functions, i.e. isolated\footnote{``Isolated" in this context means that if
$z_0$ is a singularity, there exists a neighborhood of $z_0$ free of any other
singularities. This can be contrasted with e.g. $f(z)=z^{1/2}$, which is
singular along the entire negative real axis.} 
points where these functions are not well defined
or well behaved. As we will see later, these singularities can limit Taylor
expansions, and they contribute to complex integrals. 
Let the region $B\subset\C$ be open, let $z_0\in B$ be a singularity of $f$, 
and let $f$ be holomorphic in $B-z_0$. Complex singularities
can be classified as follows:
\begin{itemize}
  \index{singularity!removable}
  \item $z_0$ is a {\it removable singularity} of $f$ if there exists a
        holomorphic function $g$ defined on $B$ such that $f(z) = g(z)$ for all
        $z\in B-z_0$.  
  \index{singularity!pole}
  \item $z_0$ is a {\it pole} of $f$ if there exists a holomorphic function $g$
        defined on $B$ with $g(z_0)\neq0$ and an $n\in\N$ such that 
        $f(z) = g(z) / (z-z_0)^n$ for all $z\in B-z_0$. The smallest such $n$ 
        is called the {\it order} of the pole. A {\it meromorphic} function
        \index{meromorphic} is analytic everywhere in $\C$ except at a finite
        number of poles.
  \index{singularity!essential}
  \item $z_0$ is an {\it essential singularity} if it is neither removable nor a
        pole.
\end{itemize}

Complex integrals are in general path-dependent.
This integral can be defined using a Riemann sum, but in practice one usually
parameterizes the curve and computes the integral that way.
It follows that the integral is linear, and that the sign of an integral
flips when the orientation\footnote{We use the convention
that curves give positive contributions if they are oriented CCW, while a
negative contribution when oriented CW.} of the contour changes.
The following Proposition establishes an integral that appears frequently
in complex analysis. Note that the integrand has a pole at $z_0$.
\begin{proposition}{}{circleIntegral}
Let $C_R(z_0)$ denote the circle of radius $R$ centered at $z_0\in\C$
and let $n\in\Z$. Then
$$
\oint_{C_R(z_0)}\dd{z}(z-z_0)^n=
\begin{cases}
 2\pi i & n=-1\\
 0      & \text{otherwise}.
\end{cases}
$$
\begin{proof}
We can parameterize the contour with $z(\theta)=z_0+Re^{i\theta}$
with $\theta\in[0,2\pi)$. The integral evaluates to
\begin{equation*}\begin{aligned}
\oint_{C_R(z_0)}\dd{z}(z-z_0)^n&=
i\int_0^{2\pi}\dd{\theta} R^{n+1}e^{i(n+1)\theta}\\
&=
\begin{cases}
 2\pi i & n=-1\\
 \frac{R^{n+1}}{n+1}e^{i(n+1)\theta}\,\big|_0^{2\pi} & \text{otherwise}. 
\end{cases}
\end{aligned}\end{equation*}
\end{proof}
\end{proposition}

Later this chapter we will construct paths that have circles around singular
points, and \propref{prp:circleIntegral} is extremely useful in those contexts.
For example it will later be used to prove the residue \index{residue theorem}
theorem, and we can see that the $2\pi i$ factor that appears there can be traced
back to parameterizing this closed path with an angle.  

We now move on to some more general facts about complex integrals.
A region $B\subset\C$ is {\it path connected} if any two points in the region
can be joined by a path. A {\it simply connected} region is a path connected
\index{path connected}\index{simply connected}
region and any loop path can be contracted to a point\footnote{In other words,
there are no holes.}. It turns out that a wide class of complex integrals
over such regions evaluate to zero.

\index{Cauchy!integral theorem}
\begin{theorem}{Cauchy integral theorem}{}
Let $B\subset\C$ be a bounded and simply connected region. Let $f:B\to\C$ be
holomorphic in $B$ and $C$ be a closed curve fully contained in $B$. Then
$$
  \oint_C\dd{z}f(z)=0.
$$
\begin{proof}
We have
\begin{equation*}\begin{aligned}
\oint_C\dd{z}f(z)=&\oint_C\left(\dd{x}u(x,y)-\dd{y}v(x,y)\right)\\
                 &+i\oint_C\left(\dd{y}u(x,y)+\dd{x}v(x,y)\right).
\end{aligned}\end{equation*}
Let $A$ be the area enclosed by $C$. By Stokes' theorem, we get
\begin{equation*}
\oint_C\dd{z}f(z)=\iint_A\dd{x}\dd{y}\left(-\pdv{v}{x}-\pdv{u}{y}\right)
                   +i\iint_A\dd{x}\dd{y}\left(\pdv{u}{x}-\pdv{v}{y}\right).
\end{equation*}
Since $f$ is holomorphic, the integrands vanish by the CR equations.
\end{proof}
\end{theorem}

It's good that we now know how to integrate analytic functions in $\C$. We will
next try integrating functions with a pole singularity. We consider a function
of the form
\begin{equation}\label{eq:simplePole}
\frac{f(z)}{z-z_0}
\end{equation}
defined in some bounded, simply connected region $B\subset\C$ with boundary
$\partial B$, where it is continuous. Here $f$ should be holomorphic in $B$.
The strategy, which is a
common one when calculating integrals in complex analysis, is to define a new
contour $C'$ that excises the pole $z_0$, like shown in \figref{fig:cauchy}.
The function~\eqref{eq:simplePole} is then holomorphic in the region
enclosed by $C'$, and hence it vanishes along $C'$ by the Cauchy integral
theorem.
What remains is to figure out what the difference is between $C'$ and 
$\partial B$.

\begin{figure}
\includegraphics[width=\linewidth]{figs/cauchy_integral-cropped.pdf}
\caption{Contour $C'$ that excises the pole $z_0$ of the
function~\eqref{eq:simplePole}. Following the notation of
\propref{prp:circleIntegral}, $\overline{C}_{\epsilon}(z_0)$ is a circle of
radius $\epsilon$ centered on $z_0$, with the bar indicating that it has a CW
orientation.}
\label{fig:cauchy}
\end{figure}

We choose our excision to have arbitrarily small radius $\epsilon$.
In the limit $\epsilon\to0$, we have
\begin{equation}
  C'=\partial B + L_1 + \overline{C}_{\epsilon}(z_0) + L_2.
\end{equation}
In this limit, $L_1$ and $L_2$ will cancel since they have opposite
orientations.
We can now parameterize $z$ on $\overline{C}_{\epsilon}(z_0)$ 
by $z-z_0=\epsilon e^{i\phi}$, and putting everything together, we find
\begin{equation}\begin{aligned}\label{eq:cauchyn0}
  \oint_{\partial B}\dd{z}\frac{f(z)}{z-z_0}
&=-\lim_{\epsilon\to 0}\int_{2\pi}^0\dd{\phi}
     i\epsilon e^{i\phi}\frac{f(z_0+\epsilon e^{i\phi})}{\epsilon e^{i\phi}}\\
&=i\lim_{\epsilon\to 0}\int_{0}^{2\pi}\dd{\phi}f(z_0+\epsilon e^{i\phi})\\
&=i\int_{0}^{2\pi}\dd{\phi}f(z_0)\\
&=2\pi if(z_0).
\end{aligned}\end{equation}

\index{Cauchy!integral formula}
\begin{theorem}{Cauchy integral formula}{}
Let $B\subset\C$ be bounded and simply connected region. Let $f:B\to\C$ be
holomorphic in $B$ and continuous on the boundary $\partial B$.
Then $\Forall n\in\N$ we get
$$
  \oint_{\partial B}\dd{z}\frac{f(z)}{(z-z_0)^{n+1}}=
\begin{cases}
\frac{2\pi i}{n!}f^{(n)}(z_0) & z_0\in B \\
 0            & \text{otherwise}.
\end{cases}
$$
\begin{proof} When $z_0\notin B$, we get 0 by the Cauchy integral theorem.
Therefore we consider for the remainder of the proof $z_0\in B$.
The $n=0$ case is given by \equatref{eq:cauchyn0}. 
From here, all one has to do is take partial derivatives of this result 
w.r.t. $z_0$.
\end{proof} 
\end{theorem}

\section{The residue theorem}\index{residue!theorem}\label{sec:residueThm}

Let $f$ be holomorphic in the annulus ${z\in\C : 0\leq r<|z-z_0|<R}$
where $r,R>0$.
Define $\Forall\rho : r<\rho<R$ and $\Forall k\in\Z$
\begin{equation}\label{eq:laurentCoeff}
  a_k\equiv\frac{1}{2\pi i}\oint_{C_{\rho}(z_0)}\dd{z}\frac{f(z)}{(z-z_0)^{k+1}}.
\end{equation}
Then the {\it Laurent expansion}\index{Laurent expansion} of $f$ is
\begin{equation}\label{eq:laurentExpansion}
  f(z)=\sum_{k=-\infty}^\infty a_k(z-z_0)^k.
\end{equation}
The $a_k$ are unique and independent of $\rho$, and the Laurent expansion
defines $f$ in the annulus.
If $f$ permits a Taylor expansion, this Taylor series is the Laurent
series, with $a_k=0$ whenever $k<0$.

To get some intuition for why this is true, we can at least show that the
definition~\eqref{eq:laurentCoeff} is consistent.
Plugging \equatref{eq:laurentExpansion} into \equatref{eq:laurentCoeff}
and applying the Cauchy integral formula, we find
\begin{equation}
  a_k\equiv\frac{1}{2\pi i}\sum_{l=-\infty}^\infty\oint_{C_{\rho}(z_0)}
  \dd{z}\frac{a_l}{(z-z_0)^{k-l+1}}
     = \sum_{l=-\infty}^\infty a_l\delta_{k-l,0}=a_k.
\end{equation}

Laurent series can also be used to classify the isolated singularities
of \secref{sec:complexPrelim}. In particular $f$ has at $z_0$
\begin{itemize}
  \item a removable singularity if $a_k=0$ $\Forall k<0$;
  \item a pole of order $m$ if the first, non-vanishing coefficient
        is $a_{-m}$; and
  \item an essential singularity if there is no first, non-vanishing
        coefficient $a_{-m}$.
\end{itemize}

Suppose $f$ has a Laurent expansion in an annulus about $z_0$.
Then the\index{residue} {\it residue} of $f$ at $z_0$ is
\begin{equation}
 \Res f(z_0)=a_{-1}=\frac{1}{2\pi i}\oint_{C_{\rho}(z_0)}\dd{z}f(z).
\end{equation}

\begin{figure}
\includegraphics[width=\linewidth]{figs/residue_theorem-cropped.pdf}
\caption{Contour $C'$ that excises all isolated singularities of
the residue theorem. Each singularity is excised by the circle
$\overline{C}_{\rho_k}(z_k)$.}
\label{fig:residue}
\end{figure}

\begin{theorem}{Residue theorem}{residueThm}
Let $B\subset\C$ be an area bounded by a finite number of piecewise
continuous curves. Let $f$ be holomorphic on $\partial B$ 
holomorphic in $B$, except for a finite number of isolated 
singularities $z_1$, ..., $z_n\in B$. Then
$$
\oint_{\partial B}\dd{z}f(z)=2\pi i\sum_{k=1}^n\Res f(z_k).
$$
\begin{proof}
We construct the contour $C'$ indicated in \figref{fig:residue}
created by excising the singularities. Each singularity is
excised with contour $\overline{C}_{\rho_k}(z_k)$, with the
radius $\rho_k$ chosen such that the Laurent expansion
of $f$ converges there. To complete the proof, we use the Laurent
expansion of $f$ around each singularity.
\begin{equation*}
  \oint\dd{z}f(z)&=\sum_{k=1}^n\sum_{l=-\infty}^\infty 
                    a_l^{(k)}\oint_{C_{\rho_k}(z_k)}(z-z_k)^l.
\end{equation*}
According to \propref{prp:circleIntegral}, each integral will contribute 
only if $l=-1$; there it contributes $2\pi i$. That the residue is
$a_{-1}$ by definition completes the proof.
\end{proof}
\end{theorem}

\begin{figure}
\centering
\includegraphics[width=0.75\linewidth]{figs/principal_value-cropped.pdf}
\caption{Example contour used to assign a principal value to a real integral.
Here the semicircle $c_2$ has radius $\epsilon$ and excises the singularity
$x_0$ on the real axis. The semicircle $c_4$ has radius $R$, and the function 
$f$ is asked to decay sufficiently quickly so that the contribution on 
$c_4$ will vanish as $R\to\infty$.}
\label{fig:principalValue}
\end{figure}

The residue theorem is extremely powerful for the evaluation of many integrals,
including real-valued ones. In particular it can be used to assign a number to
a sum of separately divergent, real-valued integrals. 
We begin with an example 
that does not require the residue theorem:
\begin{equation}\label{eq:principalEasy}
  \principal\int_{-a}^a\dd{x}\frac{1}{x}=0
\end{equation} 
with $a>0$, which you can convince yourself of since the integrand is odd. On
the other hand, if one were to split this into a contribution over negative and
positive numbers, each contribution would diverge.

In the above we have introduced the {\it principal value}\index{principal value}
symbol $\principal$. More formally let $f$ be defined in the interval $[a,b]$
with the possible exception $c$, $a<c<b$. Then 
\begin{equation}
  \principal\int_a^b\dd{x}f(x)\equiv\lim_{\epsilon\to0}
     \left(\int_a^{c-\epsilon}\dd{x}f(x)+\int_{c+\epsilon}^b\dd{x}f(x)\right).
\end{equation}

The example~\eqref{eq:principalEasy} was pretty straightforward. A more
interesting example would be
\begin{equation}
  \frac{f(x)}{x-x_0},
\end{equation}
where $f(z)$, $z=x+iy$, has no poles on the real axis. We will also require $f$
to decay in absolute value sufficiently quickly as we move away from the origin.
To determine the principal value, we can use a contour like the one
shown\footnote{If you like, you can also choose $c_2$ to close below the real
axis. Then the singularity $x_0$ will have to be considered in your residue
theorem. The result is the same in either case.}
in \figref{fig:principalValue}. The contours $c_1$ and $c_3$ combine to give
the principal value, and we therefore have by the residue theorem
\begin{equation}
  \principal\int_{-\infty}^{\infty}\dd{x}\frac{f(x)}{x-x_0}
   +\left(\int_{c_2}+\int_{c_4}\right)\dd{z}\frac{f(z)}{z-x_0}
   =2\pi i\sum_k\Res \frac{f(z_k)}{z_k-x_0}. 
\end{equation}
If $f$ decays fast enough, it will vanish on $c_4$, so that term can be
neglected. It remains to evaluate $f$ on $c_2$. We find
\begin{equation}
 \int_{c_2}\dd{z}\frac{f(z)}{z-x_0}
  =\lim_{\epsilon\to0}i\epsilon\int_\pi^0\dd{\phi}
      \frac{f(x_0+\epsilon e^{i\phi})}{\epsilon e^{i\phi}}=-\pi i f(x_0),
\end{equation}
and consequently,
\begin{equation}
  \principal\int_{-\infty}^{\infty}\dd{x}\frac{f(x)}{x-x_0}
   =\pi i f(x_0)+2\pi i\sum_k\Res \frac{f(z_k)}{z_k-x_0}. 
\end{equation}

In the special case where $f$ is analytic in the upper half plane,
there are no residues, and hence
\begin{equation}
  f(x_0)=\frac{1}{\pi i}\,\principal
     \int_{-\infty}^{\infty}\dd{x}\frac{f(x)}{x-x_0}.
\end{equation}
If we write $f(x)=\Re f(x)+i\Im f(x)$, we get from the above
\begin{equation}\begin{aligned}\label{eq:kramerKronig}
\Re f(x_0)&=\frac{1}{\pi}\,\principal
     \int_{-\infty}^{\infty}\dd{x}\frac{\Im f(x)}{x-x_0}\\
\Im f(x_0)&=-\frac{1}{\pi}\,\principal
     \int_{-\infty}^{\infty}\dd{x}\frac{\Re f(x)}{x-x_0}.
\end{aligned}\end{equation}
We call \equatref{eq:kramerKronig} the 
{\it Kramers-Kronig relations}.\index{Kramers-Kronig relations}
They show that the real and imaginary parts of a function $f$ with the given
properties are closely related to each other along the real line.



\section{Pad\'e approximants}\index{Pad\'e approximant}

To define the Pad\'e approximant, we introduce\footnote{One can always force
the first coefficient in the denominator to be 1 by factoring it out of all
the other coefficients in both the numerator and denominator.} first the {\it rational
function of order} $[m,n]$, $m,n\in\N$ as \index{function!rational}
\begin{equation}
  R_n^m(x)\equiv\frac{\sum_{i=0}^m a_ix^i}{1+\sum_{j=1}^nb_jx^j}.
\end{equation}


\bibliographystyle{unsrtnat}
\bibliography{bibliography}
