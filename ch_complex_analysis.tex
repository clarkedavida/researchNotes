\chapter{Math: Complex Analysis} 

Here we review some basics of complex analysis that often appear in physics
calculation. Some useful resources for this include
Ref.~\cite{brown_complex_2003,weber_essentials_2012}.

\section{Preliminaries}

We are interested in functions $f:\C\to\C$. Derivatives are defined similarly as
with real, $1D$ functions. The limit
\begin{equation}\label{eq:cderiv}
  \dv{f}{z}\equiv\lim_{w\to z}\frac{f(w)-f(z)}{w-z}
\end{equation}
can be approached in infinitely many ways in $\C$. We call \equatref{eq:cderiv} the
the derivative of $f$ when this limit exists and is independent of the chosen
path\footnote{This analogous to how the derivative can only be well defined in
$\R$ if the limit is the same whether one approaches from the left or right.}.
The function is said to be {\it holomorphic}. If it can be expanded as a
convergent Taylor series, it is {\it analytic}.
\index{analytic}\index{holomorphic}

Let
\begin{equation}
  f(z)=f(x+iy)=u(x,y)+iv(x,y),
\end{equation}
where $x,y\in\R$ and $u,v:\R\to\R$. \index{Cauchy-Riemann equations}
\begin{theorem}{Cauchy-Riemann equations}{}
$f$ is holomorphic if and only if
$$
  \pdv{u}{x}=\pdv{v}{y},~~~~~~~~\pdv{u}{y}=-\pdv{v}{x}.
$$
\begin{proof}
If the derivative exists, then by definition
it exists along a pure imaginary, vertical path approaching $z$, as well
as a pure real, horizontal path approaching $z$.
Then using \equatref{eq:cderiv} one obtains
$$
\pdv{u}{x}+i\pdv{v}{x}=\dv{f}{z}=-i\pdv{u}{y}+\pdv{v}{y},
$$
from which the result immediately follows.
\end{proof}
\end{theorem}

Since the definition in $\C$ is analogous to that in $\R$, the derivative obeys
the product and chain rules. The Cauchy-Riemann equations are a convenient way
to determine whether a complex function is holomorphic. For instance you can
use them to show $|z|^2$ is not holomorphic. 
It is also useful to note that $\exp(z)$ is holomorphic in $\C$.

Moving on to integration, complex integrals are in general path-dependent.
This integral can be defined using a Riemann sum, but in practice one usually
parameterizes the curve and computes the integral that way.
It follows that the integral is linear, and that the sign of an integral
flips when the orientation of the contour changes.
The following Proposition establishes an integral that appears frequently
in complex analysis.
\begin{proposition}{}{}
Let $C_R(z_0)$ denote the circle of radius $R$ centered at $z_0\in\C$
and let $n\in\Z$. Then
$$
\oint_{C_R(z_0)}\dd{z}(z-z_0)^n=
\begin{cases}
 2\pi i & n=-1\\
 0      & \text{otherwise}.
\end{cases}
$$
\begin{proof}
We can parameterize the contour with $z(\theta)=z_0+Re^{i\theta}$
with $\theta\in[0,2\pi)$. The integral evaluates to
\begin{equation*}\begin{aligned}
\oint_{C_R(z_0)}\dd{z}(z-z_0)^n&=
i\int_0^{2\pi}\dd{\theta} R^{n+1}e^{i(n+1)\theta}\\
&=
\begin{cases}
 2\pi i & n=-1\\
 \frac{R^{n+1}}{n+1}e^{i(n+1)\theta}\,\big|_0^{2\pi} & \text{otherwise}. 
\end{cases}
\end{aligned}\end{equation*}
\end{proof}
\end{proposition}

\bibliographystyle{unsrtnat}
\bibliography{bibliography}
