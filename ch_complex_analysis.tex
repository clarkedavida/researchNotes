\chapter{Math: Complex Analysis} 

Here we review some basics of complex analysis that often appear in physics
calculation. Some useful resources for this include
Ref.~\cite{brown_complex_2003,weber_essentials_2012}.


\section{Preliminaries}

We are interested in functions $f:\C\to\C$. Derivatives are defined similarly as
with real, $1D$ functions. The limit
\begin{equation}\label{eq:cderiv}
  \dv{f}{z}\equiv\lim_{w\to z}\frac{f(w)-f(z)}{w-z}
\end{equation}
can be approached in infinitely many ways in $\C$. We call \equatref{eq:cderiv} the
the derivative of $f$ when this limit exists and is independent of the chosen
path\footnote{This analogous to how the derivative can only be well defined in
$\R$ if the limit is the same whether one approaches from the left or right.}.
The function is said to be {\it holomorphic}. If it can be expanded as a
convergent Taylor series, it is {\it analytic}.
\index{analytic}\index{holomorphic}

Following the spirit of the above discussion, many of the facts we will learn
about complex analysis can be inherited from facts about real analysis. With
that in mind, for the rest of the chapter we will introduce for the complex
function $f$ the notation
\begin{equation}
  f(z)=f(x+iy)=u(x,y)+iv(x,y),
\end{equation}
where $x,y\in\R$ and $u,v:\R\to\R$. We start with a fact about when complex
functions are holomorphic.
\index{Cauchy-Riemann equations}
\begin{theorem}{Cauchy-Riemann (CR) equations}{}
$f$ is holomorphic if and only if
$$
  \pdv{u}{x}=\pdv{v}{y},~~~~~~~~\pdv{u}{y}=-\pdv{v}{x}.
$$
\begin{proof}
If the derivative exists, then by definition
it exists along a pure imaginary, vertical path approaching $z$, as well
as a pure real, horizontal path approaching $z$.
Then using \equatref{eq:cderiv} one obtains
$$
\pdv{u}{x}+i\pdv{v}{x}=\dv{f}{z}=-i\pdv{u}{y}+\pdv{v}{y},
$$
from which the result immediately follows.
\end{proof}
\end{theorem}

Since the definition in $\C$ is analogous to that in $\R$, the derivative obeys
the product and chain rules. The CR equations are a convenient way
to determine whether a complex function is holomorphic. For instance you can
use them to show $|z|^2$ is not holomorphic. 
It is also useful to note that $\exp(z)$ is holomorphic in $\C$.

Moving on to integration, complex integrals are in general path-dependent.
This integral can be defined using a Riemann sum, but in practice one usually
parameterizes the curve and computes the integral that way.
It follows that the integral is linear, and that the sign of an integral
flips when the orientation of the contour changes.
The following Proposition establishes an integral that appears frequently
in complex analysis.
\begin{proposition}{}{}
Let $C_R(z_0)$ denote the circle of radius $R$ centered at $z_0\in\C$
and let $n\in\Z$. Then
$$
\oint_{C_R(z_0)}\dd{z}(z-z_0)^n=
\begin{cases}
 2\pi i & n=-1\\
 0      & \text{otherwise}.
\end{cases}
$$
\begin{proof}
We can parameterize the contour with $z(\theta)=z_0+Re^{i\theta}$
with $\theta\in[0,2\pi)$. The integral evaluates to
\begin{equation*}\begin{aligned}
\oint_{C_R(z_0)}\dd{z}(z-z_0)^n&=
i\int_0^{2\pi}\dd{\theta} R^{n+1}e^{i(n+1)\theta}\\
&=
\begin{cases}
 2\pi i & n=-1\\
 \frac{R^{n+1}}{n+1}e^{i(n+1)\theta}\,\big|_0^{2\pi} & \text{otherwise}. 
\end{cases}
\end{aligned}\end{equation*}
\end{proof}
\end{proposition}

A region $B\subset\C$ is {\it path connected} if any two points in the region
can be joined by a path. A {\it simply connected} region is a path connected
\index{path connected}\index{simply connected}
region and any loop path can be contracted to a point\footnote{In other words,
there are no holes.}. It turns out that a wide class of complex integrals
over such regions evaluate to zero.

\index{Cauchy integral theorem}
\begin{theorem}{Cauchy integral theorem}{}
Let $B\subset\C$ be a bounded and simply connected region. Let $f:B\to\C$ be
holomorphic in $B$ and $C$ be a closed curve fully contained in $B$. Then
$$
  \oint_C\dd{z}f(z)=0.
$$
\begin{proof}
We have
\begin{equation*}\begin{aligned}
\oint_C\dd{z}f(z)=&\oint_C\left(\dd{x}u(x,y)-\dd{y}v(x,y)\right)\\
                 &+i\oint_C\left(\dd{y}u(x,y)+\dd{x}v(x,y)\right).
\end{aligned}\end{equation*}
Let $A$ be the area enclosed by $C$. By Stokes' theorem, we get
\begin{equation*}
\oint_C\dd{z}f(z)=\iint_A\dd{x}\dd{y}\left(-\pdv{v}{x}-\pdv{u}{y}\right)
                   +i\iint_A\dd{x}\dd{y}\left(\pdv{u}{x}-\pdv{v}{y}\right).
\end{equation*}
Since $f$ is holomorphic, the integrands vanish by the CR equations.
\end{proof}
\end{theorem}

\section{Pad\'e approximants}

\bibliographystyle{unsrtnat}
\bibliography{bibliography}
