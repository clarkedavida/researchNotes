\chapter{Special Topic: The Standard Model}\label{ap:spec_SM}

For this appendix we will roughly write the Lagrangian for the Standard Model in the
$R_\xi$ gauge and write down some of the corresponding Feynman rules. We will
make some general considerations involving two particle decays. Next we
specialize to the Feynman gauge and calculate decay rates for some of
the most important decay channels of the Higgs. 
%Finally we will compare our
%results to some known values to see how we did.

\section{The SM Lagrangian}
This section is based heavily off of Bardin and 
Passarino~\cite{bardin_standard_1999}. Unlike
them, however, we will use the metric with signature $(+\ -\ -\ -)$ and we will
\index{gauge!fixing}
define the gauge-fixing term slightly differently when we come to it.
Ultimately our Feynman rules should agree with Peskin and 
Schroeder~\cite{peskin_introduction_1995}.

We begin with the electroweak sector. The fields include a triplet of vector
bosons $B_{\mu}^{a}$ and a vector singlet $B_{\mu}^{0}$; a complex scalar
\index{Faddeev-Popov ghost}
doublet $K$; Faddeev-Popov (FP) ghost fields $X^{\pm}$, $Y^Z$, $Y^A$; and
fermions. The physical fields $Z$ and $A$ are given by
\begin{equation}
  \left(\begin{array}{c}
    Z \\
    A
  \end{array}\right)
  =
  \left(\begin{array}{cc}
    \ct & -\st \\
    \st & \ct
  \end{array}\right)
  \left(\begin{array}{c}
    B^3 \\
    B^0
  \end{array}\right),
\end{equation}
where $\st\coloneqq\sin\big(\theta_{W}\big)$,
$\ct\coloneqq\cos\big(\theta_{W}\big)$, and $\theta_W$ is the weak
mixing angle. The scalar field for the minimal Higgs model is
\begin{equation}
  \label{eq:Hsf}
  K=\frac{1}{\sqrt{2}}
  \left(\begin{array}{c}
    H+2M_{W}/g+i\phi^0 \\
    \sqrt{2}i\phi^-
  \end{array}\right),
\end{equation}
where $H$ is the physical Higgs field, $M_W$ is the bare $W$-boson mass, and
$g$ is the bare $SU(2)$ coupling. The first part of the Lagrangian is
\begin{equation}
  \Lagr_{gauge}=\Lagr_{YM}+\Lagr_{H},
\end{equation}
where
% what is \epsilon?
\begin{equation}
  \begin{aligned}
    \Lagr_{YM}&=-\frac{1}{4}F_{\mu\nu}^{a}F^{\mu\nu,a}
                -\frac{1}{4}F_{\mu\nu}^{0}F^{\mu\nu,0} \\
    \Lagr_{H}&=-\big(D_{\mu}K\big)^{\dagger}\big(D^{\mu}K\big)
               -\mu^2 K^{\dagger}K 
               -\frac{\lambda}{2}\big(K^{\dagger}K\big)^2 \\
    F_{\mu\nu}^{a}&=\partial_{\mu}B_{\nu}^{a}-\partial_{\nu}B_{\mu}^{a}
                   +g\epsilon^{abc}B_{\mu}^{b}B_{\nu}^{c} \\
    F_{\mu\nu}^{0}&=\partial_{\mu}B_{\nu}^{0}-\partial_{\nu}B_{\mu}^{0}
  \end{aligned}
\end{equation}
and we need $\lambda>0$ (to have a ground state) and $\mu^2<0$ (for spontaneous
symmetry breaking). The $SU(2)\times U(1)$ gauge covariant derivative is
\begin{equation}
  \label{eq:SU2U1cd}
  D_{\mu}=\partial_{\mu}-\frac{i}{2}gB_{\mu}^{a}\sigma^{a}
                       -\frac{i}{2}gg'B_{\mu}^{0},
\end{equation}
where $\sigma^{a}$ are the Pauli matrices, and $g'=-\st/\ct$.
Also \eqref{eq:Hsf} can be rewritten as
\begin{equation}
  \label{eq:Hsf2}
  K=\frac{1}{\sqrt{2}}\Big(H+\frac{2M_W}{g}+i\phi^{a}\sigma^{a}\Big)
  \left(\begin{array}{c}
    1 \\
    0
  \end{array}\right).
\end{equation}

Next let us write
\begin{equation}
  \Lagr_{gauge}=\Lagr_{YM}-\big(D_{\mu}K\big)^{\dagger}\big(D^{\mu}K\big)
                 +\Lagr_{int}.
\end{equation}
If we were to calculate $\big(D_{\mu}K\big)^{\dagger}\big(D^{\mu}K\big)$ using
\eqref{eq:SU2U1cd} and \eqref{eq:Hsf2}, collect terms proportional to $v^2$,
where
\begin{equation}
  v=\frac{2M_W}{g},
\end{equation}
and use the field definitions
\begin{equation}
  W_{\mu}^{\pm}\coloneqq\frac{1}{\sqrt{2}}\big(B_{\mu}^1 \mp iB_{\mu}^2\big)
  \qquad
  \phi^{\pm}\coloneqq\frac{1}{\sqrt{2}}\big(\phi^1 \mp i\phi^2 \big),
\end{equation}
we would get
\begin{equation}
  \begin{aligned}
    \big(D_{\mu}K\big)^{\dagger}\big(D^{\mu}K\big)&=-\frac{g^2 v^2}{8}
      \Bigg(\frac{1}{\ct^2}Z_{\mu}Z^{\mu}+2W^+_{\mu}W^{-\mu}\Bigg)+...\\
    &=-\frac{1}{2}M_Z^2 Z_{\mu}Z^{\mu}-M_W^2 W_{\mu}^+ W^{-\mu}+...
  \end{aligned}
\end{equation}
where
\begin{equation}
  M_Z \coloneqq\frac{gv}{2\ct}=\frac{M_W}{\ct}.
\end{equation}
Hence we see that through the Higgs mechanism the $W$- and $Z$-bosons
acquired mass.
Now let us focus on terms of $(D_{\mu}K\big)^{\dagger}\big(D^{\mu}K\big)$ that
do not have any power of $g$ (besides kinetic terms). We get
\begin{equation}
  \label{eq:zmp}
  \big(D_{\mu}K\big)^{\dagger}\big(D^{\mu}K\big)=-M_W \Bigg(
     \frac{1}{\ct}Z_{\mu}\partial^{\mu}\phi^0 +W_{\mu}^+ \partial^{\mu}\phi^-
     +W_{\mu}^- \partial^{\mu}\phi^+ \Bigg).
\end{equation}
This part of the Lagrangian contains $Z$-$\phi^0$ and
$W^{\pm}$-$\phi^{\pm}$ interactions, and although they are zeroth order
in the field, they must be summed over when we develop perturbation theory.
Here the Lagrangian is badly divergent.

To fix this problem, we follow the usual prescription of introducing a 
gauge-fixing term, then introducing FP ghost fields to maintain gauge invariance. We
\index{gauge!fixing}
work in the generalized $R_{\xi}$ gauge. (The Feynman gauge corresponds to
$\xi=1$, the Lorenz to $\xi=0$, and the unitary to $\xi\to\infty$.) In this
gauge, the gauge-fixing piece is
\begin{equation}
  \Lagr_{gf}=-\frac{1}{2}C^a C^a -\frac{1}{2}\big(C^0\big)^2
            =-C^+ C^- -\frac{1}{2}\Big(\big(C^3\big)^2\big(C^0\big)^2\Big),
\end{equation}
where the gauge-fixing function is
\begin{equation}
  C^a =\frac{1}{\sqrt{\xi}}\big(\partial_{\mu}B^{a\mu}
                -\xi M_W \phi^a \big)
\end{equation}
and
\begin{equation}
  \label{eq:cdefs}
  C^{\pm}=\frac{1}{\sqrt{\xi}}\big(\partial_{\mu}W^{\pm\mu}
                   -\xi M_W \phi^{\pm}\big) \qquad
  C^{0}=\frac{1}{\sqrt{\xi}}\big(\partial_{\mu}B^{0\mu}
                   -\xi\frac{\st}{\ct}M_W \phi^{0}\big).
\end{equation}
Then we write
\begin{equation}
  -\frac{1}{2}\Big(\big(C^3\big)^2 +\big(C^0\big)^2\Big)
       =-\frac{1}{2}C_Z^2 -\frac{1}{2}C_A^2
\end{equation}
so that in the so-called $Z$-$A$ basis
\begin{equation}
  \label{eq:zadefs}
  C_A=\frac{1}{\sqrt{\xi}}\partial_{\mu}A^{\mu} \qquad
  C_Z=\frac{1}{\sqrt{\xi}}\partial_{\mu}Z^{\mu}-\xi\frac{M_W}{\ct}\phi^0.
\end{equation}
The Lagrangian including gauge-fixing terms is thus
\begin{equation}
  \Lagr_{YM}=-\big(D_{\mu}K\big)^{\dagger}\big(D^{\mu}K\big)-C^+ C^-
               -\frac{1}{2}C_Z^2-\frac{1}{2}C_Z^2
            =\Lagr_{prop}+\Lagr_{int}^{bos}
\end{equation}
and contains no divergent terms such as those in \eqref{eq:zmp}. $\Lagr_{prop}$
contains the terms that give rise to gauge EW propagators as well as mass
terms. Let us examine some terms relevant to Higgs processes. We have
\begin{equation}
  \begin{aligned}
  \Lagr_{prop}=&-\partial_{\mu}W_{\nu}^+ \partial^{\mu}W^{-\nu}
     +\Bigg(1-\frac{1}{\xi}\Bigg)\partial_{\mu}W^{+\mu}\partial_{\nu}W^{-\nu}
     -M_W^2 W_{\mu}^+ W^{-\mu} \\
    &-\frac{1}{2}\partial_{\mu}Z_{\nu}\partial^{\mu}Z^{\nu}
     +\frac{1}{2}\Bigg(1-\frac{1}{\xi}\Bigg)\big(\partial_{\mu}Z^{\mu}\big)^2
     -\frac{1}{2}M_Z^2 Z_{\mu}Z^{\mu} \\
    &-\partial_{\mu}\phi^+ \partial^{\mu}\phi^- -\xi M_W^2\phi^+ \phi^-
     +...
  \end{aligned}
\end{equation}
First of all we see that the Goldstone boson mass is proportional $\sqrt{\xi}$,
which means it's unphysical. (We would find the same for $\phi^0$.) Furthermore
the first line, second line, and third line yield respectively the Feynman
rules
\begin{equation*}
  \frac{-ig^{\mu\nu}}{k^2-M_W^2+i\epsilon}~~\F1~~~~~~~~
  \frac{-ig^{\mu\nu}}{k^2-M_Z^2+i\epsilon}~~\F2~~~~~~~~
  \frac{i}{p^2-\xi M_W^2+i\epsilon}~~\F3
\end{equation*}
\begin{figure}
  \centering
  \begin{fmffile}{fr123}
    \begin{subfigure}{0.3\textwidth}
      \centering
      \begin{fmfgraph*}(70,60)
        \fmfleft{i}
        \fmfright{o}
        \fmflabel{$\mu$}{i}
        \fmflabel{$\nu$}{o}
        \fmf{photon,label=$W^\pm$}{i,o}
      \end{fmfgraph*}
    \caption*{\F1}
    \end{subfigure}
    \begin{subfigure}{0.3\textwidth}
      \centering
      \begin{fmfgraph*}(70,60)
        \fmfleft{i}
        \fmfright{o}
        \fmflabel{$\mu$}{i}
        \fmflabel{$\nu$}{o}
        \fmf{photon,label=$Z$}{i,o}
      \end{fmfgraph*}
    \caption*{\F2}
    \end{subfigure}
    \begin{subfigure}{0.3\textwidth}
      \centering
      \begin{fmfgraph*}(70,60)
        \fmfleft{i}
        \fmfright{o}
        \fmf{dashes,label=$\phi^\pm$}{i,o}
      \end{fmfgraph*}
    \caption*{\F3}
    \end{subfigure}
  \end{fmffile}
  \caption{Electroweak gauge boson propagators (except photon) and Goldstone
           boson propagator.}
\end{figure}

Meanwhile $\Lagr_{int}^{bos}$ contains the interaction terms of the vector
bosons with each other and the Higgs. We will focus on the terms
\begin{equation}
  \begin{aligned}
    \Lagr_{int}^{bos}=&-ig\ct\Big\{\partial_\nu Z_\mu W^{[+\mu}W^{-\nu ]}
                       -Z_\nu W_\mu^{[+}\partial^\nu W^{-\mu ]}
                       +Z_\mu W_\nu^{[+}\partial^\nu W^{-\mu ]}\Big\}\\
                      &-ie\Big\{\partial_\nu A_\mu W^{[+\mu}W^{-\nu]}
                       -A_\nu W_\mu^{[+}\partial^\nu W^{-\mu ]}
                       +A_\mu W_\nu^{[+}\partial^\nu W^{-\mu ]}\Big\}\\
                      &+e^2 \Big\{A_\mu A_\nu W^{+\mu}W^{-\nu}
                       +A_\mu A^\mu W^+_\nu W^{-\nu}\Big\}\\
                      &+eg\ct\Big\{A_\mu Z_\nu W^{[+\mu}W^{-\nu ]}
                       -2A_\mu Z^\mu W^+_\nu W^{-\nu}\Big\}\\
                      &+ig\frac{\ct^2-\st^2}{\ct}Z_\mu
                         \Big\{\phi^+ \partial^\mu \phi^-
                               -\phi^- \partial^\mu \phi^+ \Big\}\\
                      &+\frac{1}{2}g
                         \Big\{W_\mu^+\big(H\partial^\mu\phi^- 
                               -\phi^-\partial^\mu H\big)
                              -W_\mu^-\big(H\partial^\mu\phi^+ 
                               -\phi^+\partial^\mu H\big)\Big\}\\
                      &+ieM_W A_\mu W^{[+\mu}\phi^{-]}
                       -igM_Z \st^2 Z_\mu W^{[+\mu}\phi^{-]}
                       -gM_W H W_\mu^+ W_\nu^- \\
                      &-e^2 A_\mu A^\mu \phi^+ \phi^-
                       +\frac{i}{2}egA_\mu HW^{[+\mu}\phi^{-]}
                       -eg\frac{\ct^2-\st^2}{\ct}Z_\mu A^\mu \phi^+ \phi^- \\
                      &+ieA_\mu \big(\phi^+\partial^\mu\phi^-
                                     -\phi^-\partial^\mu\phi^+\big)
                       +...
  \end{aligned}
\end{equation}
where $e=g\st$ and $A^{[+}B^{-]}=A^+ B^- -B^- A^+$. The corresponding Feynman
rules are
\begin{gather*}
  ig\ct\big(g^{\alpha\beta}(p-k)^\mu+g^{\beta\mu}(k-q)^\mu
            +g^{\mu\alpha}(q-p)^\beta\big)~~\F4 \\
  -ie\big(g^{\alpha\beta}(p-k)^\mu+g^{\beta\mu}(k-q)^\mu
          +g^{\mu\alpha}(q-p)^\beta\big)~~\F5 \\
  -ie^2\big(2g^{\alpha\beta}g^{\mu\nu}-g^{\alpha\mu}g^{\beta\nu}
            -g^{\alpha\nu}g^{\beta\mu}\big)~~\F6 \\
  ieg\ct\big(2g^{\alpha\beta}g^{\mu\nu}-g^{\alpha\mu}g^{\beta\nu}
            -g^{\alpha\nu}g^{\beta\mu}\big)~~\F7 \\
  ig\frac{\cos(2\theta_W)}{2\ct}(p_+-p_-)^\mu~~\F8~~~~~~~~
  \mp\frac{ig}{2}(k-p)^\mu~~\F9~~~~~~~~
  -ieM_Wg^{\mu\nu}~~\F{10} \\
  -igM_Z \st^2 g^{\mu\nu}~~\F{11}~~~~~~~~
  igM_Wg^{\mu\nu}~~\F{12}~~~~~~~~
  2ie^2g^{\mu\nu}~~\F{13} \\
  -\frac{i}{2}egg^{\mu\nu}~~\F{14}~~~~~~~~
  -ieg\frac{\cos(2\theta_W)}{\ct}g^{\mu\nu}~~\F{15}~~~~~~~~
  -ie(p_+-p_-)^\mu~~\F{16}
\end{gather*}
\begin{figure}
  \centering
  \begin{fmffile}{fr4to17}
    \begin{subfigure}{0.3\textwidth}
      \centering
      \begin{fmfgraph*}(60,60)
        \fmfleft{i1,i2}
        \fmfright{o}
        \fmf{photon,label=$k$}{i1,v}
        \fmf{photon,label=$p$}{i2,v}
        \fmf{photon,label=$q$}{v,o}
        \fmflabel{$W_\beta^+$}{i1}
        \fmflabel{$W_\alpha^-$}{i2}
        \fmflabel{$Z_\mu$}{o}
      \end{fmfgraph*}
    \caption*{\F4}
    \end{subfigure}
    \begin{subfigure}{0.3\textwidth}
      \centering
      \begin{fmfgraph*}(60,60)
        \fmfleft{i1,i2}
        \fmfright{o}
        \fmf{photon,label=$k$}{i1,v}
        \fmf{photon,label=$p$}{i2,v}
        \fmf{photon,label=$q$}{v,o}
        \fmflabel{$W_\beta^+$}{i1}
        \fmflabel{$W_\alpha^-$}{i2}
        \fmflabel{$A_\mu$}{o}
      \end{fmfgraph*}
    \caption*{\F5}
    \end{subfigure}
    \begin{subfigure}{0.3\textwidth}
      \centering
      \begin{fmfgraph*}(60,60)
        \fmfleft{i1,i2}
        \fmfright{o1,o2}
        \fmf{photon}{i1,v}
        \fmf{photon}{i2,v}
        \fmf{photon}{v,o1}
        \fmf{photon}{v,o2}
        \fmflabel{$W_\alpha^+$}{i2}
        \fmflabel{$A_\mu$}{i1}
        \fmflabel{$W_\beta^-$}{o2}
        \fmflabel{$A_\nu$}{o1}
      \end{fmfgraph*}
    \caption*{\F6}
    \end{subfigure} \\
    \vspace{12mm}
   \begin{subfigure}{0.3\textwidth}
      \centering
      \begin{fmfgraph*}(60,60)
        \fmfleft{i1,i2}
        \fmfright{o1,o2}
        \fmf{photon}{i1,v}
        \fmf{photon}{i2,v}
        \fmf{photon}{v,o1}
        \fmf{photon}{v,o2}
        \fmflabel{$W_\alpha^+$}{i2}
        \fmflabel{$A_\mu$}{i1}
        \fmflabel{$W_\beta^-$}{o2}
        \fmflabel{$Z_\nu$}{o1}
      \end{fmfgraph*}
    \caption*{\F7}
    \end{subfigure}
    \begin{subfigure}{0.3\textwidth}
      \centering
      \begin{fmfgraph*}(60,60)
        \fmfleft{i1,i2}
        \fmfright{o}
        \fmf{dashes,label=$p_-$}{i1,v}
        \fmf{dashes,label=$p_+$}{i2,v}
        \fmf{photon}{v,o}
        \fmflabel{$\phi^-$}{i1}
        \fmflabel{$\phi^+$}{i2}
        \fmflabel{$Z_\mu$}{o}
      \end{fmfgraph*}
    \caption*{\F8}
    \end{subfigure}
    \begin{subfigure}{0.3\textwidth}
      \centering
      \begin{fmfgraph*}(60,60)
        \fmfleft{i1,i2}
        \fmfright{o}
        \fmf{dashes,label=$p$}{i1,v}
        \fmf{dashes,label=$k$}{i2,v}
        \fmf{photon}{v,o}
        \fmflabel{$\phi^\mp$}{i1}
        \fmflabel{$H$}{i2}
        \fmflabel{$W^\pm_\mu$}{o}
      \end{fmfgraph*}
    \caption*{\F9}
    \end{subfigure} \\
    \vspace{12mm}
    \begin{subfigure}{0.3\textwidth}
      \centering
      \begin{fmfgraph*}(60,60)
        \fmfleft{i1,i2}
        \fmfright{o}
        \fmf{photon}{i1,v}
        \fmf{dashes}{i2,v}
        \fmf{photon}{v,o}
        \fmflabel{$W_\nu^\pm$}{i1}
        \fmflabel{$\phi^\mp$}{i2}
        \fmflabel{$A_\mu$}{o}
      \end{fmfgraph*}
    \caption*{\F{10}}
    \end{subfigure}
    \begin{subfigure}{0.3\textwidth}
      \centering
      \begin{fmfgraph*}(60,60)
        \fmfleft{i1,i2}
        \fmfright{o}
        \fmf{photon}{i1,v}
        \fmf{dashes}{i2,v}
        \fmf{photon}{v,o}
        \fmflabel{$W_\nu^\pm$}{i1}
        \fmflabel{$\phi^\mp$}{i2}
        \fmflabel{$Z_\mu$}{o}
      \end{fmfgraph*}
    \caption*{\F{11}}
    \end{subfigure}
    \begin{subfigure}{0.3\textwidth}
      \centering
      \begin{fmfgraph*}(60,60)
        \fmfleft{i1,i2}
        \fmfright{o}
        \fmf{photon}{i1,v}
        \fmf{dashes}{i2,v}
        \fmf{photon}{v,o}
        \fmflabel{$W_\nu^\mp$}{i1}
        \fmflabel{$H$}{i2}
        \fmflabel{$W_\mu^\pm$}{o}
      \end{fmfgraph*}
    \caption*{\F{12}}
    \end{subfigure} \\
    \vspace{12mm}
    \begin{subfigure}{0.3\textwidth}
      \centering
      \begin{fmfgraph*}(60,60)
        \fmfleft{i1,i2}
        \fmfright{o1,o2}
        \fmf{dashes}{i1,v}
        \fmf{dashes}{i2,v}
        \fmf{photon}{v,o1}
        \fmf{photon}{v,o2}
        \fmflabel{$\phi^+$}{i2}
        \fmflabel{$\phi^-$}{i1}
        \fmflabel{$A_\mu$}{o2}
        \fmflabel{$A_\nu$}{o1}
      \end{fmfgraph*}
    \caption*{\F{13}}
    \end{subfigure}
    \begin{subfigure}{0.3\textwidth}
      \centering
      \begin{fmfgraph*}(60,60)
        \fmfleft{i1,i2}
        \fmfright{o1,o2}
        \fmf{dashes}{i1,v}
        \fmf{dashes}{i2,v}
        \fmf{photon}{v,o1}
        \fmf{photon}{v,o2}
        \fmflabel{$\phi^\pm$}{i2}
        \fmflabel{$H$}{i1}
        \fmflabel{$W_\mu^\mp$}{o2}
        \fmflabel{$A_\nu$}{o1}
      \end{fmfgraph*}
    \caption*{\F{14}}
    \end{subfigure}
    \begin{subfigure}{0.3\textwidth}
      \centering
      \begin{fmfgraph*}(60,60)
        \fmfleft{i1,i2}
        \fmfright{o1,o2}
        \fmf{dashes}{i1,v}
        \fmf{dashes}{i2,v}
        \fmf{photon}{v,o1}
        \fmf{photon}{v,o2}
        \fmflabel{$\phi^+$}{i2}
        \fmflabel{$\phi^-$}{i1}
        \fmflabel{$Z_\mu$}{o2}
        \fmflabel{$A_\nu$}{o1}
      \end{fmfgraph*}
    \caption*{\F{15}}
    \end{subfigure} \\
    \vspace{12mm}
   \begin{subfigure}{0.3\textwidth}
      \centering
      \begin{fmfgraph*}(60,60)
        \fmfleft{i1,i2}
        \fmfright{o}
        \fmf{dashes,label=$p_-$}{i1,v}
        \fmf{dashes,label=$p_+$}{i2,v}
        \fmf{photon}{v,o}
        \fmflabel{$\phi^-$}{i1}
        \fmflabel{$\phi^+$}{i2}
        \fmflabel{$A_\mu$}{o}
      \end{fmfgraph*}
    \caption*{\F{16}}
    \end{subfigure}
    \begin{subfigure}{0.3\textwidth}
      \centering
      \begin{fmfgraph*}(60,60)
        \fmfleft{i1,i2}
        \fmfright{o}
        \fmf{dashes,label=$p_-$}{i1,v}
        \fmf{dashes,label=$p_+$}{i2,v}
        \fmf{dashes}{v,o}
        \fmflabel{$\phi^-$}{i1}
        \fmflabel{$\phi^+$}{i2}
        \fmflabel{$H$}{o}
      \end{fmfgraph*}
    \caption*{\F{17}}
    \end{subfigure}
  \end{fmffile}
  \caption{Some interaction diagrams between electroweak bosons, Goldstone
           bosons, and the Higgs. Momentum arrows are taken to be toward
           the vertex center.}
\end{figure}

The interactions among only Higgs and Goldstone fields emerge from the Higgs
potential
\begin{equation}
  \Lagr_{int}^H=-\mu^2 K^\dagger K-\frac{\lambda}{2}\big(K^\dagger K\big)^2
\end{equation}
after spontaneous symmetry breaking. The only one from this sector that we
will need is
\begin{equation}
  \Lagr_{int}^H=\frac{1}{2}g\frac{M_H^2}{M_W}H\phi^+ \phi^- +...
\end{equation}
and the corresponding Feynman rule is
\begin{gather*}
  -\frac{ig}{2}\frac{M_H^2}{M_W}~~\F{17}
\end{gather*}


To finish off the scalar part of the electroweak sector, we introduce the FP
Lagrangian, which we accomplish by taking gauge transformations of the $C^a$.
The gauge transformations of the bosonic fields are
\begin{gather}
  B_\mu^a\to B_\mu^a+g\epsilon^{abc}\Lambda^bB^c_\mu \qquad
  B^0_\mu\to B^0_\mu-\partial_\mu\Lambda^0 \\
  K\to \Bigg(1-\frac{ig}{2}\Lambda^2\sigma^a-\frac{i}{2}gg'\Lambda^0\Bigg)K
\end{gather}
The $\Lambda^i,i=0,1,2,3$ are the group gauge transformation parameters. The
physical gauge transformation parameters $\Lambda^j,j=+,-,Z,A$ are
specified by
\begin{gather}
  \label{eq:lparam}
  \Lambda^1=\frac{1}{\sqrt{2}}\big(\Lambda^++\Lambda^-\big) \qquad
  \Lambda^2=\frac{i}{\sqrt{2}}\big(\Lambda^+-\Lambda^-\big) \\
  \nonumber
  \Lambda^3=\ct\Lambda^Z+\st\Lambda^A \qquad
  \Lambda^0=-\st\Lambda^Z+\ct\Lambda^A \\
  \nonumber
  \Lambda^3+g'\Lambda^0=\frac{1}{\ct}\Lambda^Z \qquad
  -\Lambda^3+g'\Lambda^0=-\frac{\ct^2-\st^2}{\ct}\Lambda^Z-2\st\Lambda^A.
\end{gather}
Using \eqref{eq:lparam} we can derive the gauge transformations of the
physical fields in terms of the physical parameters. Combining this with
\eqref{eq:cdefs} and \eqref{eq:zadefs}, we get the gauge transformations of
$C^\pm,C^A,$ and $C^Z$ in terms of physical parameters. For the purposes
of this exam, the most important part is
\begin{equation}
  \label{eq:cxform}
  C^\pm \to C^\pm +\frac{1}{\sqrt{\xi}}\big(\Box\Lambda^\pm
                         +\xi M_W^2 \Lambda^\pm\big)+...
\end{equation}
where $\Box$ is the D'Alembertian operator. Just as in the case of gauge
fields, we associate with each physical gauge parameter
$\Lambda^\pm,\Lambda^Z,\Lambda^A$ a ghost field $X^\pm,Y^Z,Y^A$ by
\begin{gather}
  X^1=\frac{1}{\sqrt{2}}\big(X^++X^-\big), \qquad
  X^2=\frac{1}{\sqrt{2}}\big(X^+-X^-\big), \\
  X^3=\ct Y^Z+\st Y^A, \qquad
  X^0=-\st Y^Z +\ct Y^A. \nonumber
\end{gather}
The gauge
invariance $C^\pm\to C^\pm$ is restored if we identify the first term in
\eqref{eq:cxform} as the ghost propagator
\begin{gather*}
  \frac{i}{k^2-\xi M_W^2+i\epsilon}~~\F{18}
\end{gather*}
and identify ghost field interactions of the form
\begin{equation}
  g\overline{X}^i L^{ij} X^j, \qquad i,j=+,-,Z,A.
\end{equation}
The interaction Lagrangian for the electroweak ghosts then becomes (again
we're only looking at some of the terms)
\begin{gather}
  \Lagr_{int}^{gf}=ie\frac{1}{\sqrt{\xi}}A_\mu
                       \big(\partial^\mu \overline{X}^+X^+
                            -\partial^\mu \overline{X}^-X^-\big)
                   +ig\ct\frac{1}{\sqrt{\xi}}Z_\mu
                       \big(\partial^\mu \overline{X}^+X^+
                            -\partial^\mu \overline{X}^-X^-\big)\\
                   -\frac{1}{2}g\sqrt{\xi}M_W H
                   \big(\overline{X}^+X^+ -\overline{X}^-X^-\big)+...\nonumber
\end{gather}
with Feynman rules
\begin{gather*}
 \mp iep^\mu~~\F{19}~~~~~~~~
 \pm ig\ct p^\mu~~\F{20}~~~~~~~~
 -\frac{ig}{2}\sqrt{\xi}M_W~~\F{21}
\end{gather*}
\begin{figure}
  \begin{fmffile}{fr18to21}
  \centering
    \begin{subfigure}{0.45\textwidth}
      \centering
      \begin{fmfgraph*}(60,60)
        \fmfleft{i}
        \fmfright{o}
        \fmf{dots,label=$X^\pm$}{i,o}
      \end{fmfgraph*}
    \caption*{\F{18}}
    \end{subfigure}
    \begin{subfigure}{0.45\textwidth}
      \centering
      \begin{fmfgraph*}(60,60)
        \fmfleft{i1,i2}
        \fmfright{o}
        \fmf{dots}{i1,v}
        \fmf{dots,label=$p$}{i2,v}
        \fmf{photon}{v,o}
        \fmflabel{$X^\pm$}{i1}
        \fmflabel{$X^\pm$}{i2}
        \fmflabel{$A_\mu$}{o}
      \end{fmfgraph*}
    \caption*{\F{19}}
    \end{subfigure} \\
    \vspace{12mm}
    \begin{subfigure}{0.45\textwidth}
      \centering
      \begin{fmfgraph*}(60,60)
        \fmfleft{i1,i2}
        \fmfright{o}
        \fmf{dots}{i1,v}
        \fmf{dots,label=$p$}{i2,v}
        \fmf{photon}{v,o}
        \fmflabel{$X^\pm$}{i1}
        \fmflabel{$X^\pm$}{i2}
        \fmflabel{$Z_\mu$}{o}
      \end{fmfgraph*}
    \caption*{\F{20}}
    \end{subfigure}
    \begin{subfigure}{0.45\textwidth}
      \centering
      \begin{fmfgraph*}(60,60)
        \fmfleft{i1,i2}
        \fmfright{o}
        \fmf{dots}{i1,v}
        \fmf{dots}{i2,v}
        \fmf{photon}{v,o}
        \fmflabel{$X^\pm$}{i1}
        \fmflabel{$X^\pm$}{i2}
        \fmflabel{$H$}{o}
      \end{fmfgraph*}
    \caption*{\F{21}}
    \end{subfigure}
  \end{fmffile}
  \caption{The ghost propagator and some ghost-electroweak boson interaction
           terms.}
\end{figure}
\noindent
It's worth noting that the ghost field mass terms are proportional to
$\sqrt{\xi}$, which makes them not physical.


Next we move to fermions. We start with
\begin{gather}
  \label{eq:ferms}
  \Psi=
  \left(\begin{array}{c}
    u \\
    d
  \end{array}\right) \qquad
  \Psi_{L,R}=\frac{1}{2}(1\pm\gamma_5)\Psi \\
  \nonumber
  D_\mu\Psi_L=\big(\partial_\mu+gB_\mu^iT^i\big)\Psi_L \qquad
  T^a=-\frac{i}{2}\sigma^a \qquad
  T^0=-\frac{i}{2}g_2I \\
  \nonumber
  D_\mu\Psi_R=\big(\partial_\mu+gB_\mu^it^i\big)\Psi_R \qquad
  t^a=0 \qquad
  t^0=-\frac{1}{2}
  \left(\begin{array}{cc}
    g_3 & 0  \\
    0   & g_4
  \end{array}\right),
\end{gather}
where $i=0,1,2,3$.
The part of the Lagrangian corresponding to interactions between fermions
and weak vector bosons is
\begin{equation}
  \Lagr_{int}^{vfer}=-\overline{\Psi}_L \slashed{D}\Psi_L
                     -\overline{\Psi}_R \slashed{D}\Psi_R, \qquad
               g_i=-\frac{\st}{\ct}\lambda_i.
\end{equation}
The parameters $\lambda_i$ are fixed by the requirement that the EM current
take the form $iQ_f e \bar{f}\gamma_\mu f$. We get
\begin{equation}
  \lambda_2=1-2Q_u=1-2Q_d, \qquad \lambda_3=-2Q_u, 
            \qquad \text{and} \qquad \lambda_4=-2Q_d,
\end{equation}
and we define $I_f^{(3)}$ by
\begin{equation}
  Q_f=2I_f^{(3)}|Q_f|.
\end{equation}
With these definitions we can recast \eqref{eq:ferms} in a more transparent
form. The parts that interest us are
\begin{equation}
  \label{eq:vfer}
  \Lagr_{int}^{vfer}=\sum\limits_f\Bigg[ieQ_fA_\mu \bar{f}\gamma_\mu f
                      +\frac{ig}{2\ct}Z_\mu\bar{f}\gamma^\mu
                        \Big(I_f^{(3)}-2Q_f\st^2+I_f^{(3)}\gamma_5\Big)f\Bigg]
                      +...
\end{equation}
where the sum is over possible fermion types. The Feynman rules corresponding
to these two terms are
\begin{gather*}
  -ieQ_f \gamma^\mu~~\F{22}~~~~~~~~
  \frac{ig}{2\ct}\gamma^\mu\Big(I_f^{(3)}
                                -2Q_f\st^2+I_f^{(3)}\gamma_5\Big)~~\F{23}
\end{gather*}
\begin{figure}
  \begin{fmffile}{fr22to25}
  \centering
    \begin{subfigure}{0.45\textwidth}
      \centering
      \begin{fmfgraph*}(60,60)
        \fmfleft{i1,i2}
        \fmfright{o}
        \fmf{fermion}{i1,v,i2}
        \fmf{photon}{v,o}
        \fmflabel{$\bar{f}$}{i1}
        \fmflabel{$f$}{i2}
        \fmflabel{$A_\mu$}{o}
      \end{fmfgraph*}
    \caption*{\F{22}}
    \end{subfigure}
    \begin{subfigure}{0.45\textwidth}
      \centering
      \begin{fmfgraph*}(60,60)
        \fmfleft{i1,i2}
        \fmfright{o}
        \fmf{fermion}{i1,v,i2}
        \fmf{photon}{v,o}
        \fmflabel{$\bar{f}$}{i1}
        \fmflabel{$f$}{i2}
        \fmflabel{$Z_\mu$}{o}
      \end{fmfgraph*}
    \caption*{\F{23}}
    \end{subfigure} \\
    \vspace{12mm}
    \begin{subfigure}{0.45\textwidth}
      \centering
      \begin{fmfgraph*}(60,60)
        \fmfleft{i1,i2}
        \fmfright{o}
        \fmf{fermion}{i1,v,i2}
        \fmf{dashes}{v,o}
        \fmflabel{$\bar{f}$}{i1}
        \fmflabel{$f$}{i2}
        \fmflabel{$H$}{o}
      \end{fmfgraph*}
    \caption*{\F{24}}
    \end{subfigure}
    \begin{subfigure}{0.45\textwidth}
      \centering
      \begin{fmfgraph*}(60,60)
        \fmfleft{i}
        \fmfright{o}
        \fmf{fermion,label=$f$}{i,o}
      \end{fmfgraph*}
    \caption*{\F{25}}
    \end{subfigure}
  \end{fmffile}
  \caption{Some fermion-boson interactions and the fermion propagator.}
\end{figure}

For the fermionic Higgs sector let
\begin{equation}
  \label{eq:ktilde}
  \widetilde{K}=-\frac{1}{\sqrt{2}}
  \left(\begin{array}{c}
    \sqrt{2}i\phi^+ \\
    H+2M_{W}/g-i\phi^0
  \end{array}\right).
\end{equation}
This part of the SM Lagrangian is
\begin{equation}
  \label{eq:lsferm}
  \Lagr_S^{ferm}=-\alpha_f \overline{\Psi}_L Ku_R
                 -\beta_f \overline{\Psi}_L\widetilde{K}d_R+\text{h.c.},
\end{equation}
where $\alpha_f$ and $\beta_f$ are Yukawa couplings. Thus we see
\eqref{eq:ktilde} allows us to give masses to down type fermions. Plugging
\eqref{eq:ktilde} into \eqref{eq:lsferm} and using equations like
\begin{equation}
  \bar{u}_R d_L=\frac{\bar{u}}{2}(1+\gamma_5)d, \qquad
  \bar{u}_L d_R=\frac{\bar{u}}{2}(1-\gamma_5)d \qquad
\end{equation}
we obtain
\begin{equation}
  \Lagr_S^{ferm}=-\frac{\alpha_f}{\sqrt{2}}\Bigg(\frac{2M_W}{g}\Bigg)\bar{u}u
                -\frac{\beta_f}{\sqrt{2}}\Bigg(\frac{2M_W}{g}\Bigg)\bar{d}d
                +...
\end{equation}
from which we see
\begin{equation}
  \alpha_f=\frac{gm_u}{\sqrt{2}M_W}, \qquad
  \alpha_f=-\frac{gm_d}{\sqrt{2}M_W}.
\end{equation}
With this identification we can write
\begin{equation}
  \label{eq:sferm}
  \Lagr_S^{ferm}=-\sum\limits_f 
          \Bigg(m_f+\frac{1}{2}gH\frac{m_f}{M_W}\Bigg)\bar{f}f+...
\end{equation}
whose second term gives the Feynman rule
\begin{gather*}
  -\frac{igm_f}{2M_W}~~\F{24}
\end{gather*}
Also note that from \eqref{eq:vfer} and
\eqref{eq:sferm} we have the terms necessary to build the propagator for
any fermion
\begin{gather*}
  \frac{i(\slashed{p}+m_f)}{p^2-m_f^2+i\epsilon}~~\F{25}
\end{gather*}

To complete the SM Lagrangian, at least to the extent necessary to do this
exam, we briefly cover the QCD sector. Let $a,b,c=1,...,8$; $i,j=1,2,3$.
Then the first two pieces of the QCD Lagrangian read
\begin{gather}
  \Lagr_{gauge}^{QCD}=-\frac{1}{2}\partial_\nu G_\mu^a\partial^\nu G^{a\mu}
           -g_S f^{abc}\partial_\mu G_\nu^aG^{b\mu}G^{c\nu}
           -\frac{1}{4}g_S^2 f^{abc}f^{ade}G_\mu^bG_{\nu}^cG^{d\mu}G^{e\nu} \\
  \Lagr_{ferm}^{QCD}=\frac{1}{2}ig_S\sum\limits_\sigma
         \big(\bar{q}_i^\sigma\gamma^\mu\lambda_{ij}^aq_j^\sigma\big)G^a_\mu,
          \nonumber
\end{gather}
where the sum in $\Lagr_{ferm}^{QCD}$ runs over quark types. There will also
need to be a gauge fixing term and a ghost field, but we do not need to
bother with these details for this final. From $\Lagr_{ferm}^{QCD}$ we can
read off the last needed Feynman rule
\begin{gather*}
  ig_S\big(\gamma^\mu\big)_{\beta\alpha}G^a_{ij}~~\F{26}
\end{gather*}
Adding together all the mentioned pieces gives the SM Lagrangian.
\begin{figure}
  \begin{fmffile}{fr26}
  \centering
    \begin{subfigure}{\textwidth}
      \centering
      \begin{fmfgraph*}(60,60)
        \fmfleft{i1,i2}
        \fmfright{o}
        \fmf{fermion}{i1,v,i2}
        \fmf{gluon}{v,o}
        \fmflabel{$\alpha,j$}{i1}
        \fmflabel{$\beta,i$}{i2}
        \fmflabel{$\mu,a$}{o}
      \end{fmfgraph*}
    \caption*{\F{26}}
    \end{subfigure}
  \end{fmffile}
  \caption{Quark-gluon vertex.}
\end{figure}

\section{Some discussion of Higgs decays}
{\it The interactions between the Higgs boson and the EW gauge bosons were
derived in the previous rules. In the case $M_{H}\approx\text{125 GeV}$, we
have $M_{H}<2M_W$ and $M_{H}<2M_Z$. But the branching ratios $\Br(H\to WW)$ and
$\Br(H\to ZZ)$ are nonzero in this range. What do they correspond to? How would
you calculate these rates?}\vspace{5mm}

If $M_H<2M_W$ or $M_H<2M_Z$ then one of the product particles must be
off-shell, which makes it virtual. Therefore this is not a decay we can
directly observe. Thankfully if the virtual particle decays into on-shell
particles, we can observe those. There are a few constraints:
\begin{enumerate}
  \item Whatever the final products are in $H\to ZX$ or $H\to WX$ (where $X$ is
        a currently unspecified collection of particles), they had better be
        kinematically allowed. Thus $M_H\geq m_{products}$.
  \item The decay of the virtual boson has to use valid Feynman rules.
  \item The decay products cannot be Goldstone or ghost particles because these
        are unphysical.
\end{enumerate}
Let us consider the two cases.

\underline{$H\to WW^*$}: Here $M_X\leq\text{45~GeV}$. Kinematically, $\gamma$,
$g$, and all fermions except $t$ are allowed. We have to eliminate $\gamma$
because there is no coupling to $W^*$ that does not involve another $W$ or $Z$.
$W^*$ does not couple to $g$, so it is also eliminated. Therefore at tree level,
the only contributing diagrams are the ones in Figure 6a, where
$|Q_{f_1}-Q_{f_2}|=1$, $f_1$ and $f_2$ have different flavors and the same
lepton number. By \F{12}, \F{27}, and \F1, the value of this diagram is
\begin{equation}
  i\ME_{f_1,f_2}=i\frac{g}{\sqrt{2}}\gamma^\mu
    \Bigg(\frac{1-\gamma_5}{2}\Bigg)\Bigg(\frac{-ig_{\mu\nu}}{k^2-M_W^2}\Bigg)
       \big(igM_Z\st^2\big)\bar{u}_{s_2}(\vec{p}_2)v_{s_1}(\vec{p}_1)
       g^{\nu\rho}\epsilon^*_\rho(p_W).
\end{equation}
Hence,
\begin{equation}
  i\ME=i\sum\limits_{f_1,f_2}\ME_{f_1,f_2},
\end{equation}
where the sum runs over all $f_1$ and $f_2$ subject to the above restrictions.

\underline{$H\to ZZ^*$}: The $Z$-boson is not much heavier than $W$, and it
cannot
decay directly into $g$ or $\gamma$ for the same reasons as before. At tree
level, the only relevant Feynman diagrams are the ones in Figure 6b because
the $Z$-boson changes neither color nor flavor. Therefore
\begin{equation}
  i\ME=\sum\limits_f\left(\frac{ig}{2\ct}\gamma^\mu\big(I_f^{(3)}
                 -2Q_f\st^2+I_f^{(3)}\gamma_5\big)\frac{-ig_{\mu\nu}}
                 {k^2-M_Z^2}\Bigg(\frac{ig}{\ct}M_Zg^{\nu\rho}\Bigg)
                 \epsilon^*_\rho(p_Z)\bar{u}_{s_2}(\vec{p}_2)
                 v_{s_1}(\vec{p}_1)\right)
\end{equation}
\begin{figure}
  \setlength{\abovecaptionskip}{15pt plus 3pt minus 2pt}
  \begin{fmffile}{prob1}
  \centering
    \begin{subfigure}{.45\textwidth}
    \setlength{\abovecaptionskip}{25pt plus 3pt minus 2pt}
      \centering
      \begin{fmfgraph*}(90,60)
        \fmfleft{i}
        \fmfright{o1,o2,o3}
        \fmf{dashes,tension=2}{i,v1}
        \fmf{photon}{v1,o1}
        \fmf{photon,label=$W^*$}{v1,v2}
        \fmf{fermion,tension=1}{o3,v2,o2}
        \fmflabel{$H$}{i}
        \fmflabel{$W$}{o1}
        \fmflabel{$f_2$}{o2}
        \fmflabel{$f_1$}{o3}
      \end{fmfgraph*}
     \caption{}
    \end{subfigure}
    \begin{subfigure}{.45\textwidth}
    \setlength{\abovecaptionskip}{25pt plus 3pt minus 2pt}
      \centering
      \begin{fmfgraph*}(90,60)
        \fmfleft{i}
        \fmfright{o1,o2,o3}
        \fmf{dashes,tension=2}{i,v1}
        \fmf{photon}{v1,o1}
        \fmf{photon,label=$Z^*$}{v1,v2}
        \fmf{fermion,tension=1}{o3,v2,o2}
        \fmflabel{$H$}{i}
        \fmflabel{$Z$}{o1}
        \fmflabel{$f$}{o2}
        \fmflabel{$\bar{f}$}{o3}
      \end{fmfgraph*}
    \caption{}
    \end{subfigure}
  \end{fmffile}
  \caption{Off-shell Higgs decays into (a) $WW^*$ and (b) $ZZ^*$.}
\end{figure}

{\it Calculate $\Gamma(H\to Q\overline{Q})$, where $Q$ is an arbitrary quark.}
\vspace{5mm}

We work in the rest frame of the Higgs. For a general particle decay in this
frame we have \cite[p.107]{peskin_introduction_1995}
\begin{equation}
  \text{dLIPS}=\frac{d\Omega}{16\pi^2}\frac{|\vec{p}|}{E_{CM}},
\end{equation}
where $E_{CM}$ is the total initial energy and $\vec{p}$ is the spatial
momentum of either of the final state particles. Therefore the decay rate
$\Gamma(H\to\text{2 particles})$ is generally
\begin{equation}
  \label{eq:gam}
  \Gamma=\frac{1}{2m_H}\int\frac{d\Omega}{16\pi^2}
             \frac{|\vec{p}|}{m_H}|\ME|^2
        =\frac{4\pi}{16\pi^2 m_H^2}|\ME|^2|\vec{p}|
        =\frac{|\vec{p}||\ME|^2}{8\pi m_H^2}.
\end{equation}
In the present case, the contributing diagram is \F{24}. Therefore the matrix
element is
\begin{equation}
  i\ME=-\frac{ig}{2}{m_Q}{M_W}\bar{u}_s(\vec{p})v_{s'}(\vec{p}').
\end{equation}
Summing over final spin states and keeping in mind that the quark-antiquark
pair has three possible color-anticolor combinations, we get by the
completeness relation
\begin{equation}
  \label{eq:ME2}
  \begin{aligned}
    \left< |\ME|^2 \right>&=3\frac{g^2 m_Q^2}{4M_W^2}\sum\limits_{s,s'}
      \Tr u_s(\vec{p})\bar{u}_s(\vec{p})
          v_{s'}(\vec{p}')\bar{v}_{s'}(\vec{p}') \\
    &=\frac{3g^2 m_Q^2}{4M_W^2}\Tr(\slashed{p}+m_Q)(\slashed{p}'-m_Q) \\
    &=\frac{3g^2 m_Q^2}{4M_W^2}\Tr(\slashed{p}\slashed{p}'-m_Q^2) \\
    &=\frac{3g^2 m_Q^2}{M_W^2}\Tr(pp'-m_Q^2) \\
    &=\frac{3g^2 m_Q^2}{M_W^2}\Bigg(-\frac{s}{2}-m_Q^2-m_Q^2\Bigg) \\
    &=\frac{3g^2 m_Q^2}{M_W^2}\Bigg(-\frac{m_H^2}{2}-2m_Q^2\Bigg) \\
    &=\frac{3g^2 m_Q^2 m_H^2}{M_W^2}\Bigg(1-\frac{4m_Q^2}{m_H^2}\Bigg).
  \end{aligned}
\end{equation}
Combining \eqref{eq:gam} and \eqref{eq:ME2} yields
\begin{equation}
  \label{eq:gamQ}
  \begin{aligned}
    \Gamma&=\frac{1}{8\pi m_H^2}{m_H}\Bigg(1-\frac{4m_Q^2}{m_H^2}\Bigg)^{1/2}
            \frac{3g^2m_Q^2m_H^2}{M_W^2}\Bigg(1-\frac{4m_Q^2}{m_H^2}\Bigg)\\
          &=\frac{3g^2m_Q^2m_H}{8\pi M_W^2}
               \Bigg(1-\frac{4m_Q^2}{m_H^2}\Bigg)^{3/2}.
  \end{aligned}
\end{equation}

{\it The Higgs can also decay into gluons, $H\to gg$. This cannot happen at
tree level, but it can happen at the one-loop level. How? Calculate
$\Gamma(H\to gg)$. You can leave your result in a form that depends on a
Feynman parameter integral.}
\vspace{5mm}

\begin{figure}
  \begin{fmffile}{prob2}
    \centering
    \begin{subfigure}{.95\textwidth}
    \centering
    \begin{fmfgraph*}(90,60)
      \fmfleft{i}
      \fmfright{o1,o2}
      \fmf{dashes,tension=2,label=$p+k$}{i,v1}
      \fmf{fermion}{v1,v3}
      \fmf{fermion,tension=0,label=$l$}{v3,v2}
      \fmf{fermion}{v2,v1}
      \fmf{photon,label=$k$,tension=2}{v2,o1}
      \fmf{photon,label=$p$,tension=2}{v3,o2}
      \fmflabel{$H$}{i}
      \fmflabel{$g$}{o1}
      \fmflabel{$g$}{o2}
    \end{fmfgraph*}
    \end{subfigure}
  \end{fmffile}
  \vspace{8mm}
  \caption{Gluon production via quark loop.}
\end{figure}

At the one loop level, the Higgs can decay via the mechanism of Figure 7,
which is determined by rules \F{24}, \F{25}, and \F{26}. There is also a
crossed diagram whose amplitude should equal the amplitude of the first by
symmetry, which cancels the factor of 2 due to \F{24}. Hence the matrix
element for this process is given by

\begin{gather}
  i\ME=\frac{igm_Q}{M_W}\int\frac{d^4l}{(2\pi)^4}\Tr\left[
        \frac{i(\slashed{l}-\slashed{k}+m_Q)}{(l-k)^2-m_Q^2}ig_S\gamma^\mu
        \frac{i(\slashed{l}+m_Q)}{l^2-m_Q^2}ig_S\gamma^\nu
        \frac{i(\slashed{l}+\slashed{p}+m_Q)}{(l+p)^2-m_Q^2}\right] \\
        \times\Tr\big[G^aG^b\big]\epsilon^{\lambda '}_\mu(k)
                           \epsilon^{\lambda }_\nu(p).
\end{gather}
The numerator can be simplified using Form (see Appendix A) to
\begin{equation}
  \label{eq:3num}
  ig_S^2N^{\mu\nu}=4im_Qg_S^2\big(p^\mu k^\nu-k^\mu p^\nu+2l^\mu k^\nu
                     -2k^\mu l^\nu+4l^\mu l^\nu-g^{\mu\nu}pk-g^{\mu\nu}l^2
                     +g^{\mu\nu}m_Q^2\big).
\end{equation}
Meanwhile we can rewrite the denominator using Feynman parameters, $p^2=k^2=0$,
and $pk=m_H^2/2$ as
\begin{equation}
  \frac{1}{\big((l-k)^2-m_Q^2\big)\big(l^2-m_Q^2\big)\big((l+p)^2-m_Q^2\big)}
    =\int_0^1dx\int_o^{1-x}dy\frac{2}{\big(q^2-D\big)^3},
\end{equation}
where
\begin{equation}
  q^\mu\coloneqq l^\mu-\big(xk^\mu-yp^\mu\big),\qquad D\coloneqq m_Q^2-xym_H^2.
\end{equation}
We arrived at these definitions by completing the square; namely
\begin{equation}
  \begin{aligned}
    l^2-2l(xk-yp)&=(l-(xk-yp))^2-(xk-yp)^2 \\
                 &=q^2+2xypk \\
                 &=q^2+xym_H^2.
  \end{aligned}
\end{equation}

Now in order to integrate \eqref{eq:3num}, we need to express $N^{\mu\nu}$
in terms of $q^\mu$. Since the denominator is a function of $q^2$, terms
proportional to $q^\mu$ will vanish. Furthermore polarization vectors are
perpendicular to the direction of gluon propagation, and $\vec{p}_1$ and
$\vec{p}_2$ must be antiparallel by momentum conservation, so
$\epsilon^*(p_i)\cdot p_j=0$ with $i,j=1,2$. Let $N'^{\mu\nu}$ represent
$N^{\mu\nu}$ with these terms dropped. Then
\begin{equation}
  \begin{aligned}
    N'^{\mu\nu}&=4m_Q\Big(4l^\mu l^\nu-l^2g^{\mu\nu}
                    +g^{\mu\nu}\big(m_Q^2-pk\big)\Big) \\
               &=4m_Q\Bigg(\Big(m_Q^2
                    +\Big(xy-\tfrac{1}{2}\Big)m_H^2-q^2\Big)g^{\mu\nu}
                    +4q^\mu q^\nu\Bigg).
  \end{aligned}
\end{equation}
\begin{equation}
  \begin{aligned}
    N'^{\mu\nu}&=4m_Q\Big(4l^\mu l^\nu-l^2g^{\mu\nu}
                    +g^{\mu\nu}\big(m_Q^2-pk\big)\Big) \\
               &=4m_Q\Bigg(\Big(m_Q^2
                    +\Big(xy-\tfrac{1}{2}\Big)m_H^2-q^2\Big)g^{\mu\nu}
                    +4q^\mu q^\nu\Bigg).
  \end{aligned}
\end{equation}
Since $l$ and $q$ differ additively by terms independent of $l$ or $q$,
$d^4l=d^4q$. Furthermore $\Tr G^aG^b=\delta^{ab}/2$, so
\begin{equation}
  \label{eq:53}
  i\ME=-\frac{gg_S^2m_Q}{M_W}\delta^{ab}\epsilon^{\lambda'*}_\mu(k)
           \epsilon^{\lambda*}_\nu(p)\int\frac{d^4q}{(2\pi)^4}
           \int_0^1dx\int_0^{1-x}dy\frac{N'^{\mu\nu}}{\big(q^2-D\big)^3}.
\end{equation}
Under the $d^4q$ integral, the $q^\mu q^\nu$ term gets replaced by
$g^{\mu\nu}q^2/d$. One might think that this would cancel the other $q^2$
term, but in fact the integral is more subtle, and we will have to
employ dimensional regularization to compute it. Replacing
$d\to d-2\epsilon$ we have by \cite[A.45]{peskin_introduction_1995}
\begin{equation}
  \label{eq:54}
  \begin{aligned}
    \int\frac{d^{4-2\epsilon}q}{(2\pi)^{4-2\epsilon}}
        \frac{q^2}{\big(q^2-D\big)^3}\Bigg(\frac{4}{4-2\epsilon}-1\Bigg)
     &=\frac{(-1)^2i}{(4\pi)^{2-\epsilon}}\Bigg(\frac{4-2\epsilon}{2}\Bigg)
        \frac{\Gamma(\epsilon)}{\Gamma(3)}\Bigg(\frac{1}{D}\Bigg)^\epsilon
        \Bigg(\frac{4}{4-2\epsilon}-1\Bigg) \\
     &\approx\Bigg(\frac{4\pi}{D}\Bigg)^\epsilon\frac{i}{(4\pi)^2}
        \Bigg(1-\frac{\epsilon}{2}\Bigg)\Gamma(\epsilon)
        \Bigg(1+\frac{\epsilon}{2}+\mathcal{O}\big(\epsilon^2\big)-1\Bigg) \\
     &\approx\frac{i}{(4\pi)^2}\Big(1+\mathcal{O}\big(\epsilon\big)\Big)
        \Bigg(1-\frac{\epsilon}{2}\Bigg)\Bigg(\frac{1}{\epsilon}-\gamma
        +\mathcal{O}\big(\epsilon\big)\Bigg)\frac{\epsilon}{2} \\
     &\to\frac{i}{2(4\pi)^2}
  \end{aligned}
\end{equation}
in the $\epsilon\to 0$ limit. (Here $\gamma$ is the Euler-Mascheroni constant.)
Meanwhile for the integral over the constant terms, we have by
\cite[A.44]{peskin_introduction_1995}
\begin{equation}
  \label{eq:55}
  \int\frac{d^4q}{(2\pi)^4}\frac{1}{\big(q^2-D\big)^3}
      =\frac{(-1)^3i}{(4\pi)^2}\frac{\Gamma(1)}{\Gamma(3)}\frac{1}{D}
      =\frac{-i}{2(4\pi)^2D}.
\end{equation}
Combining \eqref{eq:53}, \eqref{eq:54}, and \eqref{eq:55} we arrive at
\begin{equation}
  \label{eq:56}
  \begin{aligned}
    i\ME&=-\frac{gg_S^2m_Q}{M_W}\delta^{ab}\epsilon^{\lambda'*}_\mu(k)
             \epsilon^{\lambda*}_\nu(p)\frac{4m_Qi}{(4\pi)^2} \\
        &~~~~~~~~~~~~\times\int_0^1dx\int_0^{1-x}dy\frac{1}{D}
             \Bigg[\frac{1}{2}\Big(m_Q^2-xym_H^2\Big)
              -\frac{1}{2}\Big(m_Q^2+\big(xy-\tfrac{1}{2}\big)m_H^2\Big)\Bigg]
             \\
        &=-\frac{4igg_S^2m_Q^2}{(4\pi)^2M_W}\delta^{ab}
               \epsilon^{\lambda'*}_\mu(k)\epsilon^{\lambda*}_\nu(p)
               \int_0^1dx\int_0^{1-x}dy
               \frac{m_H^2\big(\tfrac{1}{4}-xy\big)}{m_Q^2-xym_H^2} \\
        &=-\frac{igg_S^2m_H^2}{(4\pi)^2M_W}\delta^{ab}
               \epsilon^{\lambda'*}_\mu(k)\epsilon^{\lambda*}_\nu(p)
               I\Big(\frac{m_H}{m_Q}\Big),  
  \end{aligned}
\end{equation}
where
\begin{equation}
  I(z)\coloneqq\int_0^1dx\int_0^{1-x}dy\frac{1-4xy}{1-xyz^2}.
\end{equation}
Finally taking the magnitude of \eqref{eq:56}, summing over color indices,
and summing over the two possible polarizations $+-$ and $-+$ we obtain
\begin{equation}
  \label{eq:58}
  \big<|\ME|^2\big>=\frac{g^2g_S^4m_H^4}{(4\pi)^4M_W^2}(8)(2)
      \Big|I\Big(\frac{m_H}{m_Q}\Big)\Big|^2.
\end{equation}
Therefore for a generic quark we have by \eqref{eq:gam} and \eqref{eq:58}
\begin{equation}
  \label{eq:59}
  \Gamma_Q=\frac{1}{2}m_H\frac{1}{8\pi m_H^2}
            \frac{g^2 g_S^4m_H^4}{(4\pi)^4 M_W^2}(8)
            \Big|I\Big(\frac{m_H}{m_Q}\Big)\Big|^2
          =\frac{2g^2 g_S^4m_H^3}{(4\pi)^5 M_W^2}
            \Big|I\Big(\frac{m_H}{m_Q}\Big)\Big|^2.
\end{equation}
The total decay width can then be found by summing \eqref{eq:56} over all
quarks. Hence we ultimately find for the decay rate of $H\to gg$
\begin{equation}
  \Gamma=\frac{2g^2 g_S^4m_H^3}{(4\pi)^5 M_W^2}
          \Big|\sum\limits_Q I\Big(\frac{m_H}{m_Q}\Big)\Big|^2.
\end{equation}



\bibliographystyle{unsrtnat}
\bibliography{bibliography}
